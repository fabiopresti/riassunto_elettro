\documentclass[
10pt, % Main document font size
a4paper, % Paper type, use 'letterpaper' for US Letter paper
oneside, % One page layout (no page indentation)
%twoside, % Two page layout (page indentation for binding and different headers)
headinclude,footinclude, % Extra spacing for the header and footer
BCOR5mm, % Binding correction
]{scrartcl}

\usepackage[
nochapters, % Turn off chapters since this is an article        
beramono, % Use the Bera Mono font for monospaced text (\texttt)
eulermath,% Use the Euler font for mathematics
pdfspacing, % Makes use of pdftex’ letter spacing capabilities via the microtype package
dottedtoc % Dotted lines leading to the page numbers in the table of contents
]{classicthesis} % The layout is based on the Classic Thesis style

\usepackage{arsclassica} % Modifies the Classic Thesis package

\usepackage[T1]{fontenc} % Use 8-bit encoding that has 256 glyphs

\usepackage[utf8]{inputenc} % Required for including letters with accents

\usepackage{graphicx} % Required for including images
\graphicspath{{Figures/}} % Set the default folder for images

\usepackage{enumitem} % Required for manipulating the whitespace between and within lists

\usepackage{lipsum} % Used for inserting dummy 'Lorem ipsum' text into the template

\usepackage{subfig} % Required for creating figures with multiple parts (subfigures)

\usepackage{amsmath,amssymb,amsthm} % For including math equations, theorems, symbols, etc
\usepackage{placeins}
\usepackage{graphicx}
\usepackage[italian]{babel}
\usepackage{varioref}

\theoremstyle{plain}
\newtheorem{teorema}{teorema}
\theoremstyle{plain}
\newtheorem{esercizio}{esercizio}
\theoremstyle{plain}
\newtheorem{applicazione}{applicazione}
\theoremstyle{plain}
\newtheorem{corollario}{corollario}[teorema]
\theoremstyle{plain}
\newtheorem{lemma}[teorema]{lemma}
\theoremstyle{definition}
\newtheorem{definizione}{definizione}
\theoremstyle{plain}
\newtheorem{proposizione}{proposizione}
\makeatletter
\def\d{\ensuremath\partial}
\def\d{\ensuremath\partial\ }
\def\n{\ensuremath\nabla}
\def\n{\ensuremath\nabla\ }
\def\th@plain{%
	\thm@notefont{}% same as heading font
	\itshape % body font
}
\def\th@definizione{%
	\thm@notefont{}% same as heading font
	\normalfont % body font
}
\hypersetup{
colorlinks=true, breaklinks=true, bookmarks=true,bookmarksnumbered,
urlcolor=webbrown, linkcolor=RoyalBlue, citecolor=webgreen, % Link colors
pdftitle={}, % PDF title
pdfauthor={\textcopyright}, % PDF Author
pdfsubject={}, % PDF Subject
pdfkeywords={}, % PDF Keywords
pdfcreator={pdfLaTeX}, % PDF Creator
pdfproducer={LaTeX with hyperref and ClassicThesis} % PDF producer
} % Include the structure.tex file which specified the document structure and layout

\hyphenation{Fortran hy-phen-ation} % Specify custom hyphenation points in words with dashes where you would like hyphenation to occur, or alternatively, don't put any dashes in a word to stop hyphenation altogether

%----------------------------------------------------------------------------------------
%	TITLE AND AUTHOR(S)
%----------------------------------------------------------------------------------------

%\title{\normalfont\spacedallcaps{Fenomeni termici}} % The article title

%\subtitle{Subtitle} % Uncomment to display a subtitle

%\author{\spacedlowsmallcaps{Fabio Prestipino}} % The article author(s) - author affiliations need to be specified in the AUTHOR AFFILIATIONS block

%\date{Maggio 2022} % An optional date to appear under the author(s)

%----------------------------------------------------------------------------------------
\begin{document}
	\begin{titlepage}
		\begin{center}
			\vspace*{1cm}
			
			\Huge{\normalfont\spacedallcaps{Elettromagnetismo}} % The article title
			
			\vspace{0.5cm}
			
			\vspace{1.5cm}
			
			\Large{\spacedlowsmallcaps{Fabio Prestipino}}
			
			\vspace{3.5cm}
			\LARGE{Un semplice riassunto per\\
				\textit{"sentire la musica"}}\\
			
			\vfill
			
			\vspace{2cm}
			\Large
			Fisica\\
			Università di Bologna\\
			settembre-dicembre 2022
			
		\end{center}
	\end{titlepage}
	
	%----------------------------------------------------------------------------------------
	%	HEADERS
	%----------------------------------------------------------------------------------------
	
	\renewcommand{\sectionmark}[1]{\markright{\spacedlowsmallcaps{#1}}} % The header for all pages (oneside) or for even pages (twoside)
	%\renewcommand{\subsectionmark}[1]{\markright{\thesubsection~#1}} % Uncomment when using the twoside option - this modifies the header on odd pages
	\lehead{\mbox{\llap{\small\thepage\kern1em\color{halfgray} \vline}\color{halfgray}\hspace{0.5em}\rightmark\hfil}} % The header style
	
	\pagestyle{scrheadings} % Enable the headers specified in this block
	
	%----------------------------------------------------------------------------------------
	%	TABLE OF CONTENTS & LISTS OF FIGURES AND TABLES
	%----------------------------------------------------------------------------------------
	
	%\maketitle % Print the title/author/date block
	
	\setcounter{tocdepth}{2} % Set the depth of the table of contents to show sections and subsections only
	
	
	
	%----------------------------------------------------------------------------------------
	%	AUTHOR AFFILIATIONS
	%----------------------------------------------------------------------------------------
	
	%\let\thefootnote\relax\footnotetext{* \textit{Dipartimento di Fisica, Università di Bologna}}
	
	%----------------------------------------------------------------------------------------
	
	\newpage % Start the article content on the second page, remove this if you have a longer abstract that goes onto the second page
	\tableofcontents % Print the table of contents
	\newpage
	%----------------------------------------------------------------------------------------
	%	INTRODUCTION
	%----------------------------------------------------------------------------------------

\section{Elettrostatica di cariche puntiformi}
La parola elettrone deriva dal greco elektron che vuol dire ambra, un materiale facilmente elettrizzabile. Sappiamo che già i greci avevano fatto esperienza del fenomeno dell' "elettrzzazione per strofinio" o "triboelettricità": strofinando due barre di vetro con un tessuto si avrà esperienza di una forza repulsiva fra di esse; se si strofinano due sbarre di plastica con delle pelli si otterrà lo stesso risultato ma se si avvicina una barra di vetro con una di plastica opportunamente sfregate queste si attrarranno. Dall'antica Grecia fino al XVII secolo questo fenomeno resta inspiegato e non si verificano sostanziali passi avanti nelle scienze; dopo l'introduzione del metodo scientifico si razionalizza quanto sperimentato traendo le seguenti conclusioni:

\begin{itemize}
	\item Esiste una forza diversa da quella gravitazionale (sicuramente trascurabile tra due barre)
	\item Questa forza può essere sia attrattiva che repulsiva
	\item Esiste un'attributo che la materia può avere o non avere (e che talvolta si nota può assumere per sfregamento) che genera questa attrazione e che chiameremo "carica" Q.
\end{itemize}
Seguendo il metodo scientifico, vogliamo fare esperimenti che offrano dati numerici relativi a queste esperienze. Il primo fondamentale strumento è l'\textbf{elettroscopio} che permette di comparare la carica di due corpi. Un classico tipo di elettroscopio è quello a foglie d'oro: un'ampolla all'interno del quale è applicato il vuoto presenta all'interno due foglie d'oro estremamente sottili e leggere attaccate ad una barretta metallica che fuoriesce dall'ampolla e che termina con una sfera metallica; se viene avvicinato un corpo carico alla sfera questa si caricherà e distribuirà la carica su tutto il corpo metallico, le due foglie metalliche, essendo entrambe cariche dello stesso segno si respingeranno tanto più quanto più forte è la carica con cui sono state elettrizzate.
\begin{figure}[h!]
	\centering
	\includegraphics[width=0.4\linewidth]{../images/elettroscopio}
	\caption{Schema di uno dei primi tipi di elettroscopio: quello a foglia d'oro.}
	\label{fig:elettroscopio}
\end{figure}
\FloatBarrier
L'elettroscopio (come il termoscopio) non serve per avere un valore quantitativo della carica ma permette di comparare cariche diverse, per avere un valore numerico della carica serve conoscere la legge che regola la forza elettrostatica. Tuttavia possiamo fornire una definizione operativa di due cariche uguali mediante l'elettroscopio nel modo seguente:\\
\textit{Due corpi possiedono la stessa carica se, posti a contatto con un elettroscopio scarico, generano una deflessione delle foglie di un'angolo uguale}\\
In questo modo tuttavia non si riconosce se due cariche sono dello stesso segno o meno, possiamo verificarlo semplicemente ponendo a contatto un corpo carico ad un elettroscopio scarico e poi porre il secondo a contatto con l'elettroscopio carico: se l'angolo aumenta la carica dei due corpi è uguale, se diminuisce è diversa.\\
Possiamo fare tre ulteriori riflessioni sulla carica:
\begin{itemize}
	\item La carica è quantizzata, la più piccola osservata sperimentalmente è quella dell'elettrone. Nonostante si ipotizzi che i quark abbiano cariche frazionarie, queste non sono mai state osservate singolarmente
	\item La carica totale si conserva
	\item La carica non dipende dal S.I.
\end{itemize}
In queste prime sezioni tratteremo dell'elettrostatica, cioè quella branca dell'elettromagnetismo che studia sistemi all'\textbf{equilibrio elettrostatico}.
\begin{definizione}[equilibrio elettrostatico]
	L'equilibrio elettrostatico è la condizione per cui tutte le cariche presenti su di un conduttore sono ferme, è quindi assente moto di carica.
\end{definizione}
\subsection{La legge di Coulomb}
Un grande passo avanti venne fatto da Coulomb (1736-1806) che mediante esperimenti con una bilancia a torsione ottenne un metodo preciso per calcolare numericamente la forza elettrostatica e mediante al quale riuscì a ricavare l'omonima legge.\\
La bilancia di Coulomb, analoga a quella di Cavendish usata per la misura della costante di gravitazione universale, è basata sulla torsione di un filo di costante elastica c nota a causa dell'interazione elettrostatica di due corpi carichi. In particolare questa è formata da una barra sospesa mediante un filo alle estremità della quale è presente una sfera metallica ed un contrappeso, vi si avvicina un'altra carica a distanza R e si misura l'angolo a cui si dispone la prima sfera a causa della repulsione e la nuova distanza fra le due sfere. Sappiamo che il modulo del momento generato dalla torsione è
\[\tau = c\theta\]
si ha inoltre che il momento è dato dal prodotto vettoriale delle forze applicate sulla sfera.
\[\tau = |\mathbf{r}\wedge \mathbf{F}|\]
dove r è la distanza dall'origine al punto in cui sono applicate le forze che, per come abbiamo costruito l'apparato e scelto il sistema di riferimento ha modulo pari a metà della lunghezza della barra : L/2. 
\begin{figure}[h!]
	\centering
	\includegraphics[width=0.6\linewidth]{../images/bilancia_coulomb}
	\caption{Schemi esplicativi della bilancia di Coulomb, con cui si ottenne la legge di Coulomb.}
	\label{fig:bilanciacoulomb}
\end{figure}
\FloatBarrier
All'equilibrio le due espressioni del momento si possono eguagliare ottenendo 
\[\frac{L}{2}|\mathbf{F}|\sin\phi = c\theta\]
\[|\mathbf{F}| = \frac{2c\theta}{L\sin\phi}\] 
Possiamo trarre tre conseguenze da quanto ottenuto:
\begin{itemize}
	\item Essendo l'angolo positivo o negativo a seconda che ci sia repulsione o attrazione, $\mathbf{F}$ può avere segno negativo o positivo ed essere quindi attrattiva o repulsiva. 
	\item La forza è diretta lungo la congiungente delle cariche (versore che chiameremo $\hat{\mathbf{r}}$)
	\item A seguito di molti esperimenti si può ricavare una formula generale che è 
	\[\mathbf{F} = k \frac{q_1q_2}{r^2}\hat{\mathbf{r}}\]
	dove k è una costante da determinare, \(q_1\) e \(q_2\) sono le cariche ed r la distanza fra di esse. 
\end{itemize}
Quest'ultima è la formalizzazione della legge di Coulomb che si enuncia come segue:\\
\textit{"Date due cariche poste ad una distanza r, tra di esse si esercita una forza direttamente proporzionale al prodotto delle cariche ed inversamente proporzionale al quadrato della distanza r. Tale forza è diretta lungo la congiungente delle cariche ed è repulsiva se le cariche sono dello stesso segno e attrattiva se di segno opposto."}\\
Si noti che la forma matematica di questa legge è analoga a quella di gravitazione universale poiché entrambe le forze sono "centrali". Nella forza di Coulomb non compare il segno negativo poiché non è sempre attrattiva come quella di gravitazione ma il segno dipende dal segno delle cariche.\\
Non resta che determinare la costante k; esistono due metodi filosoficamente molto diversi fra loro:
\begin{enumerate}
	\item Nel sistema CGS  k è una grandezza adimensionale pari ad 1. Ciò discende dal fatto che in questo sistema l'unità di misura di Q è una grandezza derivata definita come
	\[[Q] = [M]^{\frac{1}{2}}[L]^{-\frac{3}{2}}[T]^{-1}\]
	Da cui deriva che K deve essere adimensionale (in realtà la definizione di Q è funzionale al definire k come adimensionale).
	\item Nel sistema MKS (che adotteremo) l'unità di misura di Q è definita come fondamentale e prende il nome di "Coulomb" (C). Sapendo la relazione che intercorre tra F e k e conoscendo l'unità di misura di F, di Q e di r possiamo ricavare l'unità di misura di k
	\[[k] = [F][L]^2[Q]^{-2} = [M][L]^3[T]^{-2}[Q]^{-1}\]
	Poniamo per convenzione (la comodità di questa scelta sarà chiara in seguito)
	\[k \equiv \frac{1}{4\pi\varepsilon_0}\]
	Da cui possiamo riscrivere la legge di Coulomb come
	\[\mathbf{F} = \frac{1}{4\pi\varepsilon_0}\frac{q_1q_2}{r^2}\hat{\mathbf{r}}\]
\end{enumerate}
Si nota sperimentalmente che la forza esercitata dalla presenza di molteplici cariche è data dalla somma vettoriale delle forze.\\
Fino al 2019 il Coulomb era definito come la forza esercitata tra due fili a distanza di un metro nei quali passa 1 A di corrente elettrica. Recentemente si preferisce definire le grandezze fondamentali mediante costanti di natura, il Coulomb è stato definito mediante la carica dell'elettrone: la carica dell'elettrone equivale ad una piccolissima frazione di Coulomb: 
\[e^- \equiv 1.6021766208\cdot10^{-19} C\]
Possiamo paragonare la forza di gravità a quella elettrostatica facendone il rapporto
\[\frac{F_E}{F_G} = \frac{1}{4\pi\varepsilon_0 G}\frac{|q_1||q_2|}{m_1m_2}=\simeq 2\cdot10^{39}\]
La forza elettrostatica è preponderante su quella gravitazionale tanto da renderla, in teoria, trascurabile. Tuttavia nell'esperienza quotidiana percepiamo maggiormente la gravità poiché i corpi con cui solitamente abbiamo a che fare sono neutri e in generale, a causa della repulsione di cariche dello stesso segno, è difficile che si accumulino grandi cariche, a differenza del caso di grandi masse.
\subsection{Il campo elettrostatico}
Il fatto che avvicinando due cariche esse si attraggono con una forza porta ad ipotizzare che, anche se fosse presente un'unica carica q (detta sorgente), si formerebbe comunque un \textbf{campo elettrostatico}, ovvero una regione di spazio in cui, posta una carica \(q_0\) (detta di prova), per ogni punto dello spazio si può associare ad essa un vettore forza che sperimenterebbe.\\
Matematicamente l'espressione del campo elettrostatico si ottiene semplicemente aggiungendo una parentesi nell'espressione della forza 
\[\mathbf{F}=q_0\left[\frac{1}{4\pi\varepsilon_0}\frac{q}{r^2}\hat{\mathbf{r}}\right]\]
\[\mathbf{E} = \frac{\mathbf{F}}{q_0} = \frac{1}{4\pi\varepsilon_0}\frac{q}{r^2}\hat{\mathbf{r}}\]
Si noti il salto concettuale rispetto la teoria della gravità newtoniana: ora non si tratta di trovare le forze agenti su un corpo ma il campo vettoriale, in questo caso elettrico.
\begin{definizione}[Campo elettrico ed elettrostatico]
	 Il campo elettrico è un campo di forze generato nello spazio dalla presenza di una o più cariche elettriche. Il campo elettrostatico è il campo elettrico generato da cariche in quiete. L'interazione del campo elettrostatico con una carica di prova genera una forza sulla carica stessa e, uguale e contraria, sulle cariche che generano il campo.
\end{definizione}
Un campo vettoriale può essere pensato come una porzione di spazio piena di frecce rappresentanti i vettori che esso associa ad ogni punto; disegnarlo in tal modo è impossibile dunque si opta per disegnare pochi vettori sempre tangenti al campo stesso. Per farlo si disegna un primo vettore, da esso ci si sposta di un'intervallo infinitesimo in avanti e si traccia un nuovo vettore (la cui direzione e verso saranno quelle del campo), reiterando otterremo un'insieme di vettori, se li cancelliamo e lasciamo solamente i punti di applicazione otteniamo una linea detta "di campo". Il campo elettrostatico è un vettore perché direzione e verso si ricavano dai singoli vettori mentre il modulo è dato dalla densità delle linee di campo. Per convenzione le linee di campo sono uscenti dalle cariche positive ed entranti nelle cariche negative.\\
Si osserva sperimentalmente che il campo generato da un numero discreto di cariche puntiformi è la somma vettoriale dei singoli campi, questo viene detto \textbf{principio di sovrapposizione} ed è solamente frutto dell'esperienza. Ne risulta che il campo totale generato da N cariche puntiformi è
\[\mathbf{E_{tot}} = \frac{1}{4\pi\varepsilon_0}\sum_{i=1}^{N}\frac{Q_i}{ r_i^2} \hat{\mathbf{r}}_i\]
Se invece avessimo un insieme continuo di cariche, che è ciò che avviene nella maggior parte dei casi poiché il numero di cariche è solitamente estremamente elevato, dobbiamo innanzitutto definire la densità di carica
\begin{definizione}[Densità di carica]
Definiamo la densità volumetrica di carica come
\[\rho(\mathbf{r})=\lim_{\Delta V \to 0} \frac{\Delta q}{\Delta V} = \frac{dq}{dV}\]
Definiamo analogamente la densità superficiale e lineare. 
\end{definizione}
Si noti che il limite appena scritto non ha lo stesso significato che in matematica: il volume deve essere abbastanza piccolo da racchiudere una porzione di spazio abbastanza omogenea ma non troppo piccola da mettere in luce la struttura atomica della materia.\\
Possiamo ora calcolare il campo elettrostatico generato da un corpo di carica continua 
\[\mathbf{E}_{tot} = \frac{1}{4\pi\varepsilon_0}\int_{\tau}\frac{dq}{r^2}\hat{\mathbf{r}} = \frac{1}{4\pi\varepsilon_0} \int_V\frac{\rho(\mathbf{r})}{r^2}\hat{\mathbf{r}}d\tau\]
Quest'ultimo è un integrale di volume.
\begin{esercizio}[Campo elettrostatico generato da un filo infinito carico]
Si consideri un filo di lunghezza infinita, vogliamo ottenere un'espressione del campo elettrico che genera su un punto P a distanza r' dal filo. \\
Cominciamo considerando due porzioni infinitesime di filo di lunghezza dz e carica dq, poste a distanza uguale dall'origine (uno in verso positivo ed uno in verso negativo), tracciamo i vettori $\mathbf{r}$ congiungenti le porzioni di filo con il punto P e prolunghiamo i vettori per rappresentare il campo su quel punto; $\mathbf{r}$ incide sull'asse del punto con un angolo $\theta$. Da una prima approssimazione grafica, sommando i vettori, notiamo che la componente sull'asse z si elide e il campo risultante dovrà essere parallelo all'asse delle y e in verso positivo ($\hat{\mathbf{j}}$). Possiamo quindi focalizzarci sul calcolo del modulo
\[\mathbf{E}_{tot} = \int_L dE_y\hat{\mathbf{j}}\]
\[E_{tot} = \int^{+\infty}_{-\infty} \frac{1}{4\pi\varepsilon_0}\frac{\lambda dz}{r^2}\cos\theta\]
le variabili \(z, \theta, r\) sono correlate quindi dobbiamo esprimere l'integrale in funzione di una sola di queste, i calcoli si semplificano se scegliamo $\theta$. r' è semplicemente la proiezione sull'asse x di r quindi si ottiene immediatamente la relazione
\[r \cos\theta = r'\]
z invece è la proiezione sull'asse y di r, per ottenere dz differenziamo
\[z = r \sin\theta = r' \tan\theta\]
\[dz = \frac{r'}{\cos^2\theta}d\theta\]
da cui l'integrale diventa
\[E_{tot} = \int^{+\frac{\pi}{2}}_{-\frac{\pi}{2}} \frac{\lambda}{4\pi\varepsilon_0 r'}\cos\theta d\theta = \frac{\lambda}{2\pi\varepsilon_0 r'}\]
\end{esercizio}
La carica dell'elettrone e del protone sono esattamente uguali? Fin ora è stato visto sperimentalmente che coincidono con una precisione dell'ordine di \(10^{21}\). Se fossero diverse anche di una parte su un miliardo avremmo un effetto macrosocopico enorme: ponendo due sfere di ferro (z = 26; A = 55) a distanza di un metro, se le cariche sono esattamente uguali la forza fra esse è nulla, se invece avessimo 
\[\frac{|q_p|-|q_e|}{|q_e|} = \frac{|\Delta q|}{|q_e|}=10^{-9}\]
\[|\delta q| = q_p|\cdot10^{-9}\]
calcoliamo la differenza di carica totale moltiplicando quella di un singolo atomo per il numero totale di atomi
\[\Delta Q = \Delta q\cdot z \left(\frac{m}{A}\cdot N_A\right) \]
\[F=\frac{1}{4\pi\varepsilon_0}\frac{\Delta Q^2}{r^2}\simeq 1.9\cdot10^{7}\ N\]
La forza generata è enorme. 
\subsection{Potenziale elettrostatico}
Cominciamo con il chiederci se la forza elettrostatica è conservativa verificando che esiste una funzione dipendente solo dalla posizione con cui sia possibile calcolare il lavoro (il lavoro è indipendente dal percorso). 
\[L_{AB} = \int_{A}^{B}\mathbf{F}d\mathbf{l}= \frac{q_1q_2}{4\pi\varepsilon_0}\int_{A}^{B}\frac{1}{r^2}dl\cos\theta=\frac{q_1q_2}{4\pi\varepsilon_0}\int_{A}^{B}\frac{1}{r^2}dr=\frac{q_1q_2}{4\pi\varepsilon_0}\left(\frac{1}{r_A}-\frac{1}{r_B}\right)=U(B)-U(A)\]
Analogamente alla forza possiamo immediatamente verificare che il campo elettrostatico è conservativo poiché è uguale alla forza a meno di una costante
\[\int_{A}^{B}\mathbf{E}d\mathbf{l}=\frac{q}{4\pi\varepsilon_0}\left(\frac{1}{r_A}-\frac{1}{r_B}\right)=V(B)-V(A)\]
\begin{definizione}[Potenziale elettrostatico]
Definiamo la funzione di stato 
\[V(\mathbf{r}) = \frac{q}{4\pi\varepsilon_0}\frac{1}{r}\]
potenziale elettrostatico. La sua unità di misura è il Volt $V \equiv \frac{kg\cdot m^2}{s^2\cdot C}$.  Quest'ultima espressione è valida per una singola carica puntiforme. Se avessimo un numero finito di N cariche avremmo
\[V_i(\mathbf{r}) = \frac{1}{4\pi\varepsilon_0}\frac{q_i}{r_i}\]
\[V(\mathbf{r})= \frac{1}{4\pi\varepsilon_0}\sum_{i}\frac{q_i}{r_i}\]
Se avessimo una distribuzione continua di carica invece
\[V(\mathbf{r})= \frac{1}{4\pi\varepsilon_0}\ _Q\int\frac{d q}{r} =  \frac{1}{4\pi\varepsilon_0}\ _V\int\frac{\rho}{ r}dV\]
Si ha inoltre 
\[U(\mathbf{r})=qV(\mathbf{r})\]
\end{definizione}
Le superfici equipotenziali nel caso della carica singola sono i gusci sferici concentrici centrati nella carica, il campo elettrico è sempre perpendicolare ad essi.\\
Possiamo quindi scrivere le 4 proprietà equivalenti che presenta il potenziale elettrostatico in quanto conservativo
\begin{itemize}
	\item \(\oint \mathbf{E}d\mathbf{l} = 0 \)\ per ogni percorso chiuso $\Gamma$ 
	\item \(\mathbf{\n}\wedge\mathbf{E} = 0\) per ogni punto
	\item \(\exists\varphi\  \text{campo scalare} : \mathbf{E}=-\mathbf{\n}\varphi\)=$\left(E_x = -\frac{\d E}{\d x},\ E_y= - \frac{\d E}{\d y},\ E_z= -\frac{\d E}{\d z}\right)$
	\item \(\varphi(B)-\varphi(A) = -\int_{A}^{B}\mathbf{\n}\varphi d\mathbf{l}\ \) \ per ogni percorso $\Gamma$ 
\end{itemize}
dove $\varphi = V(\mathbf{r})$.
\begin{definizione}[Forza elettromotrice (f.e.m.)]
Definiamo forza elettromotrice
\[\mathcal{E} \equiv \oint \mathbf{E}d\mathbf{l}\]
che è nulla per campi elettrostatici ma che può essere diversa da zero se vi sono cariche in movimento.
\end{definizione}
\begin{esercizio}[Differenza di potenziale elettrostatico di un filo infinito]
Abbiamo già calcolato il campo elettrostatico di un filo
\[\mathbf{E}(\mathbf{r}) = \frac{\lambda}{2\pi\varepsilon_0}\frac{1}{x}\hat{x}\]
Ne calcoliamo ora la differenza di potenziale elettrostatico integrando sul percorso infinitesimo \(P_0\ , P\). 
\[V(P)-V(P_0) = \int_{P_0}^{P}\mathbf{E}d\mathbf{l} = \frac{\lambda}{2\pi\varepsilon_0} \int_{P_0}^{P}\frac{dx}{x}= \frac{\lambda}{2\pi\varepsilon_0} \left(\ln(r)-\ln(r_0)\right)\]
Notiamo che il risultato ottenuto è proprio una funzione di stato dipendente solo dalla posizione più una costante arbitraria.  
\end{esercizio}
\subsection{Teorema di Gauss}
Un ulteriore metodo di calcolo del campo elettrostatico sfrutta il teorema della divergenza che, applicato al calcolo del campo elettrostatico, prende il nome di teorema di Gauss. Innanzitutto bisogna svolgere un'introduzione su angoli e angoli solidi.\\
Un'angolo è la parte di piano compresa tra due semirette aventi il vertice in comune. Sia $\hat{u}$ il versore normale all'arco di circonferenza s compreso tra le due semirette, tale da avere direzione individuata dalla retta passante dal vertice e dal centro dell'arco e verso uscente. Detto $\theta$ l'angolo ed r la distanza dal vertice al punto d'intersezione tra arco e semiretta, l'angolo è il rapporto tra l'arco di circonferenza e il raggio 
\[\theta =\frac{s}{r}\]
Considerando archi infinitesimi si ottiene
\[d\theta = \frac{ds}{r}\] 
Possiamo estendere questo concetto considerando un'ulteriore arco di circonferenza ds' compreso tra le due rette che però abbia direzione del versore diversa da $\hat{u}$, chiameremo tale nuovo versore $\hat{u'}$. I due archi si intersecano formando un angolo $\alpha$. Essendo gli archi infinitesimi possiamo considerarli come retti, otteniamo allora che ds è il cateto e ds'l'ipotenusa di un triangolo rettangolo
\[ds' \cos\alpha = ds\]
Ricordando che l'angolo è il rapporto tra arco e raggio si ottene
\[d\theta = \frac{ds'\cos\alpha}{r}\]
1 radiante per definizione è l'angolo sotteso da un arco di circonferenza che ha lunghezza pari a quella del raggio.\\
Estendiamo questo concetto in 3 dimensioni: l'angolo solido è la parte di spazio compresa tra quattro semirette. analogamente a quanto fatto con gli archi di circonferenza, possiamo individuare una calotta sferica d$\Sigma_0$ con centro nel vertice a cui associamo un versore $\hat{u}$. L'angolo solido è il rapporto fra la calotta sferica compresa tra le quattro semirette che individuano l'angolo solido e il quadrato della distanza r dei vertici della calotta dall'intersezione delle rette
\[d\Omega = \frac{\Sigma_0}{r_0^2}\]
L'unità di misura dell'angolo solido è lo steradiante, definito come l'angolo solido sotteso da una calotta sferica di superficie \(r^2\). Analogamente a quanto fatto in 2 dimensioni, consideriamo un'altra calotta sferica d$\Sigma'$ con versore diverso dal precedente e ridefiniamo lo steradiante
\[d\Omega = \frac{d\Sigma'\cos\alpha}{r^2}\] 
Consideriamo ora un angolo solido in coordinate polari come in figura 
\begin{figure}[h!]
	\centering
	\includegraphics[width=0.3\linewidth]{../images/angolo_solido}
	\caption{Schema di un angolo solido in coordinate polari}
	\label{fig:angolosolido}
\end{figure}
\FloatBarrier
Risulta, ricordando la definizione di angolo in radianti
\begin{align*}
	&d\Sigma_0 = \hat{AB}\cdot\hat{AD}\\
	&\hat{AB}=rd\theta\\
	&\hat{AD}=r\sin\theta d\varphi\\
	&d\Sigma_0 = r^2\sin\theta d\theta d\varphi\\
	&d\Omega = \frac{r^2 \sin\theta d\varphi}{r^2} = \sin\theta d\theta d\varphi
\end{align*}
Abbiamo ottenuto un'espressione dell'angolo solido in steradianti indipendente dal raggio. Per ottenere un angolo solido basta integrare su entrambe gli angoli l'angolo solido infinitesimo
\[\Omega = \int\int d\Omega = \int \int \sin\theta d\theta d\varphi\]
Per integrare su tutto l'angolo solido si deve integrare \(\varphi\) da 0 a 180 e $\theta$ da 0 a 360. 
\[\Omega_{tot} = \int^{2\pi}_{0}\left(\int^{\pi}_{0} d\theta\right)d\varphi = 2\pi \int^\pi_{0} \sin\theta d\theta = 4\pi\]
Ne segue che l'angolo solido totale è 4$\pi$, se lo moltiplichiamo per il raggio al quadrato otteniamo la superficie della sfera, proprio come moltiplicando l'angolo totale per il raggio otteniamo la circonferenza.\\\\

Vogliamo ora calcolare il flusso del campo elettrostatico: consideriamo una calotta sferica \(\sigma\) ed una sua parte infinitesima d$\mathbf{s}$, con versore normale uscente dalla parte convessa; consideriamo inoltre un vettore campo elettrostatico $\mathbf{E}$ che forma un angolo \(\theta\) con il versore $\hat{u}$. Il flusso di $\mathbf{E}$ su \(d\mathbf{s}\) è 
\[d\Phi(\mathbf{E}) = \mathbf{E}d\mathbf{s} = Eds\cos\theta\]
Per ottenere il flusso su tutto sigma integriamo 
\[\Phi(\mathbf{E}) = \int_{\Sigma}d\Phi(\mathbf{E}) = \int_{\Sigma}\mathbf{E}d\mathbf{s} = \int_{\Sigma}E\cos\theta ds\]
Essendo la superficie considerata aperta, posso orientare il verso di d$\mathbf{s}$ a piacere cambiando semplicemente segno.\\
Consideriamo ora una superficie chiusa $\Sigma$ e una sua parte infinitesima $d\mathbf{s} = ds \hat{u}$, con direzione normale alla superficie e per convenzione con verso positivo quello uscente. Vogliamo calcolare il flusso elettrostatico totale della superficie chiusa; consideriamo prima l'effetto delle cariche esterne: prendiamo una carica esterna +Q e poniamola al centro di un sistema di riferimento, consideriamo un fascio di rette che stacca, intersecando la superficie chiusa, due superfici infinitesime d$\mathbf{s}_1$ e d$\mathbf{s}_2$, per le quali passa rispettivamente un campo $\mathbf{E}_1$ ed $\mathbf{E}_2$ (su ogni punto della superficie chiusa esiste campo elettrostatico generato dalla carica).  
\begin{figure}[h!]
	\centering
	\includegraphics[width=0.5\linewidth]{../images/thm_gauss_1}
	\caption{Schema del fascio di rette che partono da una carica esterna e staccano due superfici infinitesime dalla superficie chiusa}
	\label{fig:thmgauss1}
\end{figure}
\FloatBarrier
Si noti che l'angolo che forma il campo elettrico con la superficie infinitesima è rispettivamente \(\theta_1\) e \(\theta_2\); essendo la direzione del campo generato da una carica costante nello spazio, si ha
\[\theta_1 = \theta_2 + 180\]
\[\Rightarrow \cos\theta_1 = -\cos\theta_2\]
Possiamo calcolare il flusso delle due superfici infinitesime, ricordando la definizione di steradiante precedentemente ottenuta
\begin{align*}
	&d\Phi_2 = \mathbf{E}_2\cdot d\mathbf{s}_2 = E_2 ds_2 \cos\theta_2 = E_2 R_2^2 \frac{ ds_2 \cos\theta_2}{R_2^2} = E_2 R_2^2 d\Omega =\frac{1}{4\pi\varepsilon_0}\frac{Q}{R^2_2}R^2_2 d\Omega= \frac{Q}{4\pi\varepsilon_0}d\Omega \\
	&d\Phi_1 = \mathbf{E}_1\cdot d\mathbf{s}_1 = E_1 ds_1 \cos\theta_1 = -E_1 ds_1 \cos\theta_2 = -\frac{Q}{4\pi\varepsilon_0}d\Omega\\
	&\Rightarrow d\Phi_{12} = d\Phi_1 + d\Phi_2 = 0
\end{align*}
Non avendo usato specifiche proprietà della carica esterna o del campo o della superficie chiusa considerata possiamo dedurre che il flusso per una superficie chiusa di ogni carica esterna è nullo; per calcolare il flusso basta considerare quelle interne.\\
Consideriamo quindi una superficie \(\Sigma\) con una carica +Q al suo interno che sta al centro del nostro sistema di riferimento, consideriamo una superficie infinitesima d$\mathbf{s}$ a distanza r da Q, l'angolo fra d$\mathbf{s}$ ed $\mathbf{E}$ sarà $\theta$. Si noti che, essendo la carica positiva per ipotesi, il flusso è positivo per ogni punto della superficie mentre per una carica esterna poteva essere o positivo o negativo. Calcoliamo quindi il flusso
\begin{align*}
	&d\Phi(\mathbf{E}) = \frac{Q}{4\pi\varepsilon_0}d\Omega\\
	&\Phi(\mathbf{E}) = \oint_{\Sigma} d\Phi(\mathbf{E}) = \frac{Q}{4\pi\varepsilon_0}\oint_0^{4\pi}d\Omega = \frac{Q}{\varepsilon_0} = \int_{V(\Sigma)}\frac{\rho}{\varepsilon_0}d\tau
	\end{align*}
dove l'ultimo è un'integrale su tutto il volume racchiuso dalla superficie $\Sigma$ della densità volumetrica di carica e $\tau$ è una porzione di volume infinitesima. Questo risultato, in virtù del principio di sovrapposizione, vale anche per N cariche, dove al posto del singolo Q avremo la somma algebrica delle cariche 
\[Q_{tot} = \sum_iq_i\ \text{caso discreto}\]. 
\[Q_{tot} = \int_{V(\Sigma)}\rho d\tau\ \text{caso continuo}\]
Possiamo quindi enunciare il \textbf{teorema di Gauss}:\textit{Il flusso del campo elettrostatico attraverso una qualsiasi superficie chiusa $\Sigma$ è pari alla somma algebrica delle cariche in \(\Sigma\) diviso la costante dielettrica del vuoto $\varepsilon_0$.}
Riassumendo, la forma integrale del teorema di Gauss è:
\[\oint\mathbf{E}d\mathbf{s} = \frac{Q_{tot}}{\varepsilon_0}\]
Si noti che la legge di Gauss è valida in virtù del fatto che, per la legge di Coulomb, il campo elettrostatico diminuisce quadraticamente con la distanza ma anche la superficie che stacca l'angolo solido aumenta quadraticamente con la distanza quindi le due si compensano e ne risulta che il flusso non dipende dalla distanza.\\
Si noti inoltre che la legge di Gauss espressa in questa forma (integrale) esprime una proprietà media della superficie chiusa $\Sigma$. Possiamo esprimere la stessa legge in forma differenziale confrontando il teorema della divergenza (\ref{ap:thm_divergenza}) con quello di Gauss
\begin{align*}
	&\text{Thm. di Gauss: }\ \oint\mathbf{E}d\mathbf{s} = \int_{V(\Sigma)}\rho d\tau\\
	&\text{Thm. della divergenza: }\ \oint\mathbf{E}d\mathbf{s} = \int_{V(\Sigma)}\mathbf{\n}\cdot\mathbf{E}\ d\tau \\
	&\Rightarrow \mathbf{\n}\cdot\mathbf{E} = \frac{\rho}{\varepsilon_0}
\end{align*} 
Il teorema di Gauss in forma differenziale esprime una proprietà puntuale del flusso sulla superficie $\Sigma$. Nella forma differenziale si nota immediatamente che la divergenza del campo elettrico in un punto senza carica è nulla poiché la densità di carica in quel punto è nulla. Possiamo anche verificarlo calcolando la divergenza
\begin{align*}
	&\mathbf{E} = \frac{Q}{4\pi\varepsilon_0}\frac{\mathbf{r}}{r^3} = \frac{Q}{4\pi\varepsilon_0}\frac{x\hat{\mathbf{i}}+y\hat{\mathbf{j}}+z\hat{\mathbf{k}}}{(x^2+y^2+z^2)^{\frac{3}{2}}}\\
	&\mathbf{\n}\cdot\mathbf{E} = \frac{Q}{4\pi\varepsilon_0}\left(\frac{\d}{\d x}\frac{x}{(x^2+y^2+z^2)^{\frac{3}{2}}}+\frac{\d}{\d y}\frac{y}{(x^2+y^2+z^2)^{\frac{3}{2}}}+\frac{\d}{\d z}\frac{z}{(x^2+y^2+z^2)^{\frac{3}{2}}}\right)=\\
	& \frac{Q}{4\pi\varepsilon_0}\left(\frac{1}{(x^2+y^2+z^2)^{\frac{5}{2}}}\right)(-2x^2+y^2+z^2+x^2-2y^2+z^2+x^2+yì2-2z^2) = 0
\end{align*}
Infine, riproponiamo gli esercizi del calcolo del campo elettrostatico generato da un filo e da un piano su un punto a distanza r, già risolti in due modi differenti in precedenza
\begin{esercizio}[Campo elettrostatico generato da un filo infinito carico]
Sia $\lambda$ la densità di carica lineare del filo. Si noti che sussiste una simmetria cilindrica nel campo elettrostatico generato dal filo: consideriamo una superficie cilindrica di raggio di base r uguale alla distanza di P dal filo e di altezza h. Per il teorema di Gauss sappiamo che
\[\oint_{\Sigma}\mathbf{E}d\mathbf{s}=\frac{Q_{tot}}{\varepsilon_0} = h\lambda\]
Possiamo parimenti calcolare il flusso dividendo l'integrale chiuso in 3 integrali aperti: due per le basi ed uno per la superficie laterae del cilindro. Notiamo immediatamente che l'angolo formato fra la normale alla superficie delle basi e il campo elettrostatico è di 90 gradi, ne segue che il prodotto scalare è nullo e il flusso sulle due facce di base è nullo. Consideriamo ora il flusso sulla superficie laterale, campo elettrostatico e vettore superficie sono paralleli in ogni punto: l'angolo formato è sempre nullo e il coseno è sempre 1.
\[\oint_{\Sigma}\mathbf{E}d\mathbf{s}=\int_{\Sigma_l}\mathbf{E}d\mathbf{s}_l= \int_{\Sigma_l}Eds_l\]
Ricordiamo infine che sulla superficie laterale il campo è costante quindi lo portiamo fuori dall'integrale
\[\oint_{\Sigma}\mathbf{E}d\mathbf{s}=E\int_{\Sigma_l}ds_l=E 2\pi r h\]
Infine, eguagliando le due espressioni ottenute
\[E= \frac{\lambda}{\varepsilon_0 2\pi r }\]
\end{esercizio}
\begin{esercizio}[Campo elettrostatico generato da un piano infinito carico]\label{:campo_piano_infinito}
Sia $\sigma$ la densità di carica superficiale del piano. Notiamo che tutti i piani paralleli al piano carico presentano medesimo campo elettrostatico in ogni punto. Consideriamo il cilindro "coricato" con basi parallele al piano carico, altezza 2r (il piano taglia in due parti uguali il cilindro) e il punto P al centro del cerchio che sta alla base del cilindro. Analogamente a quanto fatto precedentemente, per il teorema di Gauss abbiamo:
\[\oint\mathbf{E}d\mathbf{s} = \frac{Q_{tot}}{\varepsilon_0}=\frac{\sigma}{\varepsilon_0}\int_Sds=\frac{A\sigma}{\varepsilon_0}\]
dove A è l'area di base del cilindro.\\
Ricalcoliamo ora il flusso dividendo l'integrale nelle tre superfici: sulla superficie laterale il vettore superficie è sempre perpendicolare a quello del campo elettrostatico quindi il flusso è nullo. Sulle due facce di base invece l'angolo è zero, inoltre essendo in entrambe i casi flusso uscente bisognerà sommare i due flussi
  \[\oint\mathbf{E}d\mathbf{s} = 2\int_{\Sigma_b}\mathbf{E}d\mathbf{s}_b= 2AE\]
Eguagliando le due espressioni ottenute si perviene a 
\[E=\frac{\sigma}{2\varepsilon_0}\]
Il campo ha direzione opposta dai due lati della superficie, la discontinuità che si osserva attraversando la superficie da un lato all'altro è dunque di
\[E_1-E_2 = \frac{\sigma}{2\varepsilon_0}-\left(-\frac{\sigma}{2\varepsilon_0}\right) = \frac{\sigma}{\varepsilon_0}\]
Si noti infine che nel caso del piano carico il flusso non dipende dalla distanza del punto poiché, essendo il piano infinito, aumentare la distanza porterebbe solo a "riscalare" il tutto ottenendo lo stesso flusso.
\end{esercizio}
\subsubsection{Discontinuità del campo elettrico attraverso una superficie carica}
Osserviamo ora delle proprietà interessanti del campo elettrostatico, quando si attraversa una superficie carica.\\
Consideriamo una superficie generica (non omogenea), costruiamo un cilindro che la interseca, visto che ci interessa sapere cosa accade nel passaggio da una faccia all'altra della superficie consideriamo un cilindro con altezza \(dh\to 0\), da ciò deriva che la componente di flusso per la superficie laterale del cilindro non è rilevante. Il flusso è quindi
\[\oint_{\Sigma}\mathbf{E}d\mathbf{s} = \mathbf{E}_{u}d\mathbf{s}_{u}+\mathbf{E}_{d}d\mathbf{s}_d=(E_u\cos\theta_u-E_d\cos\theta_d)dA = dA(E_{\perp u}-E_{\perp d})\] 
Per il teorema di Gauss, essendo la superficie del cilindro chiusa e $\sigma$ la densità di carica superficiale, si ha
\[\oint_{\Sigma}\mathbf{E}d\mathbf{s} = \frac{Q_{tot}}{\varepsilon_0} = \frac{\sigma dA}{\varepsilon_0}\] 
\[\Rightarrow |\Delta \mathbf{E}_{\perp}| = (E_{\perp u} - E_{\perp d}) = \frac{\sigma}{\varepsilon_0}\]
Ne segue che la discontinuità della componente perpendicolare alla superficie, nell'attraversarla, presenta un salto pari a \(\frac{\sigma}{\varepsilon_0}\) (non si hanno informazioni su come sia distribuita questa quantità fra le due facce). Si noti che è la stessa discontinuità ottenuta nell'esercizio (\ref{es:campo_piano_infinito}).\\
Ci chiediamo ora cosa accada alla componente tangente al campo nell'attraversare una superficie carica. Per farlo consideriamo la circuitazione del campo elettrostatico, calcolandola in un percorso rettangolare che interseca la superficie perpendicolarmente, a metà dei due lati perpendicolari. Essendo il campo conservativo abbiamo che la circuitazione deve essere nulla
\[\oint_\Gamma\mathbf{E}d\mathbf{r} = 0\]
D'altro canto possiamo calcolare la circuitazione lato per lato; visto che ci interessa il passaggio fra le due facce consideriamo un percorso rettangolare la cui altezza tenda a zero \(dh\to 0\) (considereremo solo i lati orizzontali del circuito).  
\[\oint_\Gamma \mathbf{E}d\mathbf{r} = \mathbf{E}_1 d\mathbf{l}_1 + \mathbf{E}_3d\mathbf{l}_3 = E_1dl_1\cos\theta_1 + E_3dl_3\cos\theta_3 = (E_{T1}-E_{T3})dl\]
Dove la t al pedice indica la componente tangenziale. Eguagliando le due espressioni ottenute si ha
\[\Rightarrow E_{T1}=E_{T3}\]
Ne segue quindi che la componente tangenziale resta continua attraversando la superficie. 
\newpage
\section{Elettrostatica della materia: i conduttori}
\subsection{Conduttori neutri}
I materiali si suddividono nelle due macrocategorie di conduttori e dielettrici (isolanti). Un conduttore metallico come il rame può essere modellizzato, in prima approssimazione, come un reticolo di ioni vincolati fra loro che possono solamente vibrare, immersi in un mare di elettroni tanto debolmente legati al nucleo da poter muoversi liberamente attorno al reticolo a seguito di eccitazioni energetiche minime come l'agitazione termica. Ogni atomo di rame contribuisce con circa un elettrone di conduzione, in un centimetro cubo vi sono
\[n_e = \frac{\rho_M }{A}N_A \simeq 8\cdot 10^22\ \frac{e^-}{cm^3}\simeq13000\ \frac{C}{cm^3}\]
dove $\rho_M$ è la densità ed A è la massa molare.
\begin{definizione}[Conduttore elettrico]
	Un conduttore elettrico è un materiale in grado di far scorrere corrente elettrica al suo interno. I materiali conduttori sono caratterizzati dalla presenza di elettroni liberi nella banda di valenza degli atomi del reticolo cristallino (conduttori di prima specie) o contengono specie ioniche che si fanno carico di trasportare la corrente (conduttori di seconda specie).
\end{definizione}
Di seguito elenchiamo, dimostrandole, 6 proprietà dei conduttori:
\begin{enumerate}
\item	Se inserisco un conduttore in un campo elettrico uniforme, ogni elettrone sperimenta una forza di verso opposto a quello del campo elettrostatico
\[\mathbf{F} = -e\mathbf{E} = m_e \mathbf{a}\]
Se aspettiamo un tempo sufficientemente lungo gli elettroni si sposteranno ed il sistema sarà nuovamente all'equilibrio (è un fatto sperimentale che a un certo punto gli elettroni smettono di muoversi). Se gli elettroni smettono di muoversi (intendendo movimento in media) vuol dire che una forza si oppone a quella generata dal campo elettrostatico, quale? Quando gli elettroni si spostano tutti nello stesso verso si crea una zona a prevalenza di cariche positive (il reticolo di ioni privo di elettroni) che attrae per forza di Coulomb gli elettroni a sè opponendosi alla forza generata dal campo elettrostatico. Passando da termini di forza a termini di campo, potremo dire che all'interno del conduttore si è creato un \textbf{campo elettrostatico indotto}, opposto a quello esterno (in verso e modulo). Abbiamo quindi che il campo elettrostatico totale all'interno di un conduttore è
\[\mathbf{E}_{tot} = \mathbf{E}_I+\mathbf{E}_E = 0\]
\textit{Il campo elettrico totale all'equilibrio, all'interno di un conduttore, è nullo, anche in presenza di un campo elettrostatico esterno}.
\item Possiamo applicare la legge di Gauss al caso appena visto 
\[\oint_{\Sigma}\mathbf{E}_{tot}d\mathbf{s}= \frac{Q_{tot}}{\varepsilon_0} = 0\]
\[\Rightarrow Q_{tot} = 0\]
Possiamo svolgere lo stesso ragionamento puntualmente usando la forma differenziale del teorema di Gauss
\[\mathbf{\n}\mathbf{E}_{tot}=\frac{\rho}{\varepsilon_0}=0 \]
\[\Rightarrow \rho = 0,\ \forall P \in C\]
dove C è l'insieme di punti all'interno del conduttore. \\
\textit{All'interno di un conduttore, in ogni punto, la densità di carica elettrica è nulla. In totale, all'interno del conduttore la carica totale è nulla. Ci sono quindi tante cariche positive quante negative. Il conduttore resta neutro prima e dopo l'applicazione di un campo elettrostatico $\mathbf{E}$}.
\item Essendo la densità di carica nulla all'interno di tutto il conduttore, è intuitivo pensare che le cariche si dispongano sulla superficie; a riprova di ciò ricordiamo che abbiamo già ottenuto che la componente normale di campo elettrico alla superficie di un piano carico presenta una discontinuità proprio sulla superficie di \(\frac{\sigma}{\varepsilon_0}\) da cui segue che la densità di carica non è nulla sulla superficie. \\
\textit{Le cariche di un conduttore immerso in un campo elettrostatico si dispongono sulla superficie}.
\item Osserviamo che le linee di campo non possono essere tangenti rispetto la superficie del conduttore perché sappiamo che queste devono essere continue prima e dopo la superficie e sappiamo anche che all'interno il campo è nullo. Per le stesse considerazioni non può esistere una componente tangente quindi saranno sicuramente ortogonali. Non può essere ortogonale verso l'interno perché altrimenti sposterebbe le cariche verso l'interno generando campo elettrostatico all'interno del conduttore ma sappiamo che in questa regione il campo è nullo. Ne segue che le linee di campo sono ortogonali dirette perso l'esterno (se la carica è positiva, se negativa si cambia il segno).\\
\textit{L'immersione di un conduttore in un campo elettrostatico provoca un'alterazione del campo preesistente sia all'interno del conduttore (si annulla) sia all'esterno (si piega per entrare ed uscire perpendicolarmente alla superficie del conduttore)}.
\item Studiamo il potenziale elettrostatico sulla superficie del conduttore: consideriamo i punti A, B e calcoliamone il potenziale
\[\Delta V_{AB} = V_B - V_A = \int_A^B \mathbf{E}d\mathbf{r} = 0\]
\'E nullo perché d$\mathbf{r}$ è tangente alla superficie per definizione mentre il campo è ortogonale ad essa, il prodotto scalare è dunque identicamente nullo. Inoltre, essendo il campo nullo all'interno del conduttore, presi due punti qualsiasi in questa regione, il potenziale è sempre nullo e quindi sempre uguale.\\
\textit{La superficie di un conduttore è equipotenziale come i punti al suo interno}.
\item Consideriamo ora un conduttore cavo, possiamo individuare una superficie esterna $\Sigma_E$, su cui sappiamo sono disposte le cariche, ed una superficie interna $\Sigma_I$, ci chiediamo come sono disposte le cariche su questa superficie. Consideriamo una generica superficie $\Sigma$ compresa tra quella interna e quella esterna, applichiamo la legge di Gauss
\[\oint_\Sigma \mathbf{E}d\mathbf{s} = \frac{Q_{tot}}{\varepsilon_0} = 0 \Rightarrow Q_{tot} = 0\]
La carica totale all'interno di questa superficie è nulla, essendo $\Sigma$ arbitraria possiamo prenderla infinitamente vicina a $\Sigma_I$ ottenendo che l'unica disposizione possibile sarebbe con le cariche positive e negative in numero eguale e disposte sula superficie interna come su quella esterna (metà da una parte e metà dall'altra a bilanciarsi). Tuttavia in questo caso si violerebbe la conservatività del campo elettrostatico: consideriamo un percorso chiuso passante in parte dall'interno cavo (AB) ed in parte tra $\Sigma_I$ e $\Sigma_E$, avremmo
\[\oint_{\Gamma}\mathbf{E}d\mathbf{r}= \int_{A}^{B}\mathbf{E}d\mathbf{r}+\int_{B}^{A}\mathbf{E}d\mathbf{r}\]
Il secondo addendo è nullo perché abbiamo visto che all'interno del conduttore il campo è identicamente nullo mentre il primo addendo è diverso da zero perché d$\mathbf{r}$ è parallelo a $\mathbf{E}$. Ne segue che la circuitazione non è nulla e la conservatività del campo è violata, il che è assurdo. Ne segue per esclusione che non vi sono cariche sulla superficie interna del conduttore cavo. Ciò spiega perché all'interno di un ascensore (metallico) il telefono non prende (non c'è campo). 
\textit{Se ho un conduttore cavo la redistribuzione delle cariche riguarda solo la superficie esterna}. 
\end{enumerate}
\begin{esercizio}[Campo elettrostatico di un cilindro carico di altezza infinita]
Consideriamo un cilindro carico con densità volumetrica di carica uniforme $\rho$, raggio di base R e altezza h infinita. Cominciamo calcolando il campo all'esterno del cilindro circondandolo con una superficie cilindrica di raggio \(r>R\). Per il teorema di Gauss abbiamo
\[\oint_{\Sigma}\mathbf{E}d\mathbf{s}=\frac{Q_{tot}}{\varepsilon_0}\]
Calcolando invece il flusso dividendo la superficie cilindrica in tre superfici (area laterale e due basi) notiamo che il flusso per le basi è nullo, inoltre il campo sulla superficie laterale è costante data la simmetria cilindrica dell'oggetto carico considerato.
\[\oint_{\Sigma}\mathbf{E}d\mathbf{s}=\int E ds_l = E\int ds = E2\pi r h\]
Considerato che
\[Q_{tot} = \rho V = \rho \pi R^2 h\]
si ottiene
\[\mathbf{E}_{est} = E \hat{\mathbf{i}} = \frac{\rho \pi R^2 h }{2\pi r h \varepsilon_0} \hat{\mathbf{i}}= \frac{\rho R^2 }{2 r \varepsilon_0}\hat{\mathbf{i}}\]
\end{esercizio}
Ora calcoliamo il campo all'interno del cilindro, analogamente a prima consideriamo una superficie cilindrica questa volta interna al cilindro carico (\(r<R\)). 
\[\oint_{\Sigma}\mathbf{E}d\mathbf{s}=\frac{Q_{tot}}{\varepsilon_0} = \frac{\rho 2 \pi r^2}{\varepsilon_0}\]
\[\oint_{\Sigma}\mathbf{E}d\mathbf{s}=E\int ds_l = E2\pi rh\]
\[\mathbf{E}_{int} = \frac{\rho r}{2\varepsilon_0}\hat{\mathbf{i}}\]
Possiamo ottenere il valore del campo sulla superficie usando una delle due relazioni ottenute sostituendo R = r
\[\mathbf{E}_{sup} = \frac{\rho R}{2\varepsilon_0}\hat{\mathbf{i}}\]
Infine la differenza di potenziale fra due punti a distanza \(r_1\) ed \(r_2\) è
\[V(r_2)-V(r_1) = \int_{r_1}^{r_2}\mathbf{E}d\mathbf{r}=\ln\left(\frac{r_2}{r_1}\right)\frac{\rho R^2}{2\varepsilon_0}\]
\begin{esercizio}[Campo elettrostatico di un guscio sferico carico]
Consideriamo un guscio sferico di raggio R e densità superficiale di carica $\sigma$. Per ottenere il campo all'esterno del guscio uguagliamo la legge di Gauss al calcolo del flusso integrando sulla superficie, la direzione del campo sappiamo essere sempre perpendicolare alla superficie della sfera quindi avrà versore $\hat{\mathbf{r}}$.
\[\oint_{\Sigma}\mathbf{E}_{est}d\mathbf{s} = \frac{Q_{tot}}{\varepsilon_0} = \frac{4\pi R^2 \sigma}{\varepsilon_0}\]
Per il calcolo del flusso consideriamo una superficie sferica esterna di raggio \(r>R\)
\[\oint_{\Sigma}\mathbf{E}_{est}d\mathbf{s} = E\oint ds = E_{est} 4\pi r^2\]
\[\Rightarrow \mathbf{E}_{est} = \frac{R^2 \sigma}{r^2 \varepsilon_0} \hat{\mathbf{r}}\]
Per ottenere il campo all'interno del guscio notiamo che applicando il teorema di Gauss considerando una superficie sferica \(r<R\) abbiamo che \(Q_{tot}\) interna è nulla quindi anche il campo elettrostatico lo è. Abbiamo quindi una discontinuità di $\frac{\sigma}{\varepsilon_0}$ sulla superficie proprio come dimostrato in precedenza. Infatti per calcolare il campo sulla superficie è sufficiente porre \(r=R\) nella formula per il campo esterno.\\
Per il potenziale basta integrare il campo in d$\mathbf{r}$ ottenendo
\[V(r)_{est} = \int_{r}^{+\infty} \frac{R^2 \sigma}{r^2 \varepsilon_0} dr = \frac{R^2 \sigma}{ r\varepsilon_0}\]
\[V(r)_{int} = \int^{R}_r 0 dr = costante \]
Dove la costante si ricava osservando che il potenziale è funzione continua di r quindi deve assumere lo stesso valore da r ad R; possiamo calcolarne il valore in R usando la formula per il campo esterno
\[V(r)_{int} = \frac{R \sigma}{\varepsilon_0}\]
\end{esercizio}
\begin{esercizio}[Campo elettrostatico di una sfera piena carica]
Consideriamo ora una sfera piena carica di raggio R e densità volumetrica di carica $\rho$ (la carica è distribuita uniformemente sul volume). Per il campo esterno consideriamo una superficie sferica di raggio \(r>R\), eguagliando legge di Gauss e calcolo del flusso otteniamo analogamente a prima 
\[\mathbf{E}_{est} = \frac{\rho R^3}{3 \varepsilon_0 r^2}\hat{\mathbf{r}}\]
All'interno invece consideriamo una sfera di raggio \(r<R\), ora nella legge di Gauss la carica totale non sarà quella di tutta la sfera carica ma solamente quella all'interno della superficie sferica considerata. 
\[\oint_{\Sigma}\mathbf{E}d\mathbf{s}=\frac{Q_{tot}}{\varepsilon_0} = \frac{\rho 4 \pi r^3}{3\varepsilon_0}\]
\[\oint_{\Sigma}\mathbf{E}d\mathbf{s}=E\int ds = E 4\pi r^2\]
\[\Rightarrow \mathbf{E}_{int} = \frac{\rho r}{3 \varepsilon_0}\hat{\mathbf{r}}\]
Dunque la carica all'interno cresce linearmente con la distanza dal centro della sfera fino alla superficie (dove arriva al valore di \(\frac{\rho R}{3 \varepsilon_0}\)) per poi decrescere quadraticamente con la distanza dal centro della sfera.\\
Per il potenziale basta integrare il campo in d$\mathbf{r}$ ottenendo
\[V(r)_{est} = \int_{r}^{+\infty} \frac{\rho R^3}{3 \varepsilon_0 r^2}dr = \frac{\rho R^3}{3 \varepsilon_0 r}\]
Dunque sulla superficie della sfera abbiamo
\[V(R) = \frac{\rho R^2}{3 \varepsilon_0 }\]
\[V(r)_{int} = \int_{r}^{+\infty} \frac{\rho r}{3 \varepsilon_0} dr = \int_{r}^{R}\frac{\rho r}{3 \varepsilon_0} dr+\int_{R}^{+\infty}\frac{\rho r}{3 \varepsilon_0} dr =\frac{\rho}{6\varepsilon_0}\left(R^2-r^2\right) +\frac{\rho R^2}{3 \varepsilon_0 }\]
\end{esercizio}
\begin{esercizio}[Campo elettrostatico di un anello carico]
Consideriamo un anello carico di densità lineare di carica $\lambda$, carica totale Q e raggio R ed un punto P a distanza R dal centro dell'anello come in figura.
\begin{figure}[h!]
	\centering
	\includegraphics[width=0.6\linewidth]{../images/anello}
	\caption{Schema per il calcolo del campo elettrico di un anello carico}
	\label{fig:anello}
\end{figure}
\FloatBarrier
Per il principio di sovrapposizione risulta che il campo risultante giace solo sull'asse z. Calcoliamo il campo elettrico generato da una porzione dl dell'anello, a distanza z dal punto P
\[dq = dl\lambda = R d\alpha \lambda\] 
\[d\mathbf{E} = \frac{ R d\alpha\lambda \cos\theta}{4\pi\varepsilon_0z^2}\hat{\mathbf{k}} = \]
Notiamo che $\cos\theta = \frac{r}{z}$
\[d\mathbf{E} = \frac{ R d\alpha\lambda r }{4\pi\varepsilon_0 z^3}\hat{\mathbf{k}} = \frac{ R d\alpha\lambda r }{4\pi\varepsilon_0 (R^2+r^2)^{\frac{3}{2}}}\hat{\mathbf{k}}\]
Per ottenere il campo totale integriamo 
\[\mathbf{E} = \frac{ R \lambda r }{4\pi\varepsilon_0 (R^2+r^2)^{\frac{3}{2}}}\hat{\mathbf{k}}\int_{0}^{2\pi}d\alpha= \frac{ R \lambda r }{2\varepsilon_0 (R^2+r^2)^{\frac{3}{2}}}\hat{\mathbf{k}} \]
Per ottenere il potenziale integriamo il campo in dr
\[V(r) = \frac{R \lambda}{2\varepsilon_0}\int_{r}^{+\infty}\frac{  r }{ (R^2+r^2)^{\frac{3}{2}}} dr= \frac{R \lambda}{2\varepsilon_0}\frac{1}{2}\int_{r}^{+\infty}\frac{ dx }{ x^{\frac{3}{2}}} = \frac{ R \lambda }{2\varepsilon_0 (R^2+r^2)^{\frac{1}{2}}}\]
dove nell'ultimo passaggio è stato sostituito \(R^2+r^2 = x\).
\end{esercizio}
\begin{esercizio}[Campo elettrostatico di un disco carico]
Ora, al posto di un anello consideriamo un disco di densità superficiale di carica $\sigma$ e raggio R. Possiamo sfruttare il campo dell'anello appena ottenuto ed integrarlo da 0 ad R. per farlo esprimiamo la carica infinitesima come la carica contenuta in un anello si spessore infinitesimo
\[dq = \sigma 2\pi l dl\]
dove l è stato sostituito al precedente R perché quest'ultimo era costante mentre l dovrà essere la nostra variabile di integrazione. 
\[d\mathbf{E} = \frac{  \sigma l dl r }{2\varepsilon_0 (l^2+r^2)^{\frac{3}{2}}}\hat{\mathbf{k}}\]
\[\mathbf{E} = \frac{\sigma r}{2\varepsilon_0}\int_{0}^{R}\frac{ldl}{(l^2+r^2)^{\frac{3}{2}}} = \frac{\sigma r}{2\varepsilon_0}\left[-\frac{1}{(l^2+r^2)^{\frac{1}{2}}}\right]^R_0 = \frac{\sigma }{2\varepsilon_0}\left[\frac{r}{|r|}-\frac{r}{\sqrt{R^2+r^2}}\right]\]
Si noti che il valore assoluto porta una discontinuità tale che per \(z\to0^+\) si ha \(E = \frac{\sigma}{2\varepsilon_0}\) e per \(z\to0^-\) si ha \(E = -\frac{\sigma}{2\varepsilon_0}\); la discontinuità nell'attraversare la superficie del disco è quindi \(\frac{\sigma}{\varepsilon_0}\) come dimostrato nel caso generale in precedenza.
\end{esercizio}
\begin{esercizio}[Campo elettrostatico di uno spessore di carica di superficie infinita]
Consideriamo una lastra carica di superficie infinita spessa D di densità di carica volumetrica uniforme $\rho$. Consideriamo un cilindro di asse perpendicolare al piano della lastra di basi \(B_1\) e \(B_2\), eguagliamo la legge di Gauss al calcolo del flusso (che sulla superficie laterale del cilindro è nullo) per ottenere il campo all'esterno della lastra. 
\[\Phi(\mathbf{E}) = \frac{Q_{tot}}{\varepsilon_0} = \frac{\rho A D}{\varepsilon_0}\]
\[\Phi(\mathbf{E}) = 2EA\]
\begin{align*}
	\mathbf{E} =
	\begin{cases}
	\frac{\rho D}{2\varepsilon_0}\hat{\mathbf{k}}\ se\ x>D/2\\
	-\frac{\rho D}{2\varepsilon_0}\hat{\mathbf{k}}\ se\ x<D/2\\
	\end{cases}
\end{align*}
 Per il calcolo del campo all'interno (\(-D/2<x<D/2\)), il flusso calcolato con l'integrale di superficie resta lo stesso mentre quello calcolato con Gauss diventa
  \[\Phi(\mathbf{E}) = \frac{Q_{tot}}{\varepsilon_0} = \frac{\rho A 2x}{\varepsilon_0} \]
  \[\Rightarrow \mathbf{E} = \frac{\rho x}{\varepsilon_0}\hat{\mathbf{k}}\]
\end{esercizio}
Analizziamo infine l'ultima casistica riguardante i conduttori neutri: un conduttore cavo con una carica +Q all'interno della cavità. La carica genera un campo le cui linee entrano perpendicolarmente sulla superficie intera del conduttore. Consideriamo una superficie $\Sigma'$ intermedia tra le superfici interna ed esterna del conduttore, abbiamo già dimostrato che il flusso all'interno di $\Sigma'$ è nullo poiché una superficie all'interno del conduttore, per annullare il campo generato dalla carica +Q, dovrà disporsi sulla superficie interna una carica -Q, per controbilanciare il campo. Essendo il conduttore neutro, all'esterno di $\Sigma'$ dovrà esserci una carica +Q, che genera un campo elettrico all'esterno del conduttore. Questo, per la legge di Gauss applicata ad una superficie esterna al conduttore $\Sigma''$, ha intensità $\frac{Q}{\varepsilon_0}$. Questo fenomeno, per cui tutte le linee di campo interne al conduttore escono da esso, viene detto \textbf{induzione completa}.  
\subsection{Conduttori carichi}
Generalizziamo ora quanto ottenuto per conduttori neutri a conduttori carichi. Euristicamente, le cariche essendo dello stesso segno si respingono e si dispongono a distanza massima fra di esse : sulla superficie. Questo fenomeno è del tutto simile a quello dell'induzione quindi possiamo enunciare le stesse proprietà da essa derivanti
\begin{enumerate}
	\item Dentro il conduttore il campo è nullo.
	\item La densità di carica è nulla quindi le cariche si dispongono sulla superficie.
	\item Il campo è normale alla superficie e la discontinuità del modulo della componente perpendicolare del campo è $\frac{\sigma}{\varepsilon_0}$.
	\item Ogni punto interno al conduttore è equipotenziale. Ogni punto sulla superficie del conduttore è equipotenziale 
	\item Se il conduttore è cavo le cariche stanno solo sulla superficie esterna.
\end{enumerate}
\subsection{Disposizione delle cariche sulla superficie}
Ma come si dispongono le cariche sulla superficie? Sperimentalmente si esclude l'ipotesi più semplice: le cariche non si dispongono in modo uniforme; si osserva infatti che la densità è maggiore dove il raggio di curvatura è maggiore. Per mostrarlo studiamo il caso di due sfere di raggio \(R_1\), \(R_2\) con \(R_1>R_2\) collegate da un filo di dimensioni trascurabili nel quale è presente una carica +Q. La carica si redistribuisce sulle due sfere, vogliamo studiare come. Osserviamo che possiamo considerare il sistema come un unico conduttore, dalle proprietà di questi sappiamo che tutti i punti sono equipotenziali quindi calcoliamo i potenziali \(V_1\), \(V_2\) sulla superficie delle due sfere ed eguagliamoli
\begin{align*}
	&V_1 = V(+\infty) - V(R_1) = \int_{R_1}^{+\infty} \mathbf{E}d\mathbf{r} =  \frac{Q_1}{4\pi_0}\int_{R_1}^{\infty}\frac{1}{r^2}dr = \frac{Q_1}{4\pi\varepsilon_0R_1} \\
	&V_2 = \frac{Q_2}{4\pi\varepsilon_0R_1}\\
	&\Rightarrow \frac{Q_1}{R_1}= \frac{Q_2}{R_2}\\
	&\Rightarrow \frac{\sigma_1}{R_2} = \frac{\sigma_2}{R_1}
\end{align*}
Da cui segue che la densità superficiale di carica aumenta con il raggio di curvatura.\\
Considerando un conduttore appuntito, sulla punta la curvatura è estremamente alta e così la sua densità di carica; in questo punto avvengono dei fenomeni di scarica sotto forma di fulmini ("potere dispersivo delle punte"). Se la curvatura è negativa, in quella regione non si dispongono cariche. Se la densità di carica è alta il campo elettrico subito oltre la punta sarà molto grande, può accadere che un elettrone libero o uno ione siano accelerati a sufficienza da questo campo, prima di effettuare un urto, in modo da acquistare una grande quantità di energia tale da far saltare un elettrone ad un altro atomo, questo nuovo elettrone libero sarà soggetto allo stesso meccanismo: in questo modo si produce una scintilla (o un piccolo fulmine) che scaricano il condensatore. Questo meccanismo si chiama perforazione del dielettrico, è lo stesso meccanismo che avviene nei fulmini e per evitarlo bisogna costruire condensatori il puù lisci possibile per evitare la presenza di punte.\\
Alla luce di quanto ottenuto possiamo rivedere il risultato precedentemente ottenuto per cui nella superficie interna non si dispongono cariche: possiamo considerare un conduttore pieno con carica disposta in superficie e piegarlo fino ad ottenere un conduttore cavo, più lo pieghiamo più la curvatura che formerà la cavità diventa negativa, allontanando da sè le cariche.\\
Quando colleghiamo un conduttore carico al terreno questo si scarica, ciò avviene perché entra a contatto con una sfera di raggio di curvatura enorme (approssimabile ad infinito). 
\subsection{Metodo delle cariche immagine}
Un modo per determinare il campo elettrostatico in presenza di conduttori è quello di sostituire quest'ultimo con una ipotetica carica che genererebbe esattamente lo stesso effetto. Nonostante ciò sia intuitivamente valido bisogna dimostrare matematicamente che l'effetto di una carica può effettivamente essere equivalente ad un conduttore. Osserviamo che ricercare il potenziale generato da una carica equivale a risolvere l'equazione differenziale di Poisson
\[{\n}^2 V = -\mathbf{\n}\mathbf{E}  = -\frac{\rho}{\varepsilon_0}\]
La risoluzione di una equazione differenziale del secondo ordine necessita di due condizioni al contorno: una sul campo elettrico ed una sul potenziale.
\begin{teorema}[Teorema di unicità della soluzione dell'equazione di Poisson]
	Si consideri una regione di spazio finita  \(\tau\)  delimitata dalla superficie S. Se in questa regione la funzione densità $\rho$ è integrabile e la funzione potenziale V sulla superficie S assume un valore ben preciso \(V_s\), allora la soluzione dell'equazione di Poisson è unica.
\end{teorema}
Per assurdo, se esistessero due soluzioni diverse \(V_1,\  V_2\) in S, che soddisfano le stesse condizioni al contorno su una stessa superficie S, potremmo definire la funzione differenza come segue
\[f\equiv V_1-V_2\]
Notiamo che sulla superficie S il valore di f è costante e sempre nullo.\\
Mostriamo preliminarmente che il laplaciano di tale funzione è nullo:
\[\n^2 V_1 - \n^2 V_2 = \n^2(V_1 - V_2) = \n^2 f = -\frac{\rho}{\varepsilon_0}+\frac{\rho}{\varepsilon_0}=0 \]
Consideriamo ora \(\mathbf{\n}(f\mathbf{\n}f)\) e calcoliamone l'integrale su un volume $\tau$ ed applichiamo il teorema della divergenza
\[\int_\tau \mathbf{\n}(f\mathbf{\n}f)d\tau= \int_Sf\mathbf{\n}fd\mathbf{s}\]
L'integrale a destra è nullo perché f è nulla su tutto S, l'integrale di sinistra si sviluppa con la derivata di prodotto come
\[\int_\tau \mathbf{\n}(f\mathbf{\n}f)d\tau = \int_\tau (\mathbf{\n}f )^2 d\tau + \int_\tau (f{\n}^2f)d\tau = 0\]
Il laplaciano di f, come notato, è nullo quindi il secondo addendo si annulla. Infine otteniamo
\[\int_\tau (\mathbf{\n}f )^2 d\tau=0\]
\[\Rightarrow \mathbf{\n}f = 0\]
Da cui si deduce che f è sempre costante ma essendo 0 su S vorrà dire che sarà sempre nulla da cui \[V_1 = V_2\]. 
\\\\
Per mostrare che l'effetto di una carica è il medesimo di quello di un conduttore basta mostrare che entrambi soddisfano le stesse condizioni al contorno. Ad esempio, per calcolare il campo generato da una carica su un conduttore piano infinito le condizioni al contorno sono che le linee di campo entrano ortogonalmente alla superficie del conduttore e che il potenziale sulla superficie del conduttore è nullo (lo poniamo nullo in quella regione essendo definito a meno di una costante). Si dimostra facilmente che le stesse condizioni sono rispettate se al posto del conduttore mettiamo una carica di segno opposto alla prima a medesima distanza dalla superficie conduttrice ma dal lato opposto (e chiaramente sullo stesso asse). Per il teorema appena dimostrato il potenziale sarà lo stesso nei due casi. 
\newpage
\section{Condensatori}
Cominciamo la trattazione di un nuovo oggetto fondamentale in elettrostatica, i condensatori, introducendo il concetto di \textbf{tubo di flusso}. Considerando le linee di forza di un campo, il tubo di flusso è formato dalle linee che si appoggiano su una superficie chiusa (come quelle verdi in figura).
\begin{figure}[h!]
	\centering
	\includegraphics[width=0.6\linewidth]{../images/tubo_di_flusso}
	\caption{Rappresentazione di un tubo di flusso}
	\label{fig:tubodiflusso}
\end{figure}
\FloatBarrier
La superficie laterale ha la caratteristica di avere ogni punto d$\mathbf{s}$ ortogonale al flusso $\mathbf{E}$. Osserviamo che le superfici che fanno da "tappi" del tubo di flusso (in figura C1 e C2), possono essere scelte arbitrariamente, le prendiamo in modo che i vettori $d\mathbf{s}_1$, \(d\mathbf{s}_2\) siano paralleli al vettore campo $\mathbf{E}$. Di modo che
\[\mathbf{E}\cdot d\mathbf{s}_1 = Eds_1\cos\theta= \pm |Eds_1|\]
dove il più o meno indica il fatto che a seconda del verso avremo un angolo di 0 o di 180 gradi. Ne risulta che il flusso per un tubo di flusso è
\[d\Phi = \pm |Eds_1|\pm |Eds_2| =\frac{dQ}{\varepsilon_0}\]
Consideriamo ora una superficie chiusa carica infinitesima e calcoliamo il flusso di un tubo di flusso: le linee di campo partono dalla superficie divergendo verso l'esterno, scegliamo la superficie di "tappo" del tubo come appena detto. All'interno della superficie non è presente campo elettrico quindi a rigore non potremmo chiudere il tubo di flusso all'interno della superficie chiusa, possiamo però prolungare analiticamente le linee di campo all'interno e chiuderla con una superficie interna al conduttore. Sia \(S_1\) la superficie interna ed \(S_2\) quella esterna. Applichiamo ora la legge di Gauss
\[E_1ds_1+E_2ds_2 = E_2ds_2 = \frac{dQ}{\varepsilon_0}\]
\[\Rightarrow E_2 = \frac{dQ}{\varepsilon_0 ds_2}\]
Questa è l'espressione del campo elettrico passante per un tubo di flusso.\\
Si noti che \(E_2\) non è costante perché dipende dalla superficie individuata dal tubo di flusso: essendo le linee di campo divergenti dalla superficie, più ci si allontana più \(ds_2\) aumenta, facendo diminuire il campo.\\
Possiamo ora introdurre i condensatori
\begin{definizione}[Condensatore]
	Un condensatore è un componente elettrico formato da due conduttori chiamati armature, separati da un materiale isolante, ed ha la capacità di immagazzinare l'energia elettrostatica associata a un campo elettrostatico.
\end{definizione}
Applichiamo quanto ricavato ad un condensatore. Un condensatore è un apparato formato da due "armature", ovvero conduttori carichi uno positivamente ed uno negativamente, poste una di fronte all'altra. Le armature che consideriamo per ora sono poste nel vuoto. Sia A l'armatura di carica +Q e B quella di carica -Q. Vogliamo calcolare la differenza di potenziale fra le due armature, applicare la definizione di potenziale per questo calcolo è ostico perché non sappiamo come varia il campo nello spazio compreso fra le due armature. Consideriamo quindi un tubo di flusso infinitesimo che va da A a B, il campo elettrico per questo tubo è
\[E_i  = \frac{dQ_i}{\varepsilon_0 ds_i}\]
\[\Delta V_{AB} = \int_{A}^{B}\mathbf{E}d\mathbf{r} =dQ_i \int_{A}^{B}\frac{dr_i}{\varepsilon_0 ds_i}\]
In generale questo integrale è complesso da calcolare. Riordinando i termini otteniamo
\[dQ_i = \frac{\Delta V_{AB}}{\int_{A}^{B}\frac{dr_i}{\varepsilon_0 ds_i}}\]
\[Q = \sum_i\frac{\Delta V_{AB}}{\int_{A}^{B}\frac{dr_i}{\varepsilon_0 ds_i}}\]
Definiamo la \textbf{capacità di un condensatore} come
\[C\equiv \frac{1}{{ \int_{A}^{B}\frac{dr_i}{\varepsilon_0 ds_i}}}\]
osserviamo che questa quantità è una costante che dipende dalla geometria del conduttore e dal dielettrico in cui questo è immerso ($\varepsilon_0$ è la costante dielettrica del vuoto ma potremmo scegliere qualsiasi mezzo). L'unità di misura della capacità è il Farad (\(F\)) definito come un Coulomb su un Volt. 
\[1F \equiv \frac{1C}{1V}\]
questa capacità è enorme tanto che, in un circuito comune la capacità figura spesso il microFarad (\(\mu F\)).\\
La precedente relazione quindi si riscrive come
\[C = \frac{Q}{\Delta V}\]
Si faccia attenzione al fatto che se si diminuisce la carica nel condensatore la capacità non diminuisce perché, come detto, è una costante dipendente solo da geometria e materiale; ciò che avviene è che diminuendo la carica diminuisce anche la differenza di potenziale, mantenendo il valore di C invariato.\\
Per il calcolo della capacità in generale si seguono 3 step
\begin{enumerate}
\item Calcolare $\mathbf{E}$ tra le facce del condensatore
\item Calcolare la differenza di potenziale fra le armature \(\Delta V\)
\item Calcolare la capacità con la relazione \(C=\frac{Q}{\Delta V}\) facendo attenzione che l'espressione finale della capacità non dipenda da Q. 
\end{enumerate} 
\subsubsection*{Condensatore a facce piane parallele infinite}
Consideriamo due superfici piane parallele infinite di carica uguale ma in segno opposto. Possiamo intuitivamente ottenere dove si dispongono le linee di campo, per farlo consideriamo le linee di campo di ogni armatura singolarmente per poi unirle con il principio di sovrapposizione: la faccia di carica positiva produce un campo uscente e perpendicolare ad essa sia nel lato sinistro del condensatore, sia al centro sia nel lato destro; l'armatura carica negativamente produce linee di campo in modo analogo ma di verso opposto. Sovrapponendo le linee si ottiene che quelle nel lato sinistro e destro si eliminano (i campi generati hanno la stessa intensità essendo medesima la carica) mentre nella regione interna al condensatore le linee di campo hanno stessa direzione e verso quindi si sommano dando linee di modulo doppio.
\begin{figure}[h!]
	\centering
	\includegraphics[width=0.4\linewidth]{../images/condensatore-piano}
	\caption{Schema di un condensatore a facce piane parallele infinite, le linee di campo sono presenti solo fra le due armature e vanno dall'armatura carica positivamente a quella carica negativamente.}
	\label{fig:condensatore-piano}
\end{figure}
\FloatBarrier
Possiamo ricavare da ciò in modo formale che le cariche si dispongono sulla superficie interna (osservato precedentemente con ragionamenti non formali) applicando la legge di Gauss scegliendo un cilindro che attraversa l'armatura partendo dall'interno ed attraversando una faccia esterna: il flusso sulla superficie laterale è sicuramente nullo perché perpendicolare al campo, sulla faccia interna del cilindro è nulla perché questa è perpendicolare al campo, sulla faccia esterna è nulla perché come osservato precedentemente il campo è presente solo all'interno del condensatore (fra le due armature). 
\[\oint \mathbf{E}d\mathbf{s}= \frac{Q_{tot}}{\varepsilon_0}=0\]
Se le cariche non stanno all'interno nè sulla superficie esterna saranno allora sulla superficie interna.\\
Per ottenere la Capacità di questo condensatore seguiamo i 3 step: calcoliamo il campo elettrico fra le armature.
Per farlo calcoliamo il flusso per la faccia interna del condensatore considerando un cilindro che parte dall'interno dell'armatura e termina all'esterno attraversando la faccia interna. Osservando che l'integrale sulla superficie del cilindro è l'Area di base A e che il flusso nella faccia del cilindro interna all'armatura è nullo, si ha, eguagliando con il teorema di Gauss
\[\oint\mathbf{E}d\mathbf{s} = EA = \frac{\sigma A}{\varepsilon_0}\]
\[\Rightarrow E = \frac{\sigma}{\varepsilon_0}\]
Si noti che apparentemente questo risultato sembra contraddire quanto ottenuto in precedenza: abbiamo precedentemente dimostrato infatti ( esercizio (\ref{es:campo_piano_infinito})) che il campo generato da una superficie carica è $\frac{\sigma}{2\varepsilon_0}$. In realtà il campo calcolato è la somma del campo prodotto dalle due armature poiché inizialmente, applicando il principio di sovrapposizione al calcolo delle linee di campo, abbiamo sommato i due contributi che, essendo uguali, sono di $\frac{\sigma}{2\varepsilon_0}$ ciascuno. Si noti inoltre che sia considerati separatamente sia considerati a seguito della sovrapposizione dei campi la discontinuità è sempre $\frac{\sigma}{\varepsilon_0}$, come precedentemente dimostrato (quando si considerano insieme il campo all'interno e dietro le armature è nullo).
Una volta ottenuto il campo elettrico calcoliamo la differenza di potenziale fra le due facce 
\[\Delta V_{AB} = \int_{A}^{B}\mathbf{E}d\mathbf{r} = Ed=\frac{\sigma d}{\varepsilon_0}=\frac{Q d}{S\varepsilon_0}\]
calcoliamo infine C
\[C = \frac{Q}{\Delta V} = \frac{S\varepsilon_0}{d}\]
come previsto C non dipende da Q ma solamente dalla geometria e dal materiale.\\
Un'ultima considerazione: l'approssimazione a facce infinite è abbastanza buona ma in un condensatore finito non tiene conto che nel bordo le linee di campo non sono perpendicolari alle facce ma curvano. 
\begin{figure}[h!]
	\centering
	\includegraphics[width=0.4\linewidth]{../images/condensatore-piano-2}
	\caption{Le linee di campo di un condensatore finito curvano in corrispondenza dei bordi}
	\label{fig:condensatore-piano-2}
\end{figure}
\FloatBarrier
\subsection*{Condensatori piani in parallelo}
\begin{definizione}[Collegamento in parallelo]
Si parla di collegamento in parallelo quando n componenti sono collegati ad una coppia di conduttori in modo che la differenza di potenziale sia la stessa per ciascun componente.
\end{definizione}
Consideriamo due condensatori collegati in parallelo come in figura
\begin{figure}[h!]
	\centering
	\includegraphics[width=0.4\linewidth]{../images/Condensatori_parallelo}
	\caption{Schema di due condensatori collegati in parallelo}
	\label{fig:condensatoriparallelo}
\end{figure}
\FloatBarrier
dove le due facce a sinistra sono cariche positivamente rispettivamente di \(+Q_1,\ +Q_2\) e le due a destra negativamente con carica \(-Q_1,\ -Q_2\). Sia \(V_a\) il potenziale del blocco a sinistra e \(V_b\) quello a destra. Vogliamo ricavare un'espressione della capacità totale del sistema di condensatori, note le capacità \(C_1\), \(C_2\). 
\[\Delta V = V_a - V_b = \frac{Q_{tot}}{C_{tot}}\] 
\[Q_{tot}=Q_1+Q_2=C_{tot}\Delta V = (C_1+C_2)\Delta V\]
\[\Rightarrow C_{tot} = C_1 +C_2\]
Dunque la capacità risultante da un sistema di condensatori in parallelo è pari alla somma delle capacità dei singoli condensatori. In generale
\[C_{tot} = \sum_i C_i\]
Questo risultato era prevedibile in quanto in un condensatore piano la capacità dipende dalla superficie e mettere condensatori in parallelo equivale ad aumentarla. 
\subsection*{Condensatori piani in serie}
\begin{definizione}[Collegamento in serie]
	Si parla di collegamento in serie quando n componenti sono collegati in modo da essere attraversati da una corrente elettrica (che definiremo in seguito) di uguale intensità. La differenza di potenziale dei diversi componenti collegati in serie può essere diversa per ciascun componente.
\end{definizione}
Consideriamo due conduttori in serie come in figura
\begin{figure}[h!]
	\centering
	\includegraphics[width=0.6\linewidth]{../images/Condensatori_serie}
	\caption{Schema di due condensatori collegati in serie}
	\label{fig:condensatoriserie}
\end{figure}
\FloatBarrier
Siano \(V_a\) \(V_e\) \(V_b\) rispettivamente i potensziali del blocco a sinistra, al centro e a destra. Notiamo subito che a differenza del caso precedente abbiamo due \(\Delta V\) diverse tra i condensatori e ci sono tre armature indipendenti. Notiamo che, non essendo collegata a nulla, l'armatura centrale E è neutra ed equipotenziale dunque sulle due facce di E ci deve essere la stessa carica, a sinistra negativa e a destra positiva. Ne segue che le cariche sono, rispettivamente, per ogni armatura, da sinistra a destra, +Q, -Q, +Q, -Q. 
\[\Delta V_{tot} = \frac{Q}{C_{tot}} = \Delta V_{ab}\]
\[\Delta V_{ab} = V_a-V_b=V_a-V_e+V_e-V_b = \frac{Q}{C_1}+\frac{Q}{C_2}\]
\[\frac{1}{C_{tot}} = \frac{1}{C_1} + \frac{1}{C_2}\]
\subsection{Condensatori dal punto di vista energetico}
Possiamo vedere un condensatore come un serbatoio di energia, vogliamo trovare un'espressione della quantità di energia immagazinata. Cominciamo chiarendo i termini del problema
\begin{definizione}[Energia]
L'energia è la capacità di un sistema o di un corpo di fare lavoro positivo
\end{definizione}
Da questa definizione risulta chiaro perché il lavoro di una forza conservativa, esprimibile come differenza di energia potenziale, è definito come \(U(A)-U(B)\) e non come \(U(B)-U(A)\) (nonostante la scelta sia arbitraria): in questo modo fare lavoro positivo vuol dire passare da uno stato iniziale ad energia maggiore U(A) per arrivare ad uno ad energia minore U(B) dove la differenza di energia è proprio il lavoro positivo fatto.\\
Osserviamo che l'energia immagazzinata in un condensatore discende dall'alta densità di carica in equilibrio nel sistema. L'energia immagazzinata deve coincidere (per la conservazione dell'energia) con il lavoro positivo fatto per portare le cariche da distanza infinita alla posizione che occupano nel condensatore. Possiamo considerare il lavoro necessario per avvicinare le cariche ad una ad una e poi generalizzare ad n cariche. Per la prima carica il lavoro è nullo, infatti possiamo scegliere il sistema di riferimento con al centro questa carica. per la seconda dobbiamo constrastare la forza di Coulomb dall'infinito fino ad \(r_{12}\), dobbiamo quindi applicare una forza $\mathbf{F}_{est} = -\mathbf{F}_{c}$.
\[L = -\int_{r_12}^{+\infty}\mathbf{F}_{c}d\mathbf{r} = -Q_2\int_{r_12}^{+\infty}\mathbf{E}d\mathbf{r} = Q_2V(r_{12}) = \frac{1}{4\pi\varepsilon_0}\frac{Q_1 Q_2}{r_{12}} = Q_1\frac{1}{4\pi\varepsilon_0}\frac{ Q_2}{r_{12}} = Q_1V(r_{21}) \]
\[\Rightarrow U_{12} = U_{21}\]
L'energia necessaria è quindi \(U_{12} = U_{21}\) 
Mettiamo ora la terza carica 
\[U_{13} = U_{31} = \frac{1}{4\pi\varepsilon_0}\frac{Q_1 Q_3}{r_{13}}\]
\[U_{23} = U_{32} = \frac{1}{4\pi\varepsilon_0}\frac{Q_2 Q_3}{r_{23}}\]
Dunque l'energia necessaria totale è 
\[U_{tot} =U_{13} + U_{23} + U_{12}  = \frac{U_{13} + U_{23} + U_{12} + U_{31} + U_{32} + U_{21}}{2}\]
Generalizzando ad n cariche
\[U_{tot} =\frac{1}{8\pi\varepsilon_0} \sum_i Q_i\sum_{j\neq i} \frac{Q_j}{r_{ij}} = \frac{1}{2}\sum_i Q_i V_i\]
dove quest'ultimo passaggio è stato possibile perché \(r_{ij} = r_{ji}\) ed esserndo il prodotto fra cariche commutativo \(Q_iQ_j = Q_jQ_i\).\\
Per passare ad una distribuzione di carica continua osserviamo
\[Q_i \to dq = \rho d\tau\]
dove $\rho$ è la densità di carica volumetrica e $\tau$ un volume infinitesimo. Ne segue
\[U_{tot} = \frac{1}{2}\int_{\tau} V\rho d\tau\]
Osserviamo ora che sull'armatura A sta una carica \(Q_a = C(V_a - V_b)\) e sull'armatura B \(Q_b = - Q_a\). L'energia totale del condensatore è
\[U_{tot} = \frac{1}{2}\sum_{i=1}^{2} Q_i V_i = \frac{1}{2} (Q_a V_a + Q_b V_b)= \frac{1}{2}C((V_a - V_b) V_a - (V_a - V_b) V_b) = \]
\[\frac{1}{2}C\Delta V^2 = \frac{1}{2}Q\Delta V = \frac{1}{2}\frac{Q^2}{C}\]
Possiamo seguire una strada concettuale diversa per ottenere il medesimo risultato: consideriamo due conduttori scarichi posti uno di fronte all'altro ed immaginiamo di spostare ad una ad una le cariche positive dell'armatura destra verso quella sinistra. La prima non necessita lavoro perché tutto è neutro ma già a partire dalla seconda esiste un campo $\mathbf{E}$ generato dalla prima carica spostata che produce una forza che si oppone al movimento da destra a sinistra della carica. Consideriamo il lavoro infinitesimo necessario per spostare di d$\mathbf{z}$ la carica dq da destra a sinistra. 
\[dL = dq\mathbf{E}d\mathbf{z} = -dqdV \]
dove il segno negativo deriva dal fatto che lo spostamento è in direzione opposta a quella del campo. Il lavoro esterno da "immettere" nel sistema è di segno opposto a quello che deve fare il sistema quindi
\[dL_{est} = -dL = dqdV\]
per ottenere il lavoro per spostare una carica dq da destra a sinistra integriamo sul percorso AB, indichiamo questo lavoro discreto con \(\Delta L\).
\[\Delta L_{est} = \int_{A}^{B} dL = \int_{A}^{B} dqdV = dq \Delta V_{ab}(q) \]
dove la differenza di potenziale dipende da q perché più cariche spostiamo da sinistra a destra più aumenta il potenziale. Per trovare il lavoro totale integriamo per tutte le cariche
\[L_{est} = \int_0^Q  \Delta V_{ab}(q) dq= \int_0^Q \frac{qdq}{4\pi\varepsilon_0 d} = \frac{q^2}{8\pi\varepsilon_0 d} = \frac{1}{2}Q\Delta V = \frac{1}{2} C \Delta V^2\]
Ma l'energia immagazzinata dove risiede? Potremmo pensare che questa domanda non è rilevante in quanto il principio fondamentale che vogliamo sia rispettato è la conservazione dell'energia, indipendentemente da dove è conservata. Tuttavia a ben pensarci la conservazione dell'energia tout court ammetterebbe ancora che l'energia può scomparire in un punto e riapparire in un altro senza che nello spazio compreso avvenga nulla; se invece troviamo un oggetto fisico a cui è legata l'energia elettrica, stabiliamo una formulazione più forte della conservazione dell'energia: la conservazione locale dell'energia. Abbiamo appena visto che spostando le cariche aumenta la differenza di potenziale e quindi il campo elettrico, l'energia è immagazzinata nel campo elettrico: in generale dove c'è campo elettrico c'è energia. In questo modo, visto che il campo elettrico non ha discontinuità e non scompare e riappare casualmente, abbiamo un principio di conservazione dell'energia più "forte". Questa formulazione è necessaria perché se non riuscissimo ad individuare con precisione il luogo in cui è immagazzinata l'energia, dal momento che massa ed energia sono equivalenti (\(E = mc^2\)), allora non saremmo in grado di dtereminare la posizione delle masse con precisione e quindi la teoria della gravità sarebbe incompleta. Usando solo la teoria dell'elettrostatica non sappiamo dove risiede l'energia ma la teoria dell'elettromagnetismo completa offre molte più informazioni.\\
Possiamo sostituire all'ultima espressione ottenuta due espressioni viste in precedenza per la capacità di un condensatore piano e per la differenza di potenziale di questo 
\[U_{tot} = [\frac{1}{2}\varepsilon_0 E^2](Sd) \]
\begin{align}\label{eq:densità_energia_elettrica}
	u_e \equiv \frac{1}{2}\varepsilon_0 E^2
\end{align}
quest'ultima è detta "densità di energia", in questo caso è stata ricavata per un condensatore piano a facce parallele ma vedremo in seguito che questo è un caso particolare di un teorema molto più generale, il teorema di Poynting, e che questo è sempre il valore della densità di energia (locale) del campo elettrico.
\newpage
\section{Dipoli elettrici}
\begin{definizione}[Dipolo elettrico]
	Un dipolo elettrico, in elettrostatica, è un sistema composto da due cariche elettriche uguali e di segno opposto e separate da una distanza costante nel tempo.
\end{definizione}
\begin{figure}[h!]
	\centering
	\includegraphics[width=0.4\linewidth]{../images/dipolo_elettrico}
	\caption{Schema delle linee di campo di un dipolo elettrico.}
	\label{fig:dipoloelettrico}
\end{figure}
\FloatBarrier
Usando solamente il principio di sovrapposizione possiamo ottenere il campo generato da un dipolo elettrico su un punto posto sullo stesso asse del dipolo.
\begin{esercizio}[Campo elettrostatico generato da due cariche puntiformi]
	Si considerino due cariche poste per semplicità entrambe sull'asse z e a distanza uguale dall'origine (una dal lato positivo ed una dal lato negativo) e di distanza d fra di esse, vogliamo ottenere un'espressione del campo elettrico che genera in un punto P che giace sull'asse y a distanza r dall'asse z. Si noti che questo è un caso particolare (date le simmetrie geometriche con cui abbiamo costruito il sistema) in cui è presente solo una componente parallela al dipolo. Consideriamo inizialmente due cariche di segno opposto (quella positiva sopra e quella negativa sotto), si traccino i vettori \(\mathbf{r}_+ \ \mathbf{r}_- \) congiungenti le cariche al punto P, dato il segno negativo della carica in basso, il vettore campo avrà verso entrante (verso la carica) mentre il campo generato dalla carica in alto è uscente dalla carica e quindi diretto verso il punto P. Sommando i vettori di campo otteniamo che il campo risultante ha direzione parallela all'asse z e verso negativo (-$\hat{\mathbf{k}}$). Chiamiamo y il modulo della proiezione sull'asse z del vettore $\mathbf{r}_+$.
	\[\hat{\mathbf{r}}_+ =\frac{ 0\hat{\mathbf{i}}+y \hat{\mathbf{j}}-\frac{d}{2}\hat{\mathbf{k}}}{r}\]
	\[\hat{\mathbf{r}}_- = \frac{0\hat{\mathbf{i}}+y \hat{\mathbf{j}}+\frac{d}{2}\hat{\mathbf{k}}}{r}\]
	\[\mathbf{E}_{//} = \mathbf{E}_{+} + \mathbf{E}_{-} = \frac{1}{4\pi\varepsilon_0}\frac{q}{r^2}\left(\hat{\mathbf{r}}_+-\hat{\mathbf{r}}_-\right)=\]
	\[\frac{1}{4\pi\varepsilon_0}\frac{q}{r^3}\left[(y\hat{\mathbf{j}}-\frac{d}{2}\hat{\mathbf{k}})-(y\hat{\mathbf{j}}+\frac{d}{2}\hat{\mathbf{k}})\right]= -\frac{1}{4\pi\varepsilon_0}\frac{q d}{r^3}\hat{\mathbf{k}}\]
	Possiamo definire una nuova grandezza legata ad una coppia di cariche opposte: il \textbf{momento di dipolo} ed esprimere il campo in funzione di esso.
	\[\mathbf{p} \equiv qd\mathbf{k}\]
	\begin{align}\label{eq:campo_dipolo}
		\mathbf{E}_{//} = \frac{1}{4\pi\varepsilon_0}\frac{\mathbf{p}}{r^3}
	\end{align}
	Se invece le cariche fossero state dello stesso segno avremmo ottenuto con ragionamenti analoghi ma cambiando il segno del vettore
	\[\mathbf{E}_{\perp} = \frac{1}{4\pi\varepsilon_0}\frac{q2y}{r^3}\hat{\mathbf{j}}\]
\end{esercizio}
Finora abbiamo ottenuto solamente la componente del campo parallela al dipolo. Sfruttando il potenziale elettrostatico possiamo ottenere un'espressione più generale
\begin{esercizio}[Espressione generale del campo generato da un dipolo elettrico]
	Consideriamo un dipolo elettrico come in figura dove chiamiamo a la distanza fra le due cariche, effettuiamo l'approssimazione \(r>>a\). Sia P il punto a cui puntano i vettori \(\mathbf{r_1}, \mathbf{r_2}, \mathbf{r}\). 
	\begin{figure}[h!]
		\centering
		\includegraphics[width=0.4\linewidth]{../images/dipolo}
		\caption{Schema di un dipolo elettrico}
		\label{fig:dipolo}
	\end{figure}
	\FloatBarrier
	Il potenziale sul punto P generato dalle due cariche è
	\[V(P) = \frac{q}{4\pi\varepsilon_0}\left(\frac{1}{r_1}-\frac{1}{r_2}\right) =  \frac{q}{4\pi\varepsilon_0}\frac{r_2-r_1}{r_1r_2}\]
	Osserviamo che per l'approssimazione fatta possiamo scrivere
	\[\Delta r = r_2-r_1 \simeq a\cos\theta\]
	\[r_1r_2 \simeq r^2\]
	\[\Rightarrow V(P) = \frac{q a \cos\theta}{4\pi\varepsilon_0 r^2}\]
	Ricordiamo che per definizione il momento di dipolo elettrico è
	\[\mathbf{p} = q\mathbf{a} = p\hat{\mathbf{k}}\]
	\[\Rightarrow V(P) = \frac{p \cos\theta}{4\pi\varepsilon_0 r^2} = \frac{\mathbf{p}\mathbf{r}}{4\pi\varepsilon_0 r^3} = \frac{pz}{4\pi\varepsilon_0 (x^2+y^2+z^2)^{\frac{3}{2}}}\]
	Calcoliamo ora le componenti parallela e perpendicolare del campo elettrico generato dal dipolo
	\[\mathbf{E}_{//} = \mathbf{E}_z = -\frac{\d V}{\d z} \hat{\mathbf{k}} = \frac{p(3\cos^2\theta-1)}{4\pi\varepsilon_0 r^3}\]
	dove nello svolgere i passaggi è stata usata \(z = r\cos\theta\). Analogamente possiamo calcolare il campo sull'asse x e sull'asse y
	\[E_x = -\frac{\d V}{\d x} \hat{\mathbf{i}}=\frac{p(3zx)}{4\pi\varepsilon_0 r^5}\]
	\[E_y = -\frac{\d V}{\d y} \hat{\mathbf{j}}=\frac{p(3zy)}{4\pi\varepsilon_0 r^5}\]
	\[E_{\perp} = \sqrt{E_x^2+E_y^2} = \frac{p}{4\pi\varepsilon_0r^3}3\cos\theta\sin\theta\]
	dove nei passaggi è stato usato \(r_{\perp} = \sqrt{x^2+y^2} = r\sin\theta\)
	Ne segue che se \(\theta = 0^\circ \)
	\[E_{\perp} = 0\]
	\[E_{//} = \frac{p}{2\pi\varepsilon_0r^3}\]
	mentre se \(\theta = 90^\circ\)
	\[E_{\perp} = 0^\circ\]
	\[E_{//} = -\frac{p}{4\pi\varepsilon_0r^3}\]
\end{esercizio}
Che è lo stesso risultato ottenuto nell'esercizio precedente.\\
Talvolta può risultare utile studiare un dipolo in coordinate polari quindi rispetto al raggio $\rho$ e gli angoli $\theta$ e $\gamma$. in questo caso il campo elettrico si esprime come 
\[E_\rho = -\frac{\d V}{\d \rho}\]
\[E_\gamma = -\frac{1}{r\sin\theta}\frac{\d V}{\d \gamma}\]
\[E_\theta = -\frac{1}{r}\frac{\d V}{\d \theta}\]
\subsection{Dipolo in un campo elettrico}
Consideriamo ora un dipolo formato da due cariche puntiformi immerso in campo elettrostatico $\mathbf{E}$, calcoliamo l'energia potenziale del dipolo calcolando il lavoro necessario ad avvicinare le due cariche partendo dall'infinito (entrambe sotto l'effetto del campo $\mathbf{E}$)
\[L = \int_{\infty}^{\mathbf{r}+\mathbf{a}} qE(r) dr -\int_{\infty}^{\mathbf{r}} qE(r) dr =\int_{\mathbf{r}}^{\mathbf{r}+\mathbf{a}} qE(r) dr= U(\mathbf{r}+\mathbf{a})-U(\mathbf{r}) = qV(\mathbf{r}+\mathbf{a})-qV(\mathbf{r})\]
Considerato che $||\mathbf{a}||<<||\mathbf{r}||$, possiamo interpretare le componenti \((a_x, a_y, a_z)\) come tendenti a zero quindi \((a_x, a_y, a_z)\simeq (\d x, \d y, \d z)\)
\[U = q[V(x+a_x, y+a_y, z+a_z)-V(x, y, z)] = qdV = q\left[\frac{\d V}{\d x} a_x + \frac{\d V}{\d y}a_y + \frac{\d V}{\d z}a_z\right] =\]
\[ q(-\mathbf{E}\mathbf{a}) = -\mathbf{p}\mathbf{E}\]
Da cui segue, ricordando che \(\mathbf{F} = \mathbf{\n} U\)
\[\mathbf{F} = \mathbf{p}\mathbf{\n}\mathbf{E}\]
Notiamo che abbiamo una condizione di equilibrio stabile quando questa funzione presenta un minimo, ovvero per \(\theta= 0^\circ,\ 180^\circ\). inoltre notiamo che, avendo considerato $\mathbf{a}$ infinitesimo, possiamo considerare in campo elettrico uguale nelle due cariche quindi la forza totale su un generico punto risulta 
\[\mathbf{F}_{tot} =\mathbf{F}_++\mathbf{F}_- = q\mathbf{E}-q\mathbf{E} = 0\]
Il momento delle forze totale invece è
\[\mathbf{M}_{tot} = \mathbf{r}_+\wedge\mathbf{F}_+ + \mathbf{r}_-\wedge\mathbf{F}_- = \mathbf{r}_+\wedge q\mathbf{E}_+ + \mathbf{r}_\wedge q\mathbf{E}_- = (\mathbf{r}_+ - \mathbf{r}_-)\wedge q \mathbf{E} = (\mathbf{a}q)\wedge\mathbf{E} = \mathbf{p}\wedge\mathbf{E}\]
Notiamo che essendo il momento \(pE\sin\theta\), si ha che esso è nullo per \(\theta= 0^\circ,\ 180^\circ\), gli unici angoli in cui il sistema è stabile. 
\subsection{Sistema di cariche come dipolo}
Consideriamo ora un sistema di cariche discrete (il caso uniforme è analogo ma con gli integrali al posto delle sommatorie). Per il principio di sovrapposizione possiamo esprimere il potenziale totale sul punto P come somma dei potenziali generati dalle i cariche \(q_i\).
\begin{figure}[h!]
	\centering
	\includegraphics[width=0.9\linewidth]{../images/sistema_dipolo}
	\caption{Schema del sistema di cariche}
	\label{fig:sistemadipolo}
\end{figure}
\FloatBarrier
\[V(P) = \frac{1}{4\pi\varepsilon_0}\sum_i \frac{q_i}{r_i}\]
Ipotizziamo che la dimensione d del sistema sia molto più piccola della distanza dal punto P \(d<<r\), ne seguirà che anche \(d_i<<r_i\). Possiamo dunque scrivere le seguenti relazioni
 \[\mathbf{r}_i = \mathbf{r} - \mathbf{d}_i\]
 \[r_i \simeq r-d_i\cos\theta_i\]
 \[r-r_i = d_i\cos_i = \mathbf{d}\cdot\hat{\mathbf{r}}\]
 \[\frac{1}{r_i} = \frac{1}{(r-\mathbf{d}_i\cdot\hat{\mathbf{r}})}\frac{r+\mathbf{d}_i\cdot\hat{\mathbf{r}}}{r+\mathbf{d}_i\cdot\hat{\mathbf{r}}} = \frac{r+\mathbf{d}_i\cdot\mathbf{r}}{r^2-(\mathbf{d}\hat{\mathbf{r}})^2} \simeq \frac{r+\mathbf{d}_i\cdot\mathbf{r}}{r^2}\]
 \[\Rightarrow V(P) =\frac{1}{4\pi\varepsilon_0}\sum_i \frac{q_i}{r^2}(r+\mathbf{d}_i\cdot\hat{\mathbf{r}}) = \frac{1}{4\pi\varepsilon_0}\frac{Q}{r}+\frac{1}{4\pi\varepsilon_0}\frac{(\sum_i(q_i\mathbf{d}_i))\hat{\mathbf{r}}}{r^2}\]
 Possiamo generalizzare la definizione di momento di dipolo ad un sistema discreto 
 \[\mathbf{p} = \sum_i q_i\mathbf{d}_i\]
 \[V(P) = \frac{1}{4\pi\varepsilon_0}\frac{Q}{r}+\frac{1}{4\pi\varepsilon_0}\frac{\mathbf{p}\hat{\mathbf{r}}}{r^2}\]
 Le due componenti del potenziali sono rispettivamente dette di monopolo \(V_0\) e di dipolo \(V_{dip}\), la prima equivale al potenziale che il sistema genererebbe se tutta la carica fosse concentrata in un punto mentre il secondo tiene conto della distribuzione di carica del sistema. Ne segue che anche se al carica totale Q fosse nulla, un sistema può comunque generare potenziale se le sue cariche sono disposte in certo modo. Possiamo dimostrare che il secondo membro è trascurabile rispetto al primo infatti
 \[\frac{V_{dip}}{V_0} = \frac{|\mathbf{p}\cdot \hat{\mathbf{r}}|}{rQ}\leq\frac{|\mathbf{p}|}{rQ} = \frac{|\sum_i q_i d_i|}{rQ}\leq \frac{|(\sum_i q_i)d|}{rQ} = \frac{d}{r}<<1\]
 In generale se cambio l'origine del sistema cambiano i \(q_i\) e i \(d_i\) quindi \(\mathbf{p}\neq\mathbf{p}'\). Tuttavia se \(\sum_i q_i = Q_{tot} = 0\) allora $\mathbf{p}$ diventa indipendente da come scegliamo il sistema di riferimento. 
\[\mathbf{d}_i = \mathbf{R}+\mathbf{d}'_i\]
\[\mathbf{p} = \sum_iq_id_i = \sum_iq_i\mathbf{R}+\sum_iq_i\mathbf{d}'_i=QR+\mathbf{p}'\]
Se \(Q = 0\) si ha \(\mathbf{p} = \mathbf{p}'\). L'analogia con la statica è in questo caso evidente.\\
Possiamo riscrivere $\mathbf{p}$ e definire il "centro di carica" per le cariche positive e negative 
\[\mathbf{p} = \sum_iq_id_i = \sum_i|q_i^+|\mathbf{d_i}-\sum_i|q_i^-|\mathbf{d_i}\]
\[\mathbf{d}_+ = \frac{\sum_i|q_i^+|\mathbf{d_i}}{\sum_i|q_i^+|} = \frac{\sum_i|q_i^+|\mathbf{d_i}}{|Q^+|}\]
\[\mathbf{d}_- = \frac{\sum_i|q_i^-|\mathbf{d_i}}{\sum_i|q_i^-|} = \frac{\sum_i|q_i^-|\mathbf{d_i}}{|Q^-|}\]
\[\mathbf{p} = Q^+\mathbf{d}^+-|Q^-|\mathbf{d}^-\]
Se il corpo è neutro abbiamo \(|Q^+|=|Q^-|=Q\) quindi \(\mathbf{p} = Q(\mathbf{d}^+-\mathbf{d}^-)\). Ne segue che $\mathbf{p}$ è nullo se i due centri di carica coincidono. Notiamo che un qualsiasi corpo con una generica distribuzione di carica può essere visto come un dipolo elettrico. Ad esempio, nell'atomo di idrogeno si ha un nucleo positivo che ha centro di carica nel suo centro attorniato da una nuvola elettronica sferica \(1s^1\) con centro nel centro dell'atomo, i due centri di carica coincidono e l'atomo d'idrogeno presenta momento di dipolo nullo. L'acqua invece, avendo la tipica forma a V, ha centri di carica positiva e negativa diversi da cui discende la forte polarità della molecola d'acqua. Inoltre, anche se un atomo 
\begin{figure}[h!]
	\centering
	\includegraphics[width=0.4\linewidth]{../images/dielettrico_idrogeno_1}\quad
	\includegraphics[width=0.5\linewidth]{../images/dipolo_H_H2O}
	\caption{L'idrogeno non presenta momento di dipolo mentre l'acqua si.}
	\label{fig:dipolohh2o}
\end{figure}
\FloatBarrier
\newpage
\section{Dielettrici}
Come già accennato, un dielettrico è un materiale schematizzabile come un reticolo cristallino in cui tutti gli elettroni sono legati. Ha proprietà isolanti ed è contrapposto ai materiali conduttori.\\
Un dielettrico, nonostante abbia gli elettroni "fissi" può polarizzarsi se esposto ad un campo elettromagnetico.
\begin{figure}[h!]
	\centering
	\includegraphics[width=0.4\linewidth]{../images/dielettrico_idrogeno_1}\quad
	\includegraphics[width=0.4\linewidth]{../images/dielettrico_idrogeno_2}
	\caption{Un dielettrico sottoposto ad un campo elettrico si polarizza.}
	\label{fig:dielettricoidrogeno}
\end{figure}
\FloatBarrier
\begin{definizione}[Dielettrico]
	Qualsiasi sostanza non conduttrice che, immersa in un campo elettrico esterno, dà luogo a fenomeni di polarizzazione più o meno intensi, che producono una variazione del campo elettrico totale.
\end{definizione}
Consideriamo un condensatore a facce piane parallele, se fra le armature vi è il vuoto abbiamo le seguenti informazioni:
\[C_0 = \frac{\varepsilon_0S}{d}\]
\[E_0 = \frac{\sigma}{\varepsilon_0} = \frac{Q}{\varepsilon_0 S}\]
\[\Delta V_0 = \int_{0}^{d}\mathbf{E}d\mathbf{r} = Ed = \frac{Q}{C_0}\]
Se però fra le armature poniamo una lastra metallica (conduttrice) di spessore D, otteniamo
\[\Delta V = \int_{0}^{d}\mathbf{E}d\mathbf{r} = \int_{0}^{d'}\mathbf{E}d\mathbf{r} + \int_{d'+D}^{d}\mathbf{E}d\mathbf{r} = Ed' + E(d-(d'+D)) = E(d-D) < \Delta V_0 \]
La differenza di potenziale diminuisce di ED e questa quantità è indipendente dalla posizione della lastra rispetto le due armature (se sia più vicina ad una o all'altra).\\
Se poniamo una lastra di dielettrico invece la differenza di potenziale riduce ma in quantità minore. Se aumentiamo lo spessore D la differenza di potenziale diminuisce fino ad un minimo (quando il dielettrico occupa tutto lo spazio fra le armature quindi \(D = d\)) $\Delta V_k$. Si nota sperimentalmente che 
\[\Delta V_k = \frac{\Delta V_0}{k}\]
dove k è una costante adimensionale che dipende dal materiale detta "costante dielettrica relativa" (contrassegneremo con una k al pedice le grandezze caratteristiche di sistemi in cui è presente un dielettrico di costante dielettrica relativa k). Possiamo ricalcolare tutte le quantità prima elencate
\[C_k = \frac{Q}{\Delta V_k} = \frac{Qk}{\Delta V_0} = kC_0 = \frac{(k\varepsilon_0)S}{d} = \frac{\varepsilon S}{d}\]
dove $\varepsilon = k\varepsilon_0$ è la costante dielettrica del materiale. 
\[\Delta V_k = \frac{\Delta V_0}{k} = \frac{1}{k}\int_{0}^{d}\mathbf{E}d\mathbf{r} = \int_{0}^{d}\mathbf{E}_kd\mathbf{r} = E_kd = \frac{E}{k}d \]
\[\Rightarrow E_k = \frac{E}{k}= \frac{\sigma}{\varepsilon_0 k} = \frac{\sigma_k}{\varepsilon_0}\]
\[\Rightarrow \sigma_k = \frac{\sigma}{k}\]
dove $\sigma_k$ è la densità di carica sulla superficie del condensatore.\\
La densità di carica sulla superficie del condensatore è più bassa di quella che ci sarebbe se il dielettrico non ci fosse; ipotizziamo (lo dimostreremo) che questa diminuzione di carica è dovuta al fatto che sulla superficie del dielettrico è presente una densità di carica di segno opposto \(\sigma_p\). Sappiamo che anche nel caso del dielettrico come in quello del conduttore le cariche stanno in superficie perché, se applichiamo la legge di Gauss considerando un cilindro che parte dall'armatura superiore e finisce all'interno del dielettrico, indipendentemente dall'altezza dal cilindro, troviamo lo stesso valore di carica totale al suo interno.
\[\oint_\Sigma \mathbf{E}d\mathbf{s} = \frac{Q_{tot}}{\varepsilon_0} = E_k A \]
la carica totale è indipendente dall'altezza del cilindro scelto quindi la carica non può stare all'interno del cilindro.\\
Riprendendo la precedente espressione, possiamo scrivere
 \[\frac{Q_{tot}}{\varepsilon_0} = E_k A = \frac{\sigma A - \sigma_p A}{\varepsilon_0}\]
dove abbiamo sottratto alla $\sigma$ che ci sarebbe in un conduttore alla $\sigma_p$, si noti che siamo sicuri della presenza di una $\sigma_p$ di segno opposto a quella dell'armatura  sappiamo che \(\sigma_k<\sigma\) e quindi possiamo scrivere \(\sigma_k = \sigma -\sigma_p\). Possiamo quindi svolgere le seguenti considerazioni
\[E_k = \frac{\sigma }{\varepsilon_0 k} = \frac{\sigma}{\varepsilon_0}\left[\frac{1}{k}-1+1\right] = \frac{\sigma}{\varepsilon_0}\left[1-\frac{k-1}{k}\right] = \frac{\sigma}{\varepsilon_0}-\frac{\sigma_p}{\varepsilon_0}\]
\[\Rightarrow \sigma_p = \frac{k-1}{k}\sigma\]
Notiamo che più k è grande più grande è la carica di polarizzazione.
\begin{definizione}[dielettrico omogeneo (o uniforme)]
	Un dielettrico è detto omogeneo (o, equivalentemente, uniforme) se k non dipende dalla posizione.
\end{definizione}
Fin ora abbiamo solamente tratto deduzioni matematiche senza chiederci il motivo fisico per cui otteniamo questi risultati. Ciò che avviene è analogo a quanto visto per l'atomo di idrogeno in un campo elettrico: le nuvole elettroniche degli atomi nel dielettrico si spostano creando una polarizzazione. 
\[\delta = \mathbf{d}_+ -\mathbf{d}_- \neq 0\]
\[\mathbf{p} = q\delta\neq 0\]
dove questo momento di dipolo è quello generato da un singolo atomo. Se il dielettrico è omogeneo La polarizzazione avviene in tutti gli atomi del dielettrico ma negli strati interni la asimmetria di carica di uno strato è compensata da quello superiore ed inferiore; la polarizzazione crea dunque una densità di carica solo in superficie dove non è presente uno strato superiore o inferiore. Abbiamo così spiegato il perché della presenza di \(\sigma_p\) da cui scaturiscono le relazioni matematiche appena dedotte. Definiamo il \textbf{vettore densità di polarizzazione} come
\[\mathbf{P} = n\mathbf{p} = nq\delta = \frac{\Delta N}{\Delta \tau} q \delta \]  
dove n è la densità volumetrica di atomi e q è la carica di un'atomo (considerato come un dipolo) data da \(q = ze\) dove z è il numero atomico. Notiamo inoltre che più aumenta il campo elettrico più aumenta $\delta$ e quindi anche il vettore di polarizzazione.\\
Si nota sperimentalmente (ma si dedurrà a breve) che la maggior parte dei dielettrici rispettano la seguente relazione
\begin{align}\label{eq:polarizzazione_lineare}
	\mathbf{P} = \varepsilon_0(k-1)\mathbf{E} = \varepsilon_0 \chi \mathbf{E}
\end{align}
dove $\chi$ è detta \textbf{suscettibilità elettrica}. I dielettrici che rispettano questa condizione sono detti lineari perché p dipende linearmente da E, limiteremo la nostra trattazione a questo tipo di materiali. Di solito P è omogeneo all'interno del dielettrico ma esistono casi in dipende dalla posizione dunque \(\mathbf{P} = \mathbf{P}(x, y, z)\) e anche \(k = k(x, y, z)\). Inoltre nella maggior parte dei casi \(\mathbf{P} // \mathbf{E}\) (per come abbiamo scritto l'ultima equazione sembra l'unica opzione possibile) tuttavia questo è vero solo se k è uno scalare (che può essere uno scalare costante se il dielettrico è omogeneo o un campo scalare se non lo è) può accadere però che k non sia uno scalare ma un tensore 3x3, in questo caso la direzione di $\mathbf{P}$ può non essere la stessa di $\mathbf{E}$; il dielettrico si dice anisotropo (ad esempio i cristalli birifrangenti). Fin ora abbiamo quindi studiato solamente dielettrici lineari isotropi.
\begin{definizione}[dielettrico anisotropo]
	Un dielettrico è detto anisotropo se la costante dielettrica relativa ad esso associata è un tensore \(3\times 3\) detto tensore di polarizzazione. Contrariamente, un dielettrico isotropo ha come costante dielettrica relativa uno scalare.
\end{definizione}
\subsection{Interpretazione microscopica della polarizzazione di un dielettrico}
Abbiamo visto che quando avviene la polarizzazione si ridistribuiscono le cariche e si genera una densità di carica sulla superficie \(\sigma_p\), abbiamo giustificato matematicamente il fatto che questa si generi proprio in superficie con la legge di Gauss e ne abbiamo reso conto anche fisicamente mediante un'interpretazione microscopica della polarizzazione.\\
A causa di questa ridistribuzione di carica si crea un campo elettrico indotto di entità molto minore rispetto a quello che viene a crearsi in un conduttore (dove la polarizzazione è totale, cioè si ferma solo quando il campo elettrico esterno è totalmente schermato); la schermatura del campo esterno è quindi molto minore rispetto ad un conduttore e la differenza di potenziale fra le armature del condensatore si riduce in quantità minore. Continuerà dunque ad esistere un campo elettrico all'interno del dielettrico $\mathbf{E}$ (a differenza del caso del conduttore)
\[\mathbf{E} = \mathbf{E}_0+\mathbf{E}_p<\mathbf{E}_0\]
dove $\mathbf{E}_0$ è il campo esterno che già era presente e $\mathbf{E}_p$ è il campo di polarizzazione (che si oppone a quello preesistente).\\
Basandoci su questa interpretazione microscopica ci chiediamo quanto valga $\sigma_p$. Conoscendo la densità volumetrica di atomi nel dielettrico (n), la moltiplichiamo alla carica di un singolo atomo (ottenendo la densità di carica del dielettrico), che poi moltiplichiamo per uno spessore infinitesimo (\(\delta\)) corrispondente alla distanza fra i centri di carica positiva e negativa del singolo atomo, ottenendo così la densità superficiale.
\[\sigma_p = n q \delta\]
Ricordando che \(P = nq\delta\), abbiamo che $\sigma_p = P$. Se però il campo non è perpendicolare alla superficie del dielettrico ma vi entra con un angolo di incidenza $\theta$, allora la distanza fra i centri di carica negativo e positivo sarà di modulo $\delta$ ma orientata obliquamente, parallela al campo elettrico. Ne segue che la densità di carica decresce secondo il coseno dell'angolo di incidenza, possiamo quindi scrivere una forma più generale della densità superficiale di polarizzazione
\begin{align}\label{eq:densità_polarizzazione}
	\sigma_p = np\delta \cos\theta= |\mathbf{P}|\cos\theta = \mathbf{P}\hat{n}
\end{align}

Solitamente la densità di carica dovuta alla polarizzazione sta sulla superficie per i motivi precedentemente spiegati, tuttavia fin ora abbiamo implicitamente presunto che il dielettrico sia omogeneo, cioè, intuitivamente, che non presenti irregolarità al suo interno.
In generale, a livello locale, non tutte le cariche generate dalla polarizzazione di uno strato interno sono annullate da quelli adiacenti; si possono quindi avere cariche anche all'interno del dielettrico (vi sarà quindi una densità di carica volumetrica dovuta alla polarizzazione \(\rho_p\)).\\
Vediamo una trattazione matematica di questo fenomeno. Consideriamo una superficie $\Sigma$ all'interno del dielettrico, se non c'è campo esterno non c'è polarizzazione quindi abbiamo
\[E = V(\Sigma) Q = 0\]
\[Q_p = \int_{V(\Sigma)}\rho_p d\tau = 0\]
dove \(Q_p\) è la carica generata dalla polarizzazione.\\
In presenza di campo invece, nel caso di un dielettrico generale, si ridistribuiscono le cariche a causa della polarizzazione; ne segue che una carica dq è transitata attraverso la superficie $\Sigma$. Consideriamo una pozione di superficie di \(\Sigma\) infinitesima $d\mathbf{s}$ e il vettore di polarizzazione $\mathbf{P}$. La carica infinitesima transitata attraverso la superficie d$\mathbf{s}$ è
\[dq = \sigma_pds=(\mathbf{P}\hat{n})ds = \mathbf{P}d\mathbf{s}\]
Ne deduciamo che la carica passata per \(\Sigma\) la otteniamo con il flusso di $\mathbf{P}$ su $\Sigma$. Se $\mathbf{P}$ è costante all'interno di tutto il dielettrico (dielettrico omogeneo) tanta carica entra e tanta ne esce da $\Sigma$ quindi il flusso è nullo e la carica interna al dielettrico anche (\(\rho_p = 0\)). Se $\mathbf{P}$ non è uniforme (materiali anisotropi) allora il flusso non è nullo e di conseguenza vi è una carica $Q_p$ all'interno del dielettrico. 
\[ Q_p = -\oint_{\Sigma} \mathbf{P}d\mathbf{s} = \int_{V(\Sigma)}\rho_p d\tau \neq 0 \]
Applicando il teorema della divergenza otteniamo
\[-\oint_{\Sigma} \mathbf{P}d\mathbf{s} = \int_{V(\Sigma)}\rho_p d\tau = -\int_{V(\Sigma)}(\mathbf{\n}\mathbf{P})d\tau \]
\begin{align}\label{eq:gradiente_vettore_polarizzazione}
\Rightarrow \mathbf{\n}\mathbf{P} = -\rho_p
\end{align}

L'analogia di questa espressione con la forma differenziale della legge di Gauss è evidente. Se il materiale è isotropo la divergenza di $\mathbf{P}$ è nulla e così la \(\rho_p\).\\
Possiamo ora ricavare la legge sperimentale che lega $\mathbf{P}$ e $\mathbf{E}$ linearmente che precedentemente abbiamo presentato come un'evidenza sperimentale: consideriamo un condensatore piano a facce parallele dove fra le due armature è posto un dielettrico (che non occupa necessariamente tutto il volume in mezzo), il campo esterno al dielettrico è $\mathbf{E}_0$ mentre quello interno è $\mathbf{E}$
\[\mathbf{E}_0 = \frac{\sigma}{\varepsilon_0}\]
\[E = \frac{\sigma - \sigma_p}{\varepsilon_0} = E_0 -\frac{P}{\varepsilon_0}\]
\[P = \varepsilon_0 E_0- \varepsilon_0 E\]
\[\Rightarrow P = \varepsilon_0 E_0 (1-\frac{1}{k}) = \varepsilon_0 E(k-1) \]
\subsection{Lo spostamento elettrico $\mathbf{D}$}
Dal teorema di Gauss in forma differenziale (si faccia attenzione al fatto che $\mathbf{E}$ è il campo interno al dielettrico)
\[\mathbf{\n}\mathbf{E} = \frac{\rho}{\varepsilon_0} = \frac{\rho_p + \rho_l}{\varepsilon_0} = \frac{\rho_l}{\varepsilon_0}-\frac{1}{\varepsilon_0}\mathbf{\n}\mathbf{P}\]
\[\Rightarrow \varepsilon_0(\mathbf{\n}\mathbf{E})+\mathbf{\n}\mathbf{P} = \mathbf{\n}(\varepsilon_0\mathbf{E}+\mathbf{P})= \rho_l\]
dove $\rho_l$ è la densità di carica libera all'interno del materiale (questa è un espressione generale che tiene conto di materiali dielettrici, conduttori ed intermedi). Questa densità di carica sta sempre in superficie perché le cariche essendo libere di muoversi si dispongono in superficie come visto nel caso dei conduttori.\\
Sembra naturale introdurre una variabile la cui divergenza sia uguale alla densità di carica libera perché la divergenza di E è legata alla densità di carica totale e quella di P a quella di polarizzazione. Introduciamo quindi il vettore di \textbf{spostamento elettrico}
\[\mathbf{D} \equiv \varepsilon_0\mathbf{E}+\mathbf{P}\]
\[\Rightarrow \mathbf{\n}\mathbf{D} =\rho_l\]
\[\Rightarrow \oint_\Sigma\mathbf{D}d\mathbf{s} = Q_l\]
Si noti che $\mathbf{D}$ non aggiunge nessuna informazione fisica (a differenza di $\mathbf{P}$ che ci dice come si polarizza il materiale) ma ha una grande utilità pratica poiché permette di calcolare semplicemente la carica interna al dielettrico prescindendo dalle sue caratteristiche. Nel vuoto, essendo $\mathbf{P} = 0$, si ha $\mathbf{D} = \varepsilon_0\mathbf{E}$. Inoltre, notiamo che se in un dielettrico la densità di carica libera interna è nulla si ha
\[\rho_l = 0 \Rightarrow \mathbf{\n}\mathbf{D} =\mathbf{\n}(\varepsilon_0\mathbf{E}+\mathbf{P}) = 0 \Rightarrow \mathbf{\n}\mathbf{E}\neq 0\]
da cui deriva che il campo all'interno del dielettrico è non conservativo.\\
Infine possiamo trovare una semplice espressione di $\mathbf{D}$ ricordando la \ref{eq:polarizzazione_lineare}
\begin{align}\label{eq:spostamento_elettrico}
\mathbf{D} = \varepsilon_0\mathbf{E}+\mathbf{P} = \varepsilon_0\mathbf{E}+ \varepsilon_0 \mathbf{E}(k-1)=  \varepsilon_0\mathbf{E}- \varepsilon_0 \mathbf{E}+\varepsilon_0 \mathbf{E}k=\varepsilon_0 k\mathbf{E} = \varepsilon \mathbf{E}
\end{align}

\subsection{Dielettrici anisotropi}
Dalla relazione (\ref{eq:polarizzazione_lineare}) possiamo sostituire E alla definizione di D e differenziare per ottenere
\[\mathbf{\n}\mathbf{P} = \mathbf{\n}\left(\frac{k-1}{k}\mathbf{D}\right)\]
Se il materiale dielettrico è omogeneo, differenziando $\mathbf{P}$, essendo k costante, si ottiene 
\[\mathbf{\n}\mathbf{P} = \frac{k-1}{k}\mathbf{\n}\mathbf{D} = 0\] 
se invece il materiale non è omogeneo \(k=k(x,y,z)\) si deve applicare la differenziazione di prodotto ottenendo
\[\mathbf{\n}\mathbf{P} = \frac{k-1}{k}\mathbf{\n}\mathbf{D} + \mathbf{D}\mathbf{\n}\left(\frac{k-1}{k}\right)= \mathbf{D}\mathbf{\n}\left(\frac{k-1}{k}\right)\neq 0\]
Osserviamo inoltre che se in materiale è lineare ed omogeneo, la direzione in cui si polarizzano gli atomi è la stessa di quella del campo elettrico perché nella \ref{eq:polarizzazione_lineare} P ed E sono legate da uno scalare quindi
\[\mathbf{D}//\mathbf{E}//\mathbf{P}\]
se però k è una matrice in generale si ha che questi non sono paralleli. L'espressione più generale di $\mathbf{D}$ e $\mathbf{P}$ per materiali anisotropi è dunque
\[\mathbf{P} = \varepsilon_0 (\mathbf{k}-I)\mathbf{E}\]
\[\mathbf{D} = \varepsilon_0\mathbf{k}\mathbf{E}\]
Possiamo esplicitare quest'ultima equazione come segue
\begin{align*}
	\begin{pmatrix}
	P_x\\
	P_y\\
	P_z
	\end{pmatrix}
=\varepsilon_0
	\begin{pmatrix}
		k_{11}-1&k_{12}&k_{13}\\
		k_{21}&k_{22}-1&k_{23}\\
		k_{31}&k_{32}&k_{33}-1\\
	\end{pmatrix}
	\begin{pmatrix}
		E_x\\
		E_y\\
		E_z
	\end{pmatrix}
\end{align*}
dove I è la matrice identità. La matrice $\mathbf{k}$ è simmetrica (\(k_{ij} = k_{ji}\)) quindi per il teorema spettrale è diagonalizzabile (analogia con il tensore d'inerzia in meccanica). La diagonalizzazione dal punto di vista fisico corrisponde ad un cambio di sistema di riferimento, ottenuto per rototraslazione, nel quale esistono al più solamente tre costanti dielettriche relative diverse (una per ogni asse).
\begin{align*}
	\begin{pmatrix}
		P'_x\\
		P'_y\\
		P'_z
	\end{pmatrix}
	=\varepsilon_0
	\begin{pmatrix}
		k'_{11}-1&0&0\\
		0&k'_{22}-1&0\\
		0&0&k'_{33}-1\\
	\end{pmatrix}
	\begin{pmatrix}
		E'_x\\
		E'_y\\
		E'_z
	\end{pmatrix}
\end{align*}
In questo sistema di riferimento abbiamo quindi
\begin{align*}
	&P'_x =(K_{11}-1)E'_x\\
	&P'_y =(K_{22}-1)E'_y\\
	&P'_z =(K_{33}-1)E'_z
\end{align*}
Il vettore P non è ancora parallelo ad E perché le costanti ne cambiano la direzione, se fossero tutte uguali (caso isotropo) allora si avrebbe \(\mathbf{E}//\mathbf{P}\). Gli assi a cui, a seguito della diagonalizzazione, è associato un diverso k, sono detti \textbf{assi ottici}, un esempio di materiale anisotropo con due diversi assi ottici è la calcite. Essendo la luce un'onda elettromagnetica, anch'essa risente delle diverse costanti dielettriche nei diversi assi ottici, cambiando la sua velocità e piegandosi in modi diversi a seconda del k; in materiali come la calcite le immagini si vedono dunque sdoppiate a causa della presenza di due assi ottici.
\begin{figure}[h!]
	\centering
	\includegraphics[width=0.6\linewidth]{../images/calcite1}\quad
	\includegraphics[width=0.4\linewidth]{../images/calcite2}
	\caption{Schema ed immagine della rifrazione della luce nella calcite, un dielettrico anisotropo con due assi ottici.}
	\label{fig:calcite1}
\end{figure}
\FloatBarrier
\subsection{Deduzione della legge di Snell (rifrazione)}
Abbiamo notato precedentemente che, essendo la circuitazione di E nulla, la componente tangente si conserva e abbiamo dimostrato che la discontinuità della componente normale è $\frac{\sigma}{\varepsilon_0}$. Consideriamo un campo elettromagnetico che attraversa una superficie sotto la quale è presente un materiale di costante dielettrica relativa \(k_1\) e sopra la quale una costante dielettrica relativa \(k_2\). Conoscendo la direzione della linea di campo entrante nella superficie, per conoscere la direzione della linea di campo uscente dovremmo conoscere $\sigma$, per sapere di quanto cambia la componente normale. Consideriamo una superficie cilindrica di area di base dA e altezza dh che attraversa la superficie perpendicolarmente, siano i vettori normali alle superfici di base \(d\mathbf{s}_1\) e d\(\mathbf{s}_2\). Calcolando il flusso del vettore spostamento elettrico otteniamo
\[\oint_{\Sigma}\mathbf{D}d\mathbf{s} = Q_l = 0\]
poiché nei dielettrici non ci sono cariche libere. Da cui segue
\[D_1ds_1\cos\theta_1 + D_2ds_2\cos\theta_2 = 0\]
Ma \(ds_1 = -ds_2\) quindi chiamando \(D_n\) la componente normale di D
\[D_{n1} = D_{n2}\]
cioè la componente normale di D si conserva, come quella tangente di E, nell'attraversare la superficie che connette un dielettrico ad un altro. Possiamo dunque ricavare $\mathbf{E_2}$
\begin{align*}
	&\begin{cases}
		E_1\sin\theta_1 &= E_2\sin\theta_2 \\
		D_1\cos\theta_1 &= D_2\cos\theta_2 
	\end{cases}
\\
	&\begin{cases}
		E_1\sin\theta_1 &= E_2\sin\theta_2 \\
		\varepsilon_0k_1E_1\cos\theta_1 &= \varepsilon_0k_2E_2\cos\theta_2 
	\end{cases}
\\
	&\Rightarrow \frac{\tan\theta_1}{\tan\theta_2} = \frac{k_1}{k_2}
\end{align*}
Ecco trovata una relazione che permette di trovare il nuovo angolo con cui esce la linea di campo dalla superficie. Essendo la luce un'onda elettromagnetica, anch'essa viene piegata nel passare da un mezzo ad un altro, questo fenomeno è comunemente detto rifrazione e l'ultima relazione ottenuta è la \textbf{legge di Snell}, ottenuta da Willebrord Snell (1580-1626) in modo sperimentale, molto prima della scoperta del campo elettrico e tutto ciò ad esso relativo. Tutte le leggi dell'ottica si ricavano a partire dalle leggi dell'elettromagnetismo. 
\subsection{Energia di un condensatore piano con dielettrico}
Consideriamo un condensatore piano a facce parallele dove lo spazio fra le armature è riempito da un dielettrico di costante dielettrica relativa k e la distanza fra le armature vale d. Sappiamo che
\[C_k = \frac{k\varepsilon_0 S}{d} = \frac{\varepsilon S}{d}\]
\[E_k = \frac{E}{k} = \frac{\sigma}{k\varepsilon_0} = \frac{\sigma}{\varepsilon}\]
Possiamo quindi ricavare l'energia immagazzinata nel condensatore come
\[U = \frac{1}{2}\frac{Q^2}{C_k} = \frac{1}{2}\frac{S^2\sigma^2 d}{\varepsilon S}\frac{\varepsilon}{\varepsilon} = \frac{1}{2}\frac{\sigma^2\varepsilon S d}{\varepsilon^2} = \frac{1}{2}E_k^2\varepsilon V\]
dividendo per il volume si ottiene la densità di energia di un condensatore
\[u = \frac{1}{2}E_k^2\varepsilon  = \frac{1}{2}E_k^2\varepsilon_0 k\]
Si noti che la differenza di energia fra un condensatore con in mezzo un dielettrico e quella con in mezzo il vuoto è data dalla polarizzazione del dielettrico: quando si polarizza il dielettrico non si fa altro che effettuare lavoro per "tirare" ed allontanare i centri di carica degli atomi, passando ad uno stato ad energia potenziale maggiore. La densità di energia necessaria a polarizzare un dielettrico è quindi espressa dalla differenza fra la densità di energia di un condensatore con in mezzo un dielettrico e quella con in mezzo il vuoto
\[\Delta u_p = u_k-u_0 = \frac{1}{2}\varepsilon_0 kE_k^2-\frac{1}{2}\varepsilon_0 E^2 = \frac{1}{2}\varepsilon_0 E^2(\frac{1}{k}-1)<0 \]
La densità è negativa perché il sistema costituito dal dielettrico, nel polarizzarsi, assorbe energia.\\
Nel caso più generale del dielettrico anisotropo si ha 
\[u = \frac{1}{2}\varepsilon_0 \mathbf{k} \mathbf{E} \mathbf{E}(\frac{1}{k}-1) =\frac{1}{2} \mathbf{D} \mathbf{E}\]
dove $\mathbf{k}$ è una matrice 3x3 che moltiplicata al primo $\mathbf{E}$ risulta nel vettore $\mathbf{D}$ che moltiplicato al secondo $\mathbf{E}$ da uno scalare.
\newpage
\section{Corrente elettrica stazionaria}
I conduttori sono tali in virtù della presenza di portatori di carica, questi possono essere positivi (protoni) o, come succede più frequentemente, negativi (elettroni). Ad esempio la densità di portatori di carica di rame e argento è
\[n_{Cu} = 8.5 \cdot 10^{28}\frac{e^-}{m^3}\]
\[n_{Ag} = 5.9\cdot 10^{28}\frac{e^-}{m^3}\]
Come si muovono i portatori di carica quando esiste un campo all'interno di un conduttore? Per saperlo è necessario avere uno strumento capace di generare un tale campo, questo è la pila, inventata da Alessandro Volta nel 1799 (prima di questa data uno studio sistematico di questi fenomeni era impossibile). La pila è un generatore di \textbf{forza elettromotrice} (che non è una forza) che intuitivamente è un qualcosa che genera movimento di elettroni. I portatori di carica si muovono solamente per agitazione termica in assenza di una f.e.m. quindi la media vettoriale delle velocità per agitazione termica è nulla
\[\mathbf{\overline{v}}_0= \sum_{i = 1}^{N}\frac{\mathbf{v}_i}{N}=0\]
Quando attacchiamo la pila si ha una forza sui portatori di carica che quindi sono accelerati; la velocità media non sarà più nulla
\[\mathbf{F}= \pm q\mathbf{E}= m \mathbf{a}\]
Se il campo è costante lungo il conduttore (quando si parla di conduttori possiamo immaginare, per fissare le idee, un filo conduttore), l'accelerazione è costante e il moto è uniformemente accelerato. Ciò vuol dire che le cariche, accelerate per un tempo sufficiente, possono raggiungere velocità infinita (o anche solamente raggiungere la velocità della luce)? Chiaramente no perché all'interno di un conduttore i portatori di carica nel muoversi urtano con il reticolo del conduttore deviando la propria traiettoria e quindi, in media, rallentando; si raggiungerà quindi una velocità di soglia in cui l' accelerazione elettrica e la decelerazione a causa degli urti si bilanciano. \'E evidente un'analogia con l'attrito viscoso
\[\mathbf{F} = \pm q\mathbf{E}-\eta\mathbf{v}=m\mathbf{a}\]
\[\pm q\mathbf{E}=\eta\mathbf{v}_{soglia} \Leftrightarrow \mathbf{a}= 0\]
La costante che compare in questo caso (e che nel caso dell'attrito viscoso è la viscosità) è la \textbf{resistività}, il concetto è lo stesso ma in questo caso è dal punto di vista microscopico. Hanno resistività nulla solamente il vuoto e i superconduttori, nei quali è possibile far raggiungere ai portatori di carica velocità prossime a quelle della luce (a causa di effetti relativistici). La velocità dei portatori di carica in modulo è ricavabile dalla teoria cinetica dei gas come valor medio della distribuzione di Maxwell-Boltzmann
\[\overline{v}_0=\sqrt{\frac{3k_b T}{m}}\]
i valori ottenuti da questa formula sono sottostimati di un fattore 10, di cui è possibile render conto solo introducendo la teoria quantistica (teoria di Fermi-Sommerfield).\\
Se è presente un campo nel conduttore (e quindi una f.e.m.) esisterà una velocità media di deriva delle cariche (le possiamo immaginare come una nuvola di portatori di carica che si muove disordinatamente e che trasla in una direzione). Vogliamo ricavare un'espressione della velocità di deriva; consideriamo un conduttore cilindrico con densità di portatori di carica n, all'interno del quale è presente un portatore di carica positivo (in modo da avere lo spostamento equiverso al campo). Consideriamo una superficie \(\Sigma\) che taglia il conduttore trasversalmente, sia $\Delta q$ la carica che attraversa $\Sigma$ nel tempo $\Delta t$
\begin{definizione}[Intensità di corrente]
	Definiamo l'intensità di corrente elettrica come
	\[i \equiv \lim_{\Delta t \to 0} \frac{\Delta q}{\Delta t}=\frac{dq}{dt}\]
	L'unità di misura è l'Ampere
	\[1 A = \frac{1 C}{1 s}\]
\end{definizione}
Quanta carica passa per \(d\mathbf{s}\) (porzione infinitesima di $\Sigma$) in un intervallo di tempo $\Delta t$? Se i portatori di carica hanno una velocità di deriva \(v_d\), quelli che passano $\Sigma$ in $\Delta t$ sono quelli contenuti nel volume di conduttore di base \(ds\cos\theta\) (la superficie $\Sigma$ è inclinata) e altezza $v_d\Delta t$, conoscendo la densità di portatori di carica, essendo i portatori di carica positivi solitamente i protoni, si ha che la densità di carica è \(\rho = n e^+\). Se $\Delta t \to 0$
\[dq = n e^+ d\tau = n e^+ v_d dt ds cos\theta\]

\[di = n e^+ v_d ds cos\theta = n e^+ \mathbf{v}_d\cdot d\mathbf{s}\]
Definiamo il vettore \textbf{densità di corrente}
\[\mathbf{J} \equiv n e^+ \mathbf{v}_d\]
\[di = \mathbf{J}d\mathbf{s}\]
\[i = \int_{\Sigma} \mathbf{J}d\mathbf{s} = \Phi_\Sigma(\mathbf{J})\]
se \(\mathbf{J}\) è costante su tutta la superficie allora la relazione si semplifica a 
\[i = J\Sigma\]
L'intensità di corrente è quindi il flusso del vettore densità di corrente sulla superficie. Qualora i portatori di carica fossero negativi, in J cambiano segno sia la carica (\(e^-\)) che la velocità quindi J rimane invariato; si ha sempre, indipendentemente dal tipo di portatore, \(\mathbf{J}\) // $\mathbf{E}$. Vediamo che è arbitrario assumere portatori positivi o negativi nella definizione dell'intensità di corrente, essendo più intuitivo il fatto che i portatori siano equiversi al campo scegliamo portatori di carica sempre positivi; ne segue che in un circuito immagineremo sempre portatori di carica che vanno dal punto a potenziale maggiore a quello minore ma nella realtà non è detto che ciò avvenga (anzi, è più frequente che i portatori siano negativi, come in tutti i conduttori). \\
Consideriamo ora una superficie chiusa $\Sigma$ in presenza di un campo vettoriale di densità di carica $\mathbf{J}$ che la attraversa; calcoliamone l'intensità di corrente
\[\oint_\Sigma \mathbf{J}d\mathbf{s} = -\frac{dq}{dt}\]
Intuitivamente, il segno meno deriva dal fatto che se l'intensità di corrente è positiva esce più carica di quanta ne entra quindi la carica interna è diminuita (derivata negativa). Essendo $\Sigma$ una superficie chiusa il vettore normale è sempre uscente, il prodotto scalare fra J e ds sarà quindi positivo quando J esce dalla superficie ($\theta<90^\circ$), negativo quando entra ($\theta>90^\circ$) e nullo quando è perpendicolare (i versi delle disuguaglianze cambiano a seconda del segno del portatore di carica).\\
Il \textbf{principio della continuità della carica} o \textbf{legge di conservazione della carica elettrica} enuncia che:\\
\textit{Il flusso della densità di corrente elettrica attraverso una qualunque superficie chiusa è pari alla variazione della carica elettrica situata nel volume racchiuso dalla superficie nell'unità di tempo.}\\
Possiamo applicare la legge di Gauss per ottenere
\[\oint_\Sigma \mathbf{J}d\mathbf{s} = \iiint_{V(\Sigma)} \mathbf{\n}\mathbf{J} d\tau\]
d'altra parte possiamo esprimere la carica dq come l'integrale di \(d\rho\) in un volume infinitesimo \(d\tau\) 
\[\oint_\Sigma \mathbf{J}d\mathbf{s} = -\iiint_{V(\Sigma)}\frac{d\rho}{dt}d\tau\]
Eguagliando le due precedenti si ottiene la forma differenziale del teorema di Gauss
\[\mathbf{\n}\mathbf{J} = -\frac{\d\rho}{\d t}\]
Nel caso stazionario (quello più comune, che solitamente considereremo) I è lo stesso in tutti i punti del conduttore ed è costante, che equivale a dire che nonostante vi sia un movimento delle cariche, la quantità di queste al'interno della superficie chiusa è sempre la stessa. 
\begin{definizione}[Regime di corrente stazionaria]\label{def:corrente_stazionaria}
	Un circuito è in regime di corrente stazionaria se si verifica
	\[\oint_\Sigma \mathbf{J}d\mathbf{s} = -\frac{dq}{dt} = 0\]
	o, in forma differenziale (puntuale)
	\[\n \mathbf{J}=0\]
\end{definizione}
\subsection{Le leggi di Ohm}
Vogliamo ricavare la relazione che lega J ad E. Storicamente quanto andremo a ricavare per via teorica è stato inizialmente osservato sperimentalmente e dall'esperienza se ne sono dedotte le leggi empiriche (macroscopiche); ciò avvenne nella prima metà dell'ottocento. Solo agli inizi del novecento si formulò un modello teorico che rendesse conto dell'esperienza; in particolare le leggi di Ohm sono del 1827 mentre il modello teorico di Lorentz-Drude è del 1906. Questo modello parte da una considerazioni e due principi:
\begin{itemize}
	\item[C] Gli elettroni si muovono per agitazione termica (in modo isotropo) molto maggiore della velocità di deriva (quindi il tempo che intercorre fra due urti non dipende dalla velocità di deriva)
	\item[P1] Essendo il tempo fra due urti molto piccolo i tratti percorsi dagli elettroni fra due urti sono approssimabili a rettilinei (moto uniformemente accelerato)
	\item[P2] Dopo un urto la distribuzione di velocità è isotropa
\end{itemize}
I materiali che soddisfano questi due principi sono detti ohmici.\\
Sia \(v_{i+1}\) la velocità subito prima di un urto, \(v_i\) quella subito dopo e $\overline{t}$ il tempo medio fra due urti. Essendo il moto uniformemente accelerato, la velocità prima dell'urto è uguale alla velocità iniziale (isotropa, dovuta all'agitazione termica) più l'accelerazione dovuta al campo per il tempo medio fra due urti
\[\mathbf{v}_{i+1}  = \mathbf{v}_i - \frac{e\mathbf{E}}{m_e}\overline{t}\]
\[\mathbf{v}_d = \sum_i\frac{\mathbf{v}_{i+1}}{N}= \frac{\sum_i\mathbf{v}_i}{N} - \frac{e\mathbf{E}}{m_e}\overline{t} \frac{N}{N} = - \frac{e\mathbf{E}}{m_e}\overline{t}\]
dove il primo addendo si elimina perché la media vettoriale di una velocità isotropa è nulla. Possiamo riscrivere il vettore di densità di corrente
\[\mathbf{J}= ne\mathbf{v_d}= \frac{n e^2\overline{t}\mathbf{E}}{m_e}\]
Definiamo la \textbf{conduttività} $\sigma'$ e la \textbf{resistività} $\rho'$
\[\sigma' \equiv \frac{ne^2\overline{t}}{m_e}\]
\[\rho' \equiv \frac{1}{\sigma'}= \frac{m_e}{n e^2\overline{t}}\]
osserviamo che il tempo medio fra due urti dipende dal passo del reticolo cristallino l e dalla velocità di agitazione termica (\(\overline{t}= \frac{l}{v_i}\)), ne segue che conduttività e resistività dipendono solo dal tipo di materiale (in N ed l) e dalla temperatura ( in $\mathbf{v}_i$). \\
In conclusione
\[\mathbf{E}= \rho' \mathbf{J}\]
\[\mathbf{J}= \sigma' \mathbf{E}\]
Queste sono forme equivalenti alle leggi di Ohm, che possiamo ricavare semplicemente.\\
Si consideri un circuito in condizioni stazionarie (i costante) e con campo elettrico costante con differenza di potenziale \(\Delta V_{AB}\) a cui è collegato un cilindro di materiale conduttore di altezza l e superficie di base S. 
\[\Delta V = \int_{A}^{B}\mathbf{E}d\mathbf{l}=El\]
\[i = \int_S \mathbf{J}d\mathbf{s}= JS= \frac{E S}{\rho'}\]
\[\Rightarrow E = \frac{\rho'}{S}i\]
\[\Rightarrow \Delta V = \frac{l\rho'}{S}i \equiv R i\]
dove R è la \textbf{resistenza}. Quest'ultima è la prima legge di Ohm, la seconda è proprio la definizione di resistenza R
\[R \equiv \frac{\rho' l}{S}\]
\begin{definizione}[Resistenza]
Proprietà di un circuito elettrico o di parte di esso che trasforma energia elettrica in calore, opponendosi al passaggio di corrente elettrica. Spesso considerata in un punto localizzato del circuito, è in realtà caratteristica di ogni parte di esso. Viene rappresentata con il simbolo
\begin{figure}[h!]
	\centering
	\includegraphics[width=0.3\linewidth]{../images/resistenza}
	\caption{Simbolo della resistenza elettrica.}
	\label{fig:resistenza}
\end{figure}
\FloatBarrier
\end{definizione}
\subsubsection*{Resistenze in serie e in parallelo}
Consideriamo due resistenze uguali collegate in serie come in figura
\begin{figure}[h!]
	\centering
	\includegraphics[width=0.6\linewidth]{../images/resistenze_serie}
	\caption{Resistenze in serie}
	\label{fig:resistenzeserie}
\end{figure}
\FloatBarrier
Intuitivamente, ci aspettiamo che la resistenza totale aumenti a causa dell'aumentare della lunghezza totale della resistenza. Considerando il circuito diviso dalle due resistenze nelle parti A, B, C, possiamo esprimere la differenza di potenziale in due modi differenti e poi uguagliare
\[\Delta V = V_a - V_c = R_{tot}i\]
\[\Delta V = V_a -V_b + V_b- V_c = (R_1 + R_2)i\]
\[\Rightarrow R_{tot} = R_1 + R_2\]
che può essere esteso al caso di n resistenze in serie e al caso continuo come segue
\[R_{tot} = \sum_{i=1}^N R_i \]
\[R_{tot} = \int dR\]
Ne deduciamo che, avendo una resistenza di forma irregolare, possiamo considerare ogni sezione trasversale come una resistenza collegata in serie con le altre ed integrare come segue
\[R_{tot}= \int_{a}^{b}\frac{\rho'}{S(l)}dl\]
Nelle resistenze in parallelo invece, intuitivamente, aumenta la superficie attraverso cui può passare la corrente e quindi ci aspettiamo una diminuzione della corrente. Consideriamo due resistenze in parallelo come in figura
\begin{figure}[h!]
	\centering
	\includegraphics[width=0.5\linewidth]{../images/resistenze_parallelo}
	\caption{Schema di due resistenze in parallelo}
	\label{fig:resistenzeparallelo}
\end{figure}
\FloatBarrier
\[\Delta V = R_1i_1= R_2i_2\]
\[i = i_1+i_2\]
\[i_1 = \frac{\Delta V}{R_1}\quad i_2 = \frac{\Delta V}{R_2} \]
\[ i = \Delta V \left(\frac{1}{R_1}+\frac{1}{R_2}\right)= \frac{\Delta V}{R_{tot}} \]
\[\Rightarrow \frac{1}{R_{tot}} = \frac{1}{R_1}+\frac{1}{R_2} \]
Che si estende a 
\[\frac{1}{R_{tot}}= \sum_{i = 1}^N\frac{1}{R_i}\]

\subsubsection*{Resistenze dal punto di vista energetico: effetto Joule}
In presenza di campo elettrico gli elettroni hanno una velocità di deriva, se spengo il campo tornano ad avere velocità media nulla, qual'è la potenza che devo erogare per mantenere costante questa velocità (e quindi sconfiggere la resistenza)? Cominciamo considerando quella necessaria ad un singolo portatore di carica
\[P_i = \frac{dL}{dt}= \frac{\mathbf{F}\mathbf{dr}}{dt}= \mathbf{F}\mathbf{v_d}= -q\mathbf{E}\mathbf{v_d}\] 
moltiplicando per la densità di portatori di carica otteniamo la densità volumetrica di potenza necessaria, osservando che la carica di un portatore coincide con quella dell'elettrone otteniamo
\begin{align}\label{eq:effetto_Joule}
	\frac{P}{V}= nP_i=  (-ne\mathbf{v_d})\mathbf{E}= \mathbf{J}\mathbf{E}= \sigma' E^2= \rho' J^2
\end{align}
quest'ultima è una versione della \textbf{legge di Joule}, essa esprime la quantità di energia dissipata per unità di tempo e volume in un conduttore. Per ottenere la forma più conosciuta ed utile nelle applicazioni, consideriamo un conduttore cilindrico di sezione S e lunghezza l
\[P = (Sl) \rho' J^2 \frac{S}{S} = \frac{\rho' l}{S}(J^2 S^2)= Ri^2= \Delta V i \]
Microscopicamente, l'energia viene dissipata a causa degli urti dei portatori di carica che generano calore, ovvero energia che viene dissipata; macroscopicamente osserviamo che il conduttore si riscalda a causa del passaggio di corrente. Si osservi che all'aumentare della temperatura la resistenza aumenta, ne segue che aumenta la potenza dissipata e quindi la temperatura, ricomincia così il circolo vizioso che potrebbe portare a temperature altissime; ciò spiega perché è fondamentale la refrigerazione di impianti e dispositivi elettrici come i computer (che rischierebbero di fondere).    
\subsection{Generatori}
Tutti i circuiti visti fin ora presuppongono l'esistenza di un generatore che mantenga costante la \(\Delta V\) fra i capi del circuito (sopperendo quindi alla perdita di energia dovuta all'effetto Joule). Consideriamo la differenza di potenziale tra i capi di un filo conduttore, mantenuta costante da un generatore: secondo la prima legge di Ohm abbiamo
\[\Delta V = \int_a^b \mathbf{E_{el}}\mathbf{dr}= Ri\]
Sappiamo che la circuitazione del campo elettrostatico (\(\mathbf{E}_{el}\)) è nulla, tuttavia in un circuito in cui si considerano portatori di carica in movimento si deve tener conto anche delle resistenze, che si oppongono al moto di queste. Per permettere il moto delle cariche quindi, all'interno del circuito (da A a B), la differenza di potenziale non può essere nulla perché deve fornire abbastanza energia da opporsi alle resistenze; ne segue che deve esistere un'altro campo elettrico, diverso da \(\mathbf{E}_{el}\), non elettrostatico, la cui circuitazione non è nulla. La restante parte di circuito da considerare per ottenere la circuitazione totale è quello che va da B ad A, ovvero la parte di circuito interna al generatore. Sia \(\mathbf{E^*}\) il campo prodotto dal generatore, al suo interno è presente in totale il campo \(\mathbf{E^*}+\mathbf{E_{el}}\) mentre nel resto del circuito c'è solo \(\mathbf{E_{el}}\). Ne segue che il campo \(\mathbf{E^*}\) deve non solo essere opposto a quello elettrico ma anche maggiore di esso e non conservativo.\\
Consideriamo la f.e.m. di un circuito siffatto è 
\[\oint_{\Gamma}\mathbf{E_{tot}}\mathbf{dl}= \int_{a}^{b}\mathbf{E_{el}}\mathbf{dl}+\int_{b}^{a}(\mathbf{E^*}+\mathbf{E_{el}})\mathbf{dl}= Ri + R_{gen}i\]
dove la prima resistenza è quella offerta dal conduttore mentre la seconda quella offerta dal generatore. Possiamo vedere lo stesso calcolo in modo diverso come
\[\oint_{\Gamma}\mathbf{E_{tot}}\mathbf{dl}= \int_{a}^{b}\mathbf{E_{el}}\mathbf{dl} + \int_{b}^{a}\mathbf{E^*}\mathbf{dl} +  \int_{b}^{a}\mathbf{E_{el}}\mathbf{dl}= \int_{b}^{a}\mathbf{E^*}\mathbf{dl}\]
\[\Rightarrow \mathcal{E} = \oint_{\Gamma}\mathbf{E_{tot}}\mathbf{dl} = \int_{b}^{a}\mathbf{E^*}\mathbf{dl} = (R + R_{gen})i \]
ne segue che possiamo trattare un generatore come un pezzo di circuito che offre una sua resistenza \(R_{gen}\), che si può scorporare dal generatore e considerare concentrata in un punto del circuito adiacente al generatore; un generatore si dice ideale se la sua resistenza interna è trascurabile. \(\mathbf{E^*}\) è detto \textbf{campo elementare} e non è conservativo, infatti è proprio l'unica componente di circuitazione non nulla. Possiamo ricavare le seguenti relazioni
\begin{align}\label{eq:legge_ohm_2}
	&i = \frac{\mathcal{E}}{R+R_{gen}}\nonumber \\
	&\Delta V = \mathcal{E}-R_{gen}i
\end{align}
Ne deduciamo che la differenza di potenziale che effettivamente è prodotta da un generatore è minore della f.e.m. indotta da questo. Nelle comuni pile è indicata la f.e.m. e non la $\Delta V$.\\
Da un punto di vista energetico, la potenza totale dissipata da un circuito con generatore
\[P = \mathcal{E} i = R i^2 + R_{gen}i^2\]
mentre il lavoro che richiede il passaggio di corrente stazionaria i in un tempo dt è 
\[dL = P dt =\mathcal{E} i dt= \mathcal{E}\frac{dq}{dt}dt=\mathcal{E} dq \]
\[\mathcal{E} = \frac{dL}{dq}\]
La potenza non è quindi dissipata solo dall'effetto Joule del conduttore ma anche da quello del generatore. 
\subsection{Leggi di Kirchhoff e di Ohm generalizzata}
A partire da quanto ottenuto possiamo dedurre leggi generali per circuiti che presentano sia resistenze che generatori. Le tre seguenti leggi sono utili per la risoluzione di esercizi e problemi pratici riguardanti circuiti in regime stazionario.\\
A partire dalla (\ref{eq:legge_ohm_2})  possiamo formulare un'espressione generale, detta \textbf{legge di Ohm generalizzata}, per un numero qualsiasi di generatori e resistenze
\[(V_A-V_B)+\sum_{i} \mathcal{E}_i = R_{AB}i \]
dove la somma delle f.e.m. è algebrica (a seconda di come sono orientati).\\
Le leggi di Kirchhoff offrono informazioni sui circuiti in regime stazionario, la prima è equivalente alla conservazione della carica ma applicata ad un nodo (un punto in cui il circuito si dirama in più parti)
\begin{align*}
	\sum_i i_i = 0
\end{align*}
la seconda è equivalente al fatto che la differenza di potenziale su un circuito chiuso (maglia) è nulla, ottenibile a partire dalla legge di Ohm generalizzata
\begin{align*}
	\sum_i \mathcal{E}_i -\sum_i R i_i =0
\end{align*}
\newpage
\section{Circuiti in regime quasi stazionario:\\ il circuito RC}
Fin ora, parlando di resistenze, abbiamo considerato solamente circuiti in cui la corrente è costante in tutti i punti; l'unico caso diverso che considereremo è il regime quasistazionario.
\begin{definizione}[Regime quasistazionario]
	Circuito in cui è presente una corrente variabile nel tempo, ma che scorre in strutture di dimensioni limitate, tale che l'intensità possa essere considerata costante a ogni particolare istante
\end{definizione}
Consideriamo un circuito con un condensatore, un generatore, un interruttore ed una resistenza in serie (detto circuito RC), con l'interruttore aperto non vi è corrente ed il condensatore è scarico; quando si chiude il circuito, dopo un tempo sufficientemente lungo, si raggiunge una situazione di equilibrio in cui di nuovo non scorre corrente (a causa della presenza del condensatore che lascia il circuito aperto) e le armature del condensatore sono cariche. Vogliamo sapere come evolva il sistema in questo periodo di transizione tra uno stato e l'altro (in particolare le espressioni di q(t) sulle armature del condensatore e i(t) che scorre nel circuito). 
\begin{figure}[h!]
	\centering
	\includegraphics[width=0.5\linewidth]{../images/circuito-rc}
	\caption{Schema del circuito RC.}
	\label{fig:circuito-rc}
\end{figure}
\FloatBarrier
Per ottenere q(t) impostiamo una equazione differenziale a partire dall'espressione della f.e.m.. Considerando la resistenza interna al generatore trascurabile, otteniamo da una relazione vista in precedenza 
\[\mathcal{E} = \Delta V = \Delta V_{ab} + \Delta V_{bd} = \frac{q(t)}{C}+R\frac{dq(t)}{dt}\]
Otteniamo quindi l'equazione differenziale del primo grado, risolvibile con il metodo di separazione delle variabili
\[\Delta V C-q = RC \dot{q}\]
\[\frac{dq}{q-C\Delta V}= -\frac{dt}{RC}\]
\[q(t) = -e^{-\frac{t}{RC}}C\Delta V+C\Delta V = C\Delta V(1-e^{-\frac{t}{RC}})\]
\[i(t) = \frac{dq}{dt} = \frac{\Delta V}{R}e^{-\frac{t}{RC}}\]
\[\Delta V_{ab}(t)=\frac{q}{C}= \Delta V(1-e^{-\frac{t}{RC}})\]
\[\Delta V_{bd}(t)=Ri(t)= \Delta Ve^{-\frac{t}{RC}}\]
Osserviamo come la funzione q(t) tenda asintoticamente al valore \(Q=C\Delta V \), ciò vuol dire che il condensatore è sempre in carica ma, essendo la funzione esponenziale, già dopo pochissimo sarà carico quasi del tutto e la corrente che passa nel circuito per completare la carica è minima. Possiamo avere un'idea del tempo che impiega un condensatore a caricarsi alla percentuale p come segue
\[\frac{Q-q(t)}{Q} = e^{-\frac{t}{RC}}=p\]
L'unica incognita è il tempo; per una carica del 99.9\%,99\%,95\% deve passare rispettivamente un tempo di 7RC, 4.6RC,3RC. Notiamo che l'unità di misura di RC è proprio un tempo, che è caratteristico dello specifico circuito, viene infatti detto \textbf{tempo proprio del circuito}. Un valore tipico di tempo proprio è di \(1\mu s \), per caricare al 99.9\% un condensatore servono dunque \(7\cdot10^{-6}\ s\). \\
Dal punto di vista energetico, una parte di energia per caricare il condensatore è dissipata durante il processo di carica per effetto Joule. Calcoliamo l'energia che il generatore immette nel sistema per caricare completamente il condensatore integrando l'espressione della potenza erogata in un tempo infinito (abbiamo visto che la carica è un processo infinito). 
\[L_{gen} = \int_{0}^{+\infty}P_{gen}(t)dt=\int_{0}^{+\infty}\Delta V i(t)dt = \int_{0}^{+\infty}\frac{\Delta V^2}{R} e^{-\frac{t}{RC}}dt = C\Delta V^2 \]
Ne deduciamo che L'energia fornita dal generatore per caricare il condensatore è doppia rispetto a quella immagazzinata da questo una volta carico (\(U=\frac{1}{2}X\Delta V^2\)), l'altra metà è persa per effetto Joule. Possiamo pervenire allo stesso risultato integrando la potenza dissipata per effetto Joule in un tempo infinito per ottenere l'energia dissipata nel processo di carica. 
\[L_{R} = \int_{0}^{+\infty}P_{R}(t)dt =\int_{0}^{+\infty}Ri(t)^2dt = \int_{0}^{+\infty}\frac{\Delta V^2}{R}e^{-\frac{2t}{RC}}dt = \frac{1}{2}C\Delta^2\]
\newpage
\section{Il magnetismo stazionario nel vuoto}
\subsection{Evidenze sperimentali}
A partire dalla prima metà dell'Ottocento, grazie all'avanzamento della tecnica, studiosi quali Coulomb (1736-1806), Ampere (1775-1836), \O{}rsted (1777-1851) Faraday (1791-1867), Lorentz (1853-1928)
Coulomb sperimentalmente pervenne ad una legge simile a quella ottenuta per la forza generata da un campo elettrico, ipotizzando così l'esistenza di "cariche magnetiche" positive e negative simili a quelle elettriche, spinto anche dall'analogia nella forma delle tre forze naturali che sembravano poter descrivere tutti i fenomeni conosciuti (gravità, elettricità e magnetismo). 
\[\mathbf{F}=k\frac{q_{m1}q_{m_2}}{r^2}\mathbf{\hat{r}}\]
Alla forza magnetica, in analogia con quanto visto per quella elettrica, possiamo associare un campo magnetico, indicato dalla lettera $\mathbf{B}$; la sua unità di misura è il Tesla (\(T\)) nel S.I. e il Gauss nel C.G.S. Il Tesla è un' unità di misura molto grande e si usano spesso suoi sottomultipli
\[1G=10^{-4}T\]
Il problema che rende l'approccio di Coulomb sostanzialmente sbagliato è che si verifica sperimentalmente che è impossibile isolare cariche magnetiche positive e negative; avendo un magnete superiormente carico negativamente e inferiormente positivamente, una volta spezzato in due parti produce due nuovi magneti con le stesse caratteristiche.
\begin{figure}[h!]
	\centering
	\includegraphics[width=0.5\linewidth]{../images/magnete_rotto}
	\caption{Si osserva l'impossibilità sperimentale di ottenere "monopoli magnetici" per scissione di un magnete.}
	\label{fig:magneterotto}
\end{figure}
\FloatBarrier
\'{E} interessante che in tempi recenti sono stati condotti esperimenti (basati su teorie moderne) che ricercavano l'esistenza di monopoli magnetici, finora mai osservati.\\
Ciò rende i fenomeni magnetici diversi da quelli elettrici perché le linee di forza di un campo magnetico sono sempre necessariamente chiuse, perché partendo da un polo, rientrano sempre al polo di carica opposta. Ne segue che il campo magnetico ha sempre flusso nullo
\[\mathbf{\Phi}(\mathbf{B}) = 0\] 
In seguito alcuni esperimenti misero in luce le strette relazioni tra campo magnetico ed elettrico. L'esperimento di \O{}rsted mette in luce il fatto che un filo attraversato da corrente fa ruotare un'ago magnetico (dunque produce campo magnetico) orientandolo perpendicolarmente alla direzione del flusso di corrente elettrica; deduce inoltre che il campo prodotto dal filo è circolare, ortogonale ad esso e di verso che segue la regola della mano destra. Ampere osserva che anche tra due fili attraversati da corrente si genera una forza (che ipotizza di tipo magnetico) attrattiva se direzione e verso della corrente sono paralleli e concordi, repulsiva se paralleli e discordi. Infine Faraday osserva che un filo attraversato da corrente posto ortogonalmente fra due poli di un magnete sperimenta una forza ortogonale sia al filo che all'asse congiungente i due poli.\\
\subsection{Deduzioni dai fatti sperimentali}
Da queste leggi deduciamo che la Forza generata dal campo magnetico deriva da un prodotto scalare (risulta ortogonale ai due agenti che sembrano provocarla: il campo magnetico e la corrente elettrica), l'ordine dei fattori si deduce dall'orientamento della forza risultante e si osserva, seguendo il principio della mano destra, che il primo è legato alla direzione della corrente e il secondo dal campi magnetico. Tuttavia bisogna notare che per come è stata definita la corrente è uno scalare, è dunque errato parlare di "direzione della corrente", in realtà la direzione è quella del vettore velocità degli elettroni che attraversano il filo, la corrente è in relazione diretta sia con la velocità degli elettroni sia con la quantità di questi (e quindi con la carica in generale). Da queste considerazioni Lorentz formulò l'omonima legge
\[\mathbf{F}=q\mathbf{v}\wedge\mathbf{B}\]
Notiamo subito un'altra differenza con il campo elettrico: la forza è ortogonale alle linee di campo quindi non è indifferente parlare di "linee di campo" o "linee di forza", inoltre questa forza esiste solo se le cariche sono in movimento (il magnetismo infatti non è osservabile nell'ambito dell'elettrostatica). Inoltre, calcolando il lavoro
\[L = \int_a^b \mathbf{F}\mathbf{dl}=\int_a^b (q\mathbf{v}\wedge\mathbf{B})\cdot \mathbf{dl}= \int_a^b (q\mathbf{v}\wedge\mathbf{B})\cdot \mathbf{v}dt= 0 \]
poiché il prodotto vettoriale risulta in un vettore ortogonale ad entrambe i fattori, ed il prodotto scalare fra vettori ortogonali è nullo. Ne segue che il campo magnetico non fa mai lavoro.\\
Per il teorema delle forze vive 
\[L = \frac{1}{2}m(v_f^2-v_i^2) = 0 \]
\[\Rightarrow v_f = v_i\]
Si presti attenzione al fatto che ciò non implica \(\mathbf{v_f} = \mathbf{v_i}\) poiché il campo magnetico non può cambiare il modulo della velocità di un corpo ma può cambiare la sua direzione.\\
Consideriamo una carica di velocità costante $\mathbf{v}$ ortogonale ad un campo magnetico $\mathbf{B}$, il modulo della Forza di Lorentz è
\[F = qvB\]
perpendicolare alla traiettoria (quindi centripeta) e costante. Eguagliando questa espressione a quella della forza centripeta otteniamo
\[qvB=\frac{mv^2}{r}\ \Rightarrow\ r=\frac{mv}{qB}\]
poiché r, il raggio di curvatura, è costante, il moto che ne risulta yz (considerando la velocità della carica in direzione x) è di tipo circolare uniforme. Altre grandezze di interesse possono essere facilmente ricavate
\[\omega=\frac{qB}{m}\]
\[T=\frac{2\pi m }{qB}\]
Se invece la velocità della carica non è ortogonale a $\mathbf{B}$, scomponiamo la velocità nelle componenti perpendicolare e parallela a B \(\mathbf{v}=\mathbf{v}_{\perp}+\mathbf{v_{//}}\)
\[\mathbf{F} = q(\mathbf{v}_{\perp}+\mathbf{v_{//}})\wedge \mathbf{B}=q\mathbf{v}_{\perp}\wedge \mathbf{B}+q\mathbf{v_{//}}\wedge \mathbf{B}= q\mathbf{v}_{\perp}\wedge \mathbf{B} \] 
dove il termine di velocità parallela al campo magnetico si annulla. Ricordando l'espressione generale della forza centripeta in funzione di $\omega$ ed $\mathbf{r}$
\[\mathbf{F}= q\mathbf{v}_{\perp}\wedge \mathbf{B}= m \omega\wedge(\omega\wedge\mathbf{r})= m\omega\wedge\mathbf{v}= -m\mathbf{v}\wedge\omega \] 
\[\mathbf{v_{\perp}}\wedge \mathbf{B}= -\frac{m}{q}\mathbf{v}\wedge\omega =\mathbf{v}\wedge\left(-\frac{m}{q}\omega\right) \]
\[\Rightarrow \omega= -\frac{q}{m}\mathbf{B}\]
Ma che moto ha la carica nello spazio? La forza sull'asse x è nulla dunque in questa direzione il moto è rettilineo uniforme con velocità pari a quella costante iniziale. Sul piano yz invece c'è una forza centripeta che produce un moto circolare uniforme.
\[\mathbf{F}= q\mathbf{v}\wedge\mathbf{B}= qvB\sin\theta\]
La composizione di questi due moti è detta \textbf{moto elicoidale}
\begin{figure}[h!]
	\centering
	\includegraphics[width=0.5\linewidth]{../images/moto_elicoidale}
	\caption{Rappresentazione del moto elicoidale}
	\label{fig:motoelicoidale}
\end{figure}
\FloatBarrier
Nel caso più generale di una carica sottoposta sia a campo elettrico costante che magnetico abbiamo
\[\mathbf{F}= q\mathbf{E}+q\mathbf{v}\wedge\mathbf{B}\]
Fin ora abbiamo considerato la forza di Lorentz su una carica, in un conduttore le cariche si muovono per agitazione termica, che ha somma vettoriale nulla, se vi è applicato un campo magnetico avremo una velocità di deriva risultante che genera anche campo magnetico. La forza di Lorentz in una porzione di filo conduttore di sezione $\Sigma$ e lunghezza dl è
\[\mathbf{dF}=n(\Sigma dl)\mathbf{F_i} = n\Sigma dl q\mathbf{v}\wedge\mathbf{B}= -n\Sigma dl e\mathbf{v}\wedge\mathbf{B} = \Sigma dl \mathbf{J}\wedge\mathbf{B}= i\mathbf{ds}\wedge\mathbf{B}\]  
\[\Rightarrow \mathbf{F}=i\mathbf{l}\wedge\mathbf{B}\]
dove n è la densità dei portatori di carica e \(\mathbf{F_i}\) la forza di Lorentz generata da una singola carica, q è stato sostituito con -e, la carica dell'elettrone, in quanto in un conduttore metallico i portatori di carica sono gli elettroni liberi. Si dimostra che la forza non dipende dal cammino del filo e che $\mathbf{l}$ è in realtà la distanza tra due punti del filo.  
\subsection{Alcune applicazioni}
\begin{applicazione}[Spettrometro di massa]
Basandosi sul fatto che il campo magnetico è legato alla velocità della carica è possibile creare un apparato che selezioni particelle che viaggiano ad una specifica velocità, basandosi poi sul fatto che il campo magnetico genera una forza centripeta che curva il moto delle cariche in base non solo loro velocità ma anche a massa e carica di queste, è possibile ottenere informazioni su queste grandezze se i campi magnetici hanno intensità nota.\\
Inizialmente abbiamo un campione di particelle già cariche o previamente ionizzate, di velocità differenti (ad esempio secondo una distribuzione di Maxwell-Boltzmann), innanzitutto bisogna selezionare le particelle di una specifica velocità con l'utilizzo di un \textbf{selettore di velocità}, costituito da un condensatore all'interno del quale è presente un campo magnetico ortogonale alle facce in modo da produrre una forza di Lorentz  parallela e discorde a quella generata dal campo elettrico. Una carica, supponiamo un elettrone, che passa per questo apparato è sottoposta alle forze (che consideriamo in modulo essendo tutte parallele e verticali)
\[F_{tot} = e\mathbf{E}-(evB_1)\]
Se la forza totale non è nulla gli elettroni saranno deflessi, gli unici a proseguire dritto saranno quelli di velocità \(v_0\) tale che
\[F_{tot}= e\mathbf{E}-(ev_0B_1)=0\]
\[\Rightarrow v_0=\frac{E}{B_1}\]
\begin{figure}[h!]
	\centering
	\includegraphics[width=0.5\linewidth]{../images/selettore_velocità}
	\caption{Schema di un selettore di velocità. Il funzionamento è indipendente dal segno della carica del portatore.}
	\label{fig:selettorevelocita}
\end{figure}
\FloatBarrier
Una volta selezionata la velocità vogliamo avere informazioni su carica e massa di queste particelle. Ci avvaliamo di uno strumento come quello in figura (il selettore di velocità non è rappresentato).
\begin{figure}[h!]
	\centering
	\includegraphics[width=0.7\linewidth]{../images/spettrometro}
	\caption{Schema di uno spettrometro di massa}
	\label{fig:spettrometro}
\end{figure}
\FloatBarrier
Innanzitutto la velocità iniziale dovrebbe essere aumentata di un valore noto (mediante campi elettrici) per amplificare le grandezze da misurare, non teniamo conto di questo accorgimento di facile implementazione. In seguito si applica un campo magnetico $\mathbf{B_2}$, la forza di Lorentz, centripeta, vale
\[F = evB_2= \frac{mv_0^2}{R}\]
\[eB_2 = \frac{mv_0}{R}= \frac{mE}{RB_1} \]
\[\Rightarrow \frac{e}{m}= \frac{E}{RB_1B_2}\]
In base al raggio di curvatura R, misurabile mediante un detector posto all'estremità dell'apparato, essendo i campi applicati tutti noti, è possibile ricavare il rapporto carica/massa.
\end{applicazione}
\begin{applicazione}[Effetto Hall]
	Consideriamo un parallelepipedo conduttore di spessore d, larghezza l e lunghezza h, attraversato da un flusso densità di corrente $\mathbf{J}$ e da un campo magnetico \(\mathbf{B}\) ad esso ortogonale
\begin{figure}[h!]
	\centering
	\includegraphics[width=0.6\linewidth]{../images/hall_effect}
	\caption{Schema dell'effetto Hall.}
	\label{fig:halleffect}
\end{figure}
\FloatBarrier
Inizialmente il conduttore è neutro, le cariche, di velocità di deriva $\mathbf{v_d}$ sono sottoposte ad una forza di Lorentz che, considerandole di carica negativa, le spinge verso la faccia sinistra (si veda la figura)
\[\mathbf{F}= q\mathbf{v_d}\wedge\mathbf{B}\] 
Possiamo definire il campo di Hall come
\[\mathbf{E_H}=\frac{\mathbf{F}}{q}=\mathbf{v_d}\wedge\mathbf{B}=\frac{\mathbf{J}}{nq}\wedge\mathbf{B}\]
Le cariche negative col tempo si addensano sulla faccia sinistra generando una differenza di potenziale fra le facce e quindi un campo elettrico che si oppone al campo di Hall. Dopo un intervallo di tempo sufficientemente lungo i due campi si annullano e si raggiunge l'equilibrio. Possiamo definire la forza elettromotrice di Hall come
\[\Delta V_H = E_H l =\frac{BJ}{nq}l =\frac{iB}{ldnq}l=\frac{i B}{nqd} = R_H \frac{iB}{d}\] 
dove è stata usata la formula della \(\Delta V\) per i condensatori, ed \(R_H\) è la \textbf{costante di Hall}.
\end{applicazione}
\subsection{Misurare il campo magnetico}
Come già osservato l'ideazione di un'apparato sperimentale per la misura di una grandezza offre la possibilità di dedurne informazioni teoriche. Nel caso del campo magnetico il primo semplice metodo di misura del campo magnetico è basato sullo studio del moto di una spira immersa in un campo magnetico
\begin{definizione}[Spira]
	 In elettrotecnica la spira di un avvolgimento solenoidale è il tratto di conduttore corrispondente a un passo dell’elica di avvolgimento; poiché tale tratto è proporzionale al campo magnetico generato dall'intero avvolgimento percorso da corrente, il termine ha finito con l’essere usato, in elettromagnetismo, per indicare un circuito chiuso, per es. un circuito filiforme circolare (spira circolare) o poligonale (spira rettangolare, quadrata, ecc.).
\end{definizione}
Consideriamo una spira rettangolare di lati a,b attraversata da corrente in senso antiorario. Il circuito individua una superficie S di vettore normale $\mathbf{\hat{u}_n}$ (il verso si determina con la regola della mano destra). Si consideri un campo magnetico $\mathbf{B}$ passante per S che forma un angolo $\theta$ con $\mathbf{\hat{u}_n}$. Consideriamo per semplicità il vettore $\mathbf{B}$ giacente nel piano parallelo ai lati superiore ed inferiore del rettangolo, se giacesse parallelamente all'altra coppia di lati i calcoli seguenti sarebbero analoghi ma con gli assi invertiti (l'asse di rotazione sarebbe quello orizzontale al posto di quello verticale), se invece giacesse su una qualsiasi direzione il problema si complicherebbe (avremmo una rotazione del solenoide su entrambi gli assi).\\
Le forze che agiscono sui lati del solenoide sono a due a due uguali,
\begin{figure}[h!]
	\centering
	\includegraphics[width=0.5\linewidth]{../images/spira_campo_magnetico}
	\caption{Spira immersa in un campo magnetico, le forze sui lati superiore ed inferiore si annullano a vicenda, quelle sui lati destro e sinistro generano momento.}
	\label{fig:spiracampomagnetico}
\end{figure}
\FloatBarrier
La forza di Lorentz totale sulla spira è
\[\mathbf{F}= \int_{\Gamma}dq\mathbf{v}\wedge\mathbf{B}= \int_{\Gamma}dq\frac{\mathbf{dl}}{dt}\wedge\mathbf{B}= i \int_{\Gamma}\mathbf{dl}\wedge\mathbf{B}\]
Possiamo spezzare questo integrale di percorso nei 4 lati, chiamiamo 1 e 3 quelli superiore e inferiore di lunghezza a, 2 e 4 quelli destro e sinistro di lunghezza b.\\
Calcoliamo le forze e il momento da esse generato: l'angolo compreso fra il vettore spostamento $\mathbf{dl}$ ed il campo magnetico sui lati 1 e 3 è pari a \(\frac{\pi}{2}+\theta\), per i lati 2 e 4 è invece \(\frac{3\pi}{2}\), indipendentemente da $\theta$ (osservare attentamente la figura per convincersene)
\[F_1=F_3=ibB\sin(\frac{\pi}{2}+\theta)\]
\[F_2=F_4 = ibB\sin(\frac{3}{2}\pi)= -ibB\]
Per il calciolo del momento delle forze prendiamo come polo il centro del rettangolo, il braccio per le forze 1 e 3 ha modulo b/2 ed è parallelo ai lati 2 e 4 mentre per le forze 2 e 4 vale a/2 ed è parallelo ai lati 1 e 3.\\
Indipendentemente dal valore delle forze  \(F_1\) ed \(F_3\), il momento da esse generato è nullo perché l'angolo fra braccio e forza è nullo, essendo inoltre di modulo uguali ma di direzione opposte esse non imprimono alla spira neanche un'accelerazione lineare.
Il momento delle forze sulla spira dato da \(F_2\) ed \(F_4\), osservando che l'angolo formato con il braccio è in entrambi i casi $\theta$, è
\[\tau= \mathbf{r}\wedge\mathbf{F_2}+\mathbf{r}\wedge\mathbf{F_4}=2(-\frac{ab}{2}i\sin\theta)= -i SB \sin\theta\]
Ricordando l'espressione del momento delle forze in funzione di $\theta$, impostiamo l'equazione differenziale che descrive il moto della spira
\[\tau=I\ddot{\theta}=- iSB\sin\theta\]
Per piccoli valori di $\theta$ possiamo approssimare $\sin\theta \simeq \theta$, ottenendo l'equazione differenziale armonica: spostando la spira, a partire da una posizione di equilibrio, di un piccolo angolo $\theta$, il campo tende a farla tornare alla posizione iniziale instaurando un moto armonico. 
\[\ddot{\theta}+ \frac{iSB}{I}\theta= 0 \]
Osserviamo che l'espressione di $\tau$ è esprimibile come prodotto vettoriale
\[\tau=iS\mathbf{\hat{u}_n}\wedge\mathbf{B}=-iSB\sin\theta\]
Possiamo definire il vettore \textbf{momento magnetico della spira} e sostituire
\[\mathbf{m}\equiv iS\mathbf{\hat{u}_n}\] 
\begin{align}\label{eq:momento_spira}
	\tau=\mathbf{m}\wedge\mathbf{B}
\end{align}
Osserviamo un'analogia fra il momento generato da una spira percorsa da corrente e quello generato da un dipolo elettrico ($\tau=\mathbf{P}\wedge\mathbf{E}$). A partire da queste considerazioni Ampere formula l'omonimo principio di equivalenza: \textit{una spira attraversata da corrente equivale ad un dipolo magnetico e, più in generale una calamita equivale ad un circuito attraversato da corrente}.\\
Risolvendo l'equazione differenziale otteniamo
\[\omega=\sqrt{\frac{mB}{I}}\]
\[T=2\pi\sqrt{\frac{I}{mB}}\]
con semplici misure risulta possibile misurare l'intensità del campo magnetico.\\
Sperimentalmente, anche grazie a questo strumento, si ottiene che il campo magnetico generato da un filo di lunghezza dl attraversato da corrente è
\begin{align}\label{eq:campo_magnetico_filo_infinitesimo}
&\mathbf{dB}=\frac{\mu_0}{4\pi}\frac{i\mathbf{dl}\wedge\mathbf{r}}{r^3}
\end{align}
Quest'ultima è anche detta \textbf{legge di Biot-Savart}.\\
\[\mathbf{dB}=k_m\frac{i\mathbf{dl}\wedge\mathbf{\hat{u}_r}}{r^2}=\]
dove \(k_m\) è una costante che si pone (per ragioni che saranno chiare in seguito)
\[k_m \equiv\frac{\mu_0}{4\pi}\]
dove $\mu_0$ è detta \textbf{permeabilità magnetica del vuoto}.\\
Anche per il campo magnetico vale il principio di sovrapposizione, considerando un oggetto qualsiasi è possibile scomporlo in fili
\[\mathbf{dB}=\frac{\mu_0}{4\pi}\frac{i\mathbf{dl}\wedge\mathbf{\hat{u}_r}}{r^2}=\frac{\mu_0}{4\pi}\frac{(\Sigma dl)\mathbf{J}\wedge\mathbf{\hat{u}_r}}{r^2}=\frac{\mu_0}{4\pi}\frac{\mathbf{J}\wedge\mathbf{\hat{u}_r}dV}{r^2}=\frac{\mu_0}{4\pi}\frac{q\mathbf{v}\wedge\mathbf{\hat{u}_r}n dV}{r^2}\]
Il campo magnetico creato da un solo portatore di carica si ha per \(ndV=1\)
\begin{align*}
&\mathbf{B_i}=\frac{\mu_0}{4\pi}\frac{q\mathbf{v}\wedge\mathbf{\hat{u}_r}}{r^2} 
\end{align*}
Ricordando l'espressione del campo elettrico possiamo sostituire per ottenere
\[\mathbf{B_i}=\mu_0\mathbf{v}\wedge\varepsilon_0\mathbf{E}\]
Risulta che 
\[\mu_0\varepsilon_0=\frac{1}{c^2}\]
\[\mathbf{B_i}=\frac{1}{c^2}\mathbf{v}\wedge\mathbf{E}\]
dove c è la velocità della luce.
\subsection{Circuitazione del campo magnetico: legge di Ampere-Maxwell}
Consideriamo due fili infiniti paralleli a distanza d in cui la densità di corrente fluisce con verso concorde verso l'alto, qual è la forza per unità di lunghezza esercitata dal filo 1 (a sinistra) sul 2 (a destra)? \\
Partendo dall'espressione del campo magnetico generato da un tratto infinitesimo di filo 1 attraversato da corrente (\ref{eq:campo_magnetico_filo_infinitesimo}) su un punto P del filo 2, integriamo per ottenere il campo magnetico totale generato dal filo 1 du un punto P del filo 2. 
\[\mathbf{dB} = \frac{\mu_0}{4\pi}\frac{i_1\mathbf{dl}\wedge\mathbf{r}}{r^3} \]
visto che, per la regola della mano destra, il campo magnetico ha sempre direzione entrante nel foglio, possiamo considerare solo i moduli
\[B=\int_{filo\ 1}dB= \int_{filo\ 1}\frac{\mu_0 i_1 }{4\pi}\frac{dl \sin\theta}{r^2}\]
Integriamo rispetto a $\theta$ che varia da 0 a $\pi$ radianti; esprimiamo r e dl in funzione di $\theta$
\[B = \int_{0}^{\pi}\frac{\mu_0 i_1 }{4\pi}\frac{\left(\frac{d}{\sin^2\theta}\right) \sin\theta}{\left(\frac{d^2}{\sin^2\theta}\right)^2}= \frac{\mu_0 i_1}{2\pi d}\]
La forza che genera il filo 1 su un tratto dl di filo 2 attraversato da corrente \(i_2\) è 
\[dF_{12} = i_2dl_2 B\sin\frac{\pi}{2}=i_2dl_2 \frac{\mu_0 i_1}{2\pi d}\]
La forza per unità di lunghezza è 
\[\frac{dF_{12}}{dl_2}= \frac{\mu_0 i_1i_2}{2\pi d}=-\frac{dF_{21}}{dl_1}\]
Ponendo solo grandezze unitarie ottengo
\[dF = \frac{\mu_0}{2\pi}\]
ovvero la forza generata da un filo infinito su un filo infinitesimo ad esso parallelo se in entrambi fluisce una densità di corrente equiversa, misurabile con un dinamometro. Fino al 2019 questa relazione era usata per definire l'intensità di corrente elettrica da cui poi si ricavava il Coulomb; oggi queste grandezze sono definite a partire dalla carica dell'elettrone in quanto si è preferito legare le definizioni a costanti naturali universali.\\
Da quanto visto osserviamo che il campo magnetico generato da un filo presenta analogie con il campo elettrico
\[B = \frac{\mu_0 i}{2\pi r}\]
\[E = \frac{\lambda}{2\pi \varepsilon_0 r}\]
Entrambi presentano simmetria cilindrica, dipendendo da \(r^{-1}\) ma il campo magnetico è tangente alle circonferenze di livello in ogni punto mentre il campo elettrico è perpendicolare. Osserviamo inoltre che le linee di campo del campo magnetico sono chiuse e il flusso è conseguentemente nullo come osservato in precedenza. 
\begin{figure}[h!]
	\centering
	\includegraphics[width=0.6\linewidth]{../images/esperimento_ampere}
	\caption{Se le direzioni sono concordi la forza è attrattiva, se discordi repulsiva. Il campo magnetico generato dal filo è costituito da circonferenze concentriche al filo di modulo uguale e direzione tangente ad ogni punto della circonferenza.}
	\label{fig:esperimentoampere}
\end{figure}
\FloatBarrier
Risulta ovvio che la circuitazione del campo magnetico è diversa da zero se scegliamo come percorso una linea di campo circolare concentrica ad un filo attraversato da corrente, ci chiediamo se questo sia vero per ogni linea chiusa. Consideriamo un percorso come quello in figura 
\begin{figure}[h!]
	\centering
	\includegraphics[width=0.6\linewidth]{../images/circuitazione_campo_magnetico}
	\caption{Schema utile alla dimostrazione che la circuitazione del campo magnetico è sempre diversa da zero.}
	\label{fig:circuitazionecampomagnetico}
\end{figure}
\FloatBarrier
Spezziamo la circuitazione nei 4 tratti, osservando che nei tratti radiali l'angolo tra il percorso e il campo è retto dunque l'integrale di percorso è nullo
\[\oint_{\Gamma}\mathbf{B}\mathbf{dl}=\int_{D}^{E}\mathbf{B}\mathbf{dl}+\int_{F}^{G}\mathbf{B}\mathbf{dl}\]
Osserviamo che l'angolo fra campo e spostamento è nullo e che il campo è costante sulle circonferenze, possiamo quindi portarne il modulo fori dall'integrale. Dai due integrali risultano gli archi di circonferenza \(\widehat{DE}\) ed \(\widehat{FG}\), per la definizione di angolo in radianti otteniamo
\[\oint_{\Gamma}\mathbf{B}\mathbf{dl}=B_1\int_{D}^{E}dl+B_2\int_{F}^{G}dl= \frac{\mu_0 i}{2\pi r_1}\widehat{DE}+\frac{\mu_0 i}{2\pi r_2}\widehat{FG}=\]
 \[\frac{\mu_0 i}{2\pi r_1}\theta_{DE}r_1+\frac{\mu_0 i}{2\pi r_2}\theta_{FG}r_2=\frac{\mu_0 i}{2\pi}(\theta_{DE}+\theta_{FG})=\mu_0 i \]
Dove nell'ultimo passaggio si è usato il fatto che la somma dei due angoli è l'angolo giro, come risulta evidente in figura.\\
Si dimostra che qualsiasi percorso può essere suddiviso il tratti radiali e tratti ad arco dunque la circuitazione del campo magnetico è sempre pari a \(\mu_0 i\).\\
Questo risultato, valido per ogni configurazione di filo (non per forza rettilinea) è detta \textbf{Legge di Ampere}
\[\oint_{\Gamma}\mathbf{B}\mathbf{dl}=\mu_0 i_{conc}\] 
dove la corrente da considerare è quella \textbf{concatenata}.
\begin{definizione}[Corrente concatenata a $\Gamma$]
	Considerando la linea chiusa orientata $\Gamma$ come bordo di una qualsiasi superficie aperta (ne esistono infinite che hanno $\Gamma$ come bordo), la corrente passante per la superficie è detta corrente concatenata a $\Gamma$.
\end{definizione}
\textit{La circuitazione del campo induzione magnetica B lungo una qualsiasi linea chiusa orientata \(\Gamma\) è pari all'intensità di corrente complessiva con cui la linea chiusa $\Gamma$ si concatena moltiplicata per la permeabilità magnetica del vuoto $\mu_0$}.\\
Possiamo esprimere equivalentemente la legge di Ampere usando il vettore densità di corrente
\[\oint_{\Gamma}\mathbf{B}\mathbf{dl}= \mu_0\int_{\sigma(\Gamma)}\mathbf{J}\mathbf{ds}\]
Dove l'orientamento della superficie $\mathbf{ds}$ dipende dalla scelta del verso di $\mathbf{dl}$, seguendo la regola della mano destra. 
Applicando la legge di Stokes all'espressione precedente otteniamo
\[\mathbf{\n}\wedge\mathbf{B}=\mu_0\mathbf{J}\]
La legge di Ampere non ha validità del tutto generale a causa di un "bias" di cui erano affetti i suoi apparati sperimentali, legato al momento storico: i fili da lui considerati erano attraversati da una corrente continua generata da una pila, unico generatore allora conosciuto, in questo modo il campo elettrico che genera la corrente resta costante nel tempo, rendendo l'esperienza un caso particolare. In questo aneddoto si evidenzia la limitatezza di un approccio puramente sperimentale in confronto ad uno teorico, adottato da Maxwell. Consideriamo l'equazione di continuità vista in precedenza e la forma differenziale della legge di Gauss per il campo elettrico
\[\mathbf{\n}\mathbf{J}=-\frac{\d \rho}{\d t}\]
\[\mathbf{\n}\mathbf{E}=\frac{ \rho}{\varepsilon_0}\]
\[\Rightarrow \mathbf{\n}\mathbf{J}= -\frac{\d (\varepsilon_0\mathbf{\n}\mathbf{E})}{\d t}\]
\[\mathbf{\n}\left(\mathbf{J}+\varepsilon_0\frac{\d\mathbf{E}}{\d t}\right) = 0\]
Possiamo definire un nuovo vettore $\mathbf{J'}$ tale che
\[\mathbf{\n}\mathbf{J'}=0\]
Nel caso di Ampere la divergenza di J è nulla (definizione di regime stazionario \ref{def:corrente_stazionaria}), nel caso di Maxwell invece, dedotto in modo puramente teorico, quella di J' è nulla; abbiamo aggiunto un termine che tiene conto della variabilità nel tempo del campo elettrico. La legge di Ampere, nella forma generale di Ampere-Maxwell, diventa
\[\mathbf{\n}\wedge\mathbf{B}=\mu_0\mathbf{J'}=\mu_0\mathbf{J}+\mu_0\varepsilon_0\frac{\d\mathbf{E}}{\d t}\]
dove notiamo che il termine
\[\varepsilon_0\frac{\d\mathbf{E}}{\d t}\]
ha la stessa dimensione di \(\mathbf{J}\), e viene detto \textbf{vettore densità di corrente di spostamento}. Esprimendo la legge di Ampere-Maxwell in forma integrale (usando Stokes) otteniamo
\[\oint_{\Gamma}\mathbf{B}\mathbf{dl}= \mu_0i + \mu_0\varepsilon_0\frac{\d\Phi(\mathbf{E})}{\d t}\]
il termine
\[\varepsilon_0\frac{\d\Phi(\mathbf{E})}{\d t}\]
ha la stessa dimensione della corrente e viene detta \textbf{corrente di spostamento}.\\
Quando venne pubblicato, il risultato di Maxwell fu ignorato fintanto che non ci si rese conto della sua importanza.\\
Il fatto che la legge di Maxwell abbia validità più generale di quella di Ampere si verifica semplicemente considerando un circuito non stazionario (in cui la corrente è variabile), l'unico caso da noi trattato è quello quasi stazionario del circuito RC. Applichiamo la legge di Ampere in due casi: quello in cui la superficie $\Sigma$ interseca il filo del circuito e quello in cui non lo fa, passando per lo spazio tra le armature del condensatore. Nel primo caso passa corrente per $\Sigma$ secondo l'espressione i(t) ricavata in precedenza
\[\oint_{\Gamma}\mathbf{B}\mathbf{dl}=\mu_0 i(t)= \mu_0 \frac{\mathcal{E}}{R}e^{-\frac{t}{RC}}\neq0\]
Nel secondo caso invece non passa corrente per la superficie poiché fra le armature non passano cariche
\[\oint_{\Gamma}\mathbf{B}\mathbf{dl}=\mu_0 i(t)= 0\]
La legge di Ampere, che dovrebbe essere uguale per ogni superficie avente come bordo $\Gamma$, fornisce risultati diversi, cadendo nell'assurdo: la legge non è valida in questo caso. Sappiamo che il caso sbagliato è il secondo perché abbiamo già osservato come il campo magnetico abbia sempre circuitazione diversa da zero.\\
Ricalcoliamo i due casi con la legge di Ampere-Maxwell
\[\mathbf{\n}\wedge\mathbf{B}=\mu_0\mathbf{J'}=\mu_0\left[\mathbf{J}+\varepsilon_0\frac{\d \mathbf{E}}{\d t}\right]\]
Nel primo caso il campo elettrico non varia nel tempo quindi resta solo il primo addendo della somma. Nel secondo caso non passa corrente quindi il primo addendo è nullo, ma nel caricarsi il condensatore aumenta il campo elettrico compreso fra le armature nel tempo quindi 
\[\mathbf{\n}\wedge\mathbf{B}=\mu_0\varepsilon_0\frac{\d \mathbf{E}}{\d t}\]
Si dimostra che quest'ultimo termine è uguale a \(\mu_0\mathbf{J}\).
Riassumendo, la legge di Maxwell-Ampere rispettivamente in forma differenziale ed integrale è
\begin{align}\label{eq:legge_Maxwell_Ampere}
	&\mathbf{\n}\wedge\mathbf{B}=\mu_0\mathbf{J}+\mu_0\varepsilon_0\frac{\d\mathbf{E}}{\d t}\\
	&\oint\mathbf{B}_{\Gamma}\mathbf{dl}= \mu_0 i + \mu_0\varepsilon_0\frac{d}{dt}\Phi(\mathbf{E})
\end{align}
Resta aperto un problema: in questo modo si spiega l'origine dei fenomeni magnetici legati a correnti elettriche, come si spiega però il campo magnetico generato da una calamita? Questo è dato dal moto degli elettroni nelle molecole di essa che generano una corrente atomica che genera campo magnetico (secondo il primo termine della legge di Ampere-Maxwell). Approfondiremo e giustificheremo le calamite più approfonditamente in seguito.
\subsection{Spire e solenoidi}
Abbiamo già visto la definizione di spira, analogamente a quanto fatto per il campo elettrico calcoliamo il campo magnetico generato da una spira attraversata da corrente in senso antiorario su un punto P giacente sul suo asse a distanza d. 
\begin{esercizio}[Campo magnetico spira]
Consideriamo una porzione infinitesima di spira \(\mathbf{dl}\) attraversata da corrente, per la legge di Biot-Savart il campo generato sul punto P è
\[\mathbf{dB}=\frac{\mu_0 i}{4\pi}\frac{\mathbf{dl}\wedge\mathbf{r}}{r^3}\]
Per determinare la direzione di questo campo usiamo la regola della mano destra: il campo deve essere ortogonale sia al vettore \(\mathbf{r}\) che a \(\mathbf{dl}\), che a loro volta sono ortogonali fra loro, il campo avrà direzione tale da formare un angolo $\alpha$ con l'asse y, dove $\alpha$ è lo stesso angolo compreso fra \(\mathbf{r}\) e l'asse z. Visto che nè angoli compresi nè moduli cambiano spostando \(\mathbf{dl}\) sulla spira, la somma di tutti i contribuiti (principio di sovrapposizione) risulta in un vettore con componente solo sull'asse y. 
\begin{figure}[h!]
	\centering
	\includegraphics[width=0.7\linewidth]{../images/campo_magnetico_spira}
	\caption{Schema del campo generato da una spira attraversata da corrente in senso antiorario su un punto a distanza d sull'asse della spira}
	\label{fig:campomagneticospira}
\end{figure}
\FloatBarrier
\[\mathbf{B}=\int_{spira}\mathbf{B}= B_y\mathbf{\hat{j}}=\int_{spira}dB_y\mathbf{\hat{j}}=\int_{spira}dB\cos\alpha\mathbf{\hat{j}}=\frac{\mu_0 i}{4\pi}\frac{\cos\alpha}{r^2}\int_{0}^{2\pi}dl\mathbf{\hat{j}}=\]
\[\frac{\mu_0 i R}{2r^2}\cos\alpha \mathbf{\hat{j}}=\frac{\mu_0 i R^2}{2r^3} \mathbf{\hat{j}}=\frac{\mu_0 i R^2}{2(R^2+y^2)^{\frac{3}{2}}} \mathbf{\hat{j}}=\frac{\mu_0 (i \pi R^2)}{2\pi(R^2+y^2)^{\frac{3}{2}}} \mathbf{\hat{j}}=\frac{\mu_0 \mathbf{m}}{2\pi(R^2+y^2)^{\frac{3}{2}}}\]
Ipotizzando r>>R, R può essere trascurato al denominatore ottenendo 
\[\mathbf{B}=\frac{\mu_0 \mathbf{m}}{2\pi y^{3}}\]
Si osservi come questa formula sia simile a quella del campo elettrico generato da un dipolo elettrico (\ref{eq:campo_dipolo}). Analogamente al caso del dipolo si mostra che la formula per un punto P in posizione generica è  
\[\mathbf{B}=\frac{\mu_0 m }{4\pi r^3}[2\cos\theta\mathbf{\hat{u}_r}+\sin\theta\mathbf{\hat{u}_{\theta}}]\]
dove $\theta$ è l'angolo compreso tra il vettore \(\mathbf{m}\) della spira e il vettore distanza \(\mathbf{r}\) tra il centro della spira e il punto. Ne segue che le linee di campo hanno la seguente forma
\begin{figure}[h!]
	\centering
	\includegraphics[width=0.7\linewidth]{../images/linee_campo_magnetico_spira}
	\caption{Linee di campo magnetico generate da una spira circolare percorsa da corrente.}
	\label{fig:lineecampomagneticospira}
\end{figure}
\FloatBarrier
Si noti che, come osservato in precedenza, le linee di campo magnetico devono chiudersi, anche quelle che nell'immagine sembrano restare aperte in realtà si chiudono all'infinito. Si faccia attenzione a questo fatto che tornerà utile in seguito.  
\end{esercizio}
Se immergiamo una spira attraversata da corrente in un campo magnetico B esterno, come già osservato nel caso della spira rettangolare,il momento della forza è 
\[\tau = \mathbf{m}\wedge\mathbf{B}_{ext}\]
ma il campo magnetico generato dalla spira \(\mathbf{B}_{spira}\) ha la stessa direzione di \textbf{m}, ne segue che la spira è in quiete quando si allinea il campo magnetico da essa prodotto a quello esterno.\\
Calcoliamo ora il lavoro che della spira immersa in un campo magnetico
\[L = \int M d\theta= \int m B \sin\theta d\theta= -mB\cos\theta + c= -\mathbf{m}\mathbf{B}+c \equiv U\]
dove U è l'energia potenziale derivante dal lavoro fatto dal campo magnetico sulla spira, definita a meno della costante c. Si presenta però un problema concettuale importante: l'energia potenziale è definita solo per campi conservativi, cosa che abbiamo visto il campo magnetico non è, come interpretare fisicamente questo passaggio matematico? La non conservatività di B dipende dal fatto che la sua circuitazione non è mai nulla, che a sua volta è legato al fatto che le linee di campo si chiudono sempre, creando percorsi in cui la circuitazione non è nulla. Abbiamo però osservato che in una spira alcune linee di campo si chiudono all'infinito, ne segue che localmente le linee di campo restano aperte: otteniamo così un campo localmente conservativo per cui ha senso definire un'energia potenziale locale (approssimazione)\\ 
Consideriamo ora N spire circolari collegate fra loro in serie come in figura, questo apparato è detto solenoide.
\begin{figure}[h!]
	\centering
	\includegraphics[width=0.6\linewidth]{../images/Solenoid}
	\caption{Solenoide}
	\label{fig:solenoide}
\end{figure}
\FloatBarrier
Una grandezza spesso usata in relazione ai solenoidi è la densità lineare di spire 
\[n = \frac{N}{l}\]
Ci chiediamo che forma abbia il campo magnetico generato da un solenoide, possiamo seguire due strategie: considerare il solenoide come unione di N spire e sfruttare il principio di sovrapposizione o osservare che se N tendesse ad infinito le linee di campo sarebbero tutte dritte al suo interno a causa della simmetria cilindrica che eliminerebbe le componenti verticali. Seguendo il secondo approccio, possiamo usare la formula di Ampere, poiché è presente solo una corrente e non un campo elettrico variabile nel tempo, prendendo come $\Gamma$ un qualsiasi percorso interno al solenoide
\[\oint_{\Gamma}\mathbf{B}\mathbf{dl}= \mu_0 i\]
dove in questo caso i è la corrente concatenata al solenoide. La corrente passante per la superficie sottesa da $\Gamma$ (che possiamo scegliere interna al solenoide) è nulla (la corrente passa nel circuito che non è interno a $\Gamma$) dunque la circuitazione è sicuramente nulla. 
Ricalcoliamo la circuitazione seguendo la definizione: consideriamo una linea chiusa rettangolare all'interno del solenoide di lati AB, DE paralleli all'asse del solenoide e di lunghezza l e CD,DA perpendicolari all'asse. Avendo osservato che con infinite spire il campo verticale si annulla restano solo due componenti da considerare, in cui il campo è parallelo allo spostamento
\[\oint_{\Gamma}\mathbf{B}\mathbf{dl}= \int_{A}^{B}Bdl+\int_{E}^{D}Bdl=lB_{AB}-lB{CD}=0\]
\[\Rightarrow B_{AB}=B{CD}\]
Ne deduciamo che il campo all'interno del solenoide ha lo stesso valore, direzione e verso in tutti i punti.\\
Per conoscere il modulo del campo all'interno del solenoide consideriamo ora un percorso rettangolare, analogo a quello di prima, ma che sia a cavalo del solenoide come in figura e tale che i due lati verticali siano infinitesimi
\begin{figure}[h!]
	\centering
	\includegraphics[width=0.6\linewidth]{../images/circuitazione_solenoide}
	\caption{Schema di un solenoide sezionato verticalmente e di un percorso a cavallo di esso su cui calcolare la circuitazione}
	\label{fig:circuitazionesolenoide}
\end{figure}
\FloatBarrier
Sul tratto AB sappiamo che la circuitazione vale \(lB_{AB}\), sui tratti verticali la trascuriamo poiché li abbiamo presi per ipotesi infinitesimi. Infine sul lato esterno possiamo considerare il campo magnetico trascurabile rispetto che all'interno perché, come si vede dalla figura, per ogni punto che fornisce un contributo di campo positivo, ne esiste un'altro che ne fornisce uno negativo (usando la regola della mano destra) che annullerà il primo, la somma totale non è nulla perché, solo per i due punti che si trovano sullo stesso asse del punto esterno, i contributi del campo non si annullano perché uno dei due è più vicino dell'altro quindi la differenza (e quindi i moduli non sono uguali), per quanto piccola, non è nulla. 
\begin{figure}[h!]
	\centering
	\includegraphics[width=0.7\linewidth]{../images/campo_esterno_spira}
	\caption{Schema del campo generato da una spira su un punto ad essa esterno, si noti che il campo generato dal vettore in alto a sinistra è uscente dal foglio mentre per quello in basso a destra è entrante.}
	\label{fig:campoesternospira}
\end{figure}
\FloatBarrier
\[\Rightarrow \oint_{\Gamma}\mathbf{B}\mathbf{dl}\simeq lB_{AB}= lB\]
Questa volta la corrente concatenata a $\Gamma$ non è nulla perché questo contiene parte del solenoide dentro il quale passa corrente i in ogni spira; essendoci \(N = n l\) spire all'interno del rettangolo di lato orizzontale l, la corrente concatenata totale è \(inl\). Per la legge di Ampere dunque
\[\oint_{\Gamma}\mathbf{B}\mathbf{dl}\simeq lB= \mu_0 inl\]
\[\Rightarrow B = \mu_0 i n\]
Ricordiamo che questo ragionamento parte dall'approssimazione di una spira infinita, che equivale ad una spira il cui raggio sia molto minore della sua lunghezza (\(r<<l\)), in questo caso le linee di campo della spira si presentano come nell'immagine seguente
\begin{figure}[h!]
	\centering
	\includegraphics[width=0.6\linewidth]{../images/linee_campo_solenoide}
	\caption{Linee di campo generate da un solenoide tale che \(r<<l\)}
	\label{fig:lineecamposolenoide}
\end{figure}
\FloatBarrier
Ricordando l'immagine delle linee di campo prodotte da un magnete vista a inizio sezione notiamo che quelle generate da un tale solenoide sono identiche, per il principio di equivalenza di Ampere infatti una calamita è equivalente ad un solenoide infinito percorso da corrente; il grande vantaggio del solenoide è che è possibile modificare agevolmente il campo magnetico prodotto dal magnete semplicemente variando la corrente. Nei moderni apparati di risonanza magnetica sono presenti grandi avvolgimenti di spire di forma particolare attraverso cui è fatta passare corrente, per ottenere grandi campi magnetici è necessaria la massima quantità di corrente a parità di differenza di potenziale (ovvero resistenze minori possibili); per questo motivo nella produzione di magneti si adottano materiali superconduttori. 
\subsubsection{Continuità del campo magnetico}
Abbiamo osservato come, attraversando una superficie carica, la componente tangente del campo elettrico si conservi mentre quella normale si modifichi. Nel campo magnetico avviene il contrario: la componente normale si conserva perché il flusso è nullo (similmente nel campo elettrico abbiamo dimostrato che la componente normale non si conserva perché il flusso non è nullo) mentre quella tangente non si conserva perché la circuitazione non è nulla (mentre nel campo elettrico abbiamo dimostrato che si conserva perché la circuitazione è nulla). Possiamo valutarne la discontinuità considerando un percorso di altezza infinitesima, come fatto nel caso del solenoide, ed applicare sta volta non la legge di Ampere ma quella più generale di Ampere-Maxwell
\[\oint_{\Gamma}\mathbf{B}\mathbf{dl}=\mu_0 i + \mu_0\varepsilon_0 \frac{\d \Phi(\mathbf{E})}{\d t}\]
\subsection{Mutua induzione e autoinduzione}
Abbiamo già osservato che il flusso del campo magnetico per una superficie chiusa è sempre nullo; vogliamo conoscere il flusso per una superficie aperta. Per farlo consideriamo due superfici aperte, $\Sigma_1$ e $\Sigma_2$, con lo stesso bordo $\gamma$, singolarmente i flussi sono diversi da zero ma se si uniscono le due superfici facendo combaciare i bordi in totale il flusso è nullo perché si forma una superficie chiusa. Cominciamo chiedendoci se il flusso di $\Sigma_1$ sia uguale a quello di $\Sigma_2$.
\[\phi_{12} = \iint_{\Sigma(\Gamma)}d\mathbf{B}_1\mathbf{\hat{u}_n}d\Sigma_2= i\left(\int_{\Sigma(\Gamma)}\oint_{\Gamma_1}\frac{\mu_0 }{4\pi}\frac{\mathbf{ds}_1\wedge\mathbf{r}}{r^3}\mathbf{\hat{u}_n}d\Sigma_2\right)\equiv M_{12}i\]
Dove, notando che l'integrale doppio all'interno delle parentesi dipende dolo dal mezzo e dalla geometria della superficie, abbiamo definito il \textbf{coefficiente di mutua induzione} M.
Analogamente abbiamo
\[\phi_{12}= M_{21}i\]
Visto che i due flussi sono uguali, anche i due coefficienti di mutua induzione lo sono, in generale \(M_{12}=M_{21}\) per qualsiasi coppia di superfici.\\
Ne deduciamo che l'espressione generale del flusso del campo magnetico è
\[\phi(B) = M i\]
Il coefficiente di mutua induzione ha una sua unità di misura: l'Henry (H)
\[H = \frac{Wb}{A}\]
\begin{esercizio}[Coefficiente di mutua induzione per due solenoidi concentrici]
Verifichiamo quanto ottenuto nel caso particolare di due solenoidi concentrici
\[\phi_{12}=\iint_{\Sigma_2}\mathbf{B}_1\mathbf{\hat{u}_n}d\Sigma_2 = B_1\Sigma_2 n_2=B_1 S n_2 = \mu_0 i_1 n_1 n_2 S = M_{12}i_1\]
\[\phi_{21}=\iint_{\Sigma_1}\mathbf{B}_2\mathbf{\hat{u}_n}d\Sigma_1 = B_2\Sigma_1 n_1=B_2 S n_1 = \mu_0 i_2 n_1 n_2 S = M_{21}i_2\]
\[\Rightarrow M_{12}=\mu_0 n_1 n_2 S=M_{21}\]
\end{esercizio}
Un fatto di fondamentale importanza è che si crea un flusso non solo fra le due superfici ma anche di una superficie con sè stessa
\[\iint_{\Sigma}\mathbf{B}_1\mathbf{\hat{u}_n}d\Sigma_1 = \int_{\Sigma}\oint_{\Gamma}\frac{\mu_0 i }{4\pi}\frac{\mathbf{ds}\wedge\mathbf{r}}{r^3}\mathbf{\hat{u}_n}d\Sigma\equiv L i\]
dove, osservando che nell'integrale tutto dipende dalla geometria e dal mezzo tranne la corrente, definiamo il coefficiente di autoinduzione o \textbf{induttanza} L.
\newpage
\section{Campi magnetici variabili nel tempo}
In elettrostatica valevano le relazioni
\begin{align*}
&\n\wedge\mathbf{E}=0&&\n\wedge\mathbf{B}=\mu_0\mathbf{J}\\
&\n\mathbf{E}=\frac{\rho}{\varepsilon_0}&&\n\mathbf{B}=0
\end{align*}
ma abbiamo già visto che in elettrodinamica queste non sono più valide in quanto un campo elettrico variabile genera campo magnetico, secondo le legge di Ampere-Maxwell, che corregge la seconda di queste relazioni. A partire dalle seguenti evidenze sperimentali, raccolte durante la prima metà dell''800, si giunse ad una correzione della prima relazione.\\
\begin{itemize}
	\item Se si avvicina un magnete ad un circuito in cui non passa corrente e a cui è collegato un galvanometro, questo segna un flusso di elettroni nel circuito (si genera corrente elettrica). Disponendo il magnete con il polo nord verso il circuito, la corrente è positiva se si allontana il magnete e negativa se lo si avvicina, se si invertono i poli si inverte anche il segno della corrente. Inoltre, più velocemente si muove il magnete maggiore è l'intensità di corrente rilevata.
	\item Lo stesso effetto si riscontra se al posto del magnete si usa un circuito attraversato da corrente elettrica. In questo caso a seconda del verso della corrente nel secondo filo si ottiene una corrente sul primo di segno positivo o negativo. (in conformità con il principio di equivalenza di Ampere magnete e spira attraversata da corrente hanno lo stesso effetto).
\end{itemize}
Ciò che risulta evidente è che in qualche modo il campo magnetico è capace di indurre corrente elettrica su un filo. Queste evidenze portarono a formulare la \textbf{legge di Faraday} nel 1831, questa viene anche detta legge di Faraday-Henry o Faraday-Neumann o Faraday-Neumann-Lenz poiché a legge di Lenz ne è un corollario.
\begin{align}\label{eq:Faraday_integrale}
	\mathcal{E}_I = -\frac{d\phi(B)}{dt}
\end{align}
\[i_I = -\frac{1}{R}\frac{d\phi(B)}{dt}\]
Si osservi che se esiste una corrente elettrica esiste un campo elettrico, quello indotto dal campo magnetico però non può essere conservativo perché la condizione di conservatività è che la circuitazione del campo sia nulla ma
\[\mathcal{E}_I = \oint_{\Gamma}\mathbf{E}\mathbf{ds}=-\frac{d\phi(B)}{dt}\neq 0\]
Cosa indica il segno negativo presente nella legge? La corrente indotta, generandosi, varia da zero ad un valore, a seconda di come varia il flusso del campo magnetico; come tutte le correnti variabili genera campo magnetico, questo ha sempre verso opposto al campo che lo genera. Da ciò segue, svolgendo semplici considerazioni applicando la regola della mano destra, che il segno della corrente è sempre opposto a quello della variazione del campo magnetico.  Se così non fosse sarebbe violata la conservazione dell'energia infatti avremmo che un campo magnetico variando genera corrente variabile che genera un campo magnetico che si somma a quello iniziale e, facendolo aumentare, produce altra corrente che porterebbe ad un loop infinito che produrrebbe energia infinita, il che è assurdo. Avviene invece che si genera un campo magnetico opposto a quello iniziale che tende ad annullarlo. Possiamo pensare lo stesso problema avvalendoci del principio di equivalenza di Ampere, il filo su cui passa corrente indotta è equivalente ad un'altro magnete che, a causa del segno negativo nella formula, ha i poli invertiti rispetto al magnete iniziale, in questo modo due poli uguali, avvicinandosi, generano una forza che li respinge e quindi si oppone all'avvicinarsi del magnete iniziale che avrebbe aumentato il campo magnetico  (e quindi in definitiva si crea un campo magnetico indotto opposto all'aumento del campo iniziale).\\
Vediamo ora alcune modalità in cui il campo elettrico può variare nel tempo:
\begin{esercizio}[Variare il flusso per rotazione del circuito]
Vi sono vari metodi per far variare il flusso del campo magnetico (la variazione di flusso nel tempo è facilmente misurabile mediante la f.e.m.), uno di questi è lasciare il campo magnetico costante (non muovere il magnete) e variare il vettore normale alla superficie individuata dal circuito (ruotando il circuito su un asse) e quindi l'angolo $\theta$ compreso fra questo versore ed il campo magnetico. La rotazione ha velocità angolare $\omega(t)$
\[\mathcal{E}_I = -\frac{d\phi (B)}{dt} = -\frac{d}{dt}\oint_{\Gamma}\mathbf{B}\mathbf{ds} = -\frac{d}{dt}B\Sigma \cos\theta=B\Sigma \sin\theta\frac{d\theta}{dt}=B\Sigma\sin\theta\omega(t)\]
\end{esercizio}
\begin{esercizio}[Variare il flusso per variazione della corrente]
Consideriamo un filo infinito giacente sull'asse z con un flusso di corrente verso l'alto variabile nel tempo ed un circuito quadrato di lato l. A priori non sappiamo che verso ha la corrente indotta quindi non sappiamo come orientare la superficie  (che convenzionalmente si orienta con la regola della mano destra a seconda del verso della corrente), quindi scegliamo arbitrariamente un'orientamento, se questo è lo stesso che induce il verso della corrente indotta allora l'espressione che otterremo sarà positiva, se no negativa (quindi l'orientamento inizialmente scelto era sbagliato e bisogna invertirlo). In questo caso scegliamo arbitrariamente verso antiorario (versore superficie uscente dal foglio), in questo modo l'angolo fra questo è il campo magnetico è nullo
\[\mathbf{B}_{filo} = \frac{\mu_0 i}{2\pi y}\mathbf{\hat{i}}\]
\[\phi(B)=\int_0^l\int_{y_0}^{y_0+l} \frac{\mu_0 i}{2\pi y}dydz= \mu_0 i l \int_{y_0}^{y_0+l}\frac{dy}{y}=\mu_0 i l \ln\left(\frac{y_0 + l}{y_0}\right)\]
\[\mathcal{E}_I=-\frac{d}{dt}\left(\mu_0 i l \ln\left(\frac{y_0 + l}{y_0}\right)\right) = \mu_0 \frac{di}{dt} l \ln\left(\frac{y_0 + l}{y_0}\right)\]
\[i_I = -\frac{\mu_0 l}{2\pi R}\ln\left(\frac{l}{y_0}\frac{di}{dt}\right)\]
Il segno della corrente risulta negativo: la scelta dell'orientamento iniziale era sbagliata. 
\end{esercizio}
\begin{esercizio}[Variare il flusso per deformazione del circuito]
Consideriamo un campo magnetico costante ed un circuito quadrato di lato l e resistenza R e superficie $\Sigma$ immerso in esso, inizialmente la corrente sul filo è nulla. In seguito il lato destro del circuito viene spostato verso destra con velocità v(t) aumentando la superficie del circuito, essendoci un campo magnetico costante il flusso aumenta nel tempo con l'aumentare della superficie. Sia x(t) la lunghezza della coppia di lati che varia nel tempo. Scegliamo arbitrariamente orientamento di \(\Sigma\) antiorario, l'angolo fra il versore superficie ed il campo è nullo, \(\cos(0)= 1\)
\[\Sigma(t) = l x(t)\] 
\[\phi(B) = \iint_{\Sigma(\Gamma)}\mathbf{B}\mathbf{\hat{u}_n}d\Sigma = B\int dx\int dy = Blx(t)\]
\[\mathcal{E}_I = -Bl\frac{dx(t)}{dt} = -Blv(t)\]
\[i_I(t) = -\frac{Bl}{R}v(t)\]
Il segno della corrente è negativo, avevamo scelto l'orientamento sbagliato, la corrente scorre in senso orario, il campo elettrico indotto è quindi verso il basso.\\
Si osservi che i portatori di carica sul lato del circuito in movimento hanno una velocità v(t) quindi sono soggetti ad una forza di Lorentz verso l'alto, si genera dunque un campo elettrico verso l'alto, opposto a quello iniziale, si verifica un piccolo effetto Hall
\[\mathbf{F}_l = q \mathbf{v}\wedge\mathbf{B}\]
\[\mathbf{E}_I=\frac{\mathbf{F}}{q} = \mathbf{v}\wedge\mathbf{B}\]
\end{esercizio}
L'espressione della legge di Faraday vista precedentemente, riferendosi al flusso su una superficie, ha forma integrale ed esprime una proprietà media di un circuito. Possiamo ricavare la forma differenziale usando il teorema di Stokes
\[\mathcal{E} = \oint_{\Gamma}\mathbf{E}_I \mathbf{ds} = -\frac{d}{dt}\oint_{\Sigma}\mathbf{B\mathbf{\hat{u}_n}}d\Sigma\]
\begin{align}\label{eq:Faraday_differenziale}
	\n\wedge\mathbf{E}_I = -\frac{\d \mathbf{B}}{\d t}
\end{align}
Da questa relazione notiamo che un campo elettrico che si genera a partire da una variazione del campo magnetico nel tempo non è conservativo, perchè il suo rotore è diverso da zero.
\newpage
\section{Circuiti in regime quasi stazionario: il circuito RL}
Solitamente l'induttanza L di un circuito è talmente bassa da essere trascurabile, tuttavia in presenza di un solenoide l'induttanza aumenta e non è più trascurabile. Il circuito RL è un modello di circuito in cui si considerano resistenza ed induttanza concentrate in due punti, il componente elettrico che genera campo magnetico a seguito di un passaggio di corrente è detto \textbf{induttore} e si indica con una spirale (che ricorda la forma di un solenoide, un ottimo induttore)
\begin{figure}[h!]
	\centering
	\includegraphics[width=0.6\linewidth]{../images/circuito_RL}
	\caption{Schema di un circuito RL.}
	\label{fig:circuitorl}
\end{figure}
\FloatBarrier
Il flusso indotto genera una f.e.m. autoindotta che si somma a quella preesistente, possiamo quindi applicare la legge di Ohm 
\[\phi(B) = L i \quad \mathcal{E}_I = -L\frac{d i}{dt}\]
\[\mathcal{E} + \mathcal{E}_I = Ri \]
\[\Rightarrow \mathcal{E} = Ri+L\frac{di}{dt}\]
Quest'ultima è un'equazione differenziale di variabile i a variabili separabili; risolvendola otteniamo la corrente in funzione del tempo. Separando le variabili otteniamo
\[\frac{di}{\mathcal{E}-Ri} = \frac{dt}{L}\]
\[\int_{0}^{i}\frac{di}{\mathcal{E}-Ri} = \frac{1}{L}\int_{0}^{t}dt\]
\[i(t) = \frac{\mathcal{E}}{R}(1-e^{-\frac{t}{\tau}})\]
dove $\tau$ è detto tempo caratteristico del circuito (analogamente a quanto visto per il circuito RC) ed è definito come \(\tau \equiv \frac{L}{R}\).
\[\mathcal{E}_I = -L\frac{di(t)}{dt}=-\mathcal{E}e^{-\frac{t}{\tau}}\]
\[i_I(t) = \frac{\mathcal{E}}{R} = -\frac{\mathcal{E}}{R}e^{-\frac{t}{\tau}}\]
Svolgiamo alcune osservazioni: il grafico della corrente nel tempo è
\begin{figure}[h!]
	\centering
	\includegraphics[width=0.4\linewidth]{../images/corrente_RL}
	\caption{Grafico della corrente (curva continua) e della corrente indotta (tratteggiata) nel tempo a partire dal momento in cui viene chiuso il circuito. Sulle ascisse sono riportati i multipli del tempo caratteristico, dopo 4\(\tau\) la corrente ha raggiunto il suo valore di regime.}
	\label{fig:correnterl}
\end{figure}
\FloatBarrier
Ci spieghiamo fisicamente questo grafico perché nell'istante in cui è chiuso il circuito la corrente aumenta, questa variazione genera una f.e.m. e duna corrente indotta che rallenta l'aumento di corrente, tendendo a farla stabilizzare a \(\frac{\mathcal{E}}{R}\). La tendenza a stabilizzarsi equivale ad una riduzione graduale della variazione e quindi ad una riduzione della forza elettromotrice indotta e quindi della corrente indotta. Si noti dunque che per \(t\to\infty\) l'effetto dell'induttore si annulla e il valore della corrente si stabilizza a quello del caso elettrostatico; è possibile vedere la corrente indotta come la differenza fra il valore a cui si stabilizza la corrente \((i_{\infty})\) e il suo valore al tempo t
\[i_I = i_\infty - i(t)\]
Possiamo infine svolgere uno studio del circuito RL da un punto di vista energetico
\[f.e.m.:\ \mathcal{E} = Ri+L\frac{di}{dt}\]
\[potenza:\ \mathcal{E}i = Ri^2+L\frac{di}{dt}i\]
\[energia:\  \mathcal{E}idt = Ri^2dt+Lidi\]
Notiamo che l'energia del circuito è formata da una componente di energia dissipata per effetto Joule ed una che è l'energia accumulata all'interno del solenoide in cui scorre corrente i aumentando quest'ultima di un valore infinitesimo di. Concentrandosi sul secondo termine, integrando, otteniamo l'energia immagazzinata in un solenoide in cui passa corrente i. 
\[U_{sol} = \int_{0}^{i}Lidi=\frac{1}{2}Li^2\]
In analogia a quanto fatto per i condensatori, dove avevamo ricavato prima l'energia di un condensatore a facce piane parallele (caso particolare) in funzione del campo elettrico per poi ricavare un'espressione della densità di energia del campo elettrico di validità generale, ci chiediamo ora se sia possibile ricavare una densità di energia del campo magnetico di validità generale a partire dall'energia del solenoide (caso particolare). 
\[\phi(B)_{sol} = \iint_{\Sigma}\mathbf{B}\mathbf{\hat{u}_n}d\Sigma = N \mu_0 \frac{N}{l}\Sigma i = Li\]
dove la N aggiuntiva sta nel fatto che bisogna considerare il flusso passante per tutte le N spire formanti il solenoide.\\
\[\Rightarrow L = \frac{\mu_0 N^2\Sigma}{l}\]
\[U_{sol} = \frac{1}{2}Li^2 = \frac{1}{2}\frac{\mu_0 N^2\Sigma}{l}i^2=\frac{1}{2}\frac{\mu_0^2 N^2(\Sigma l)}{\mu_0 l^2}i^2=\frac{1}{2}\frac{B^2}{\mu_0}(\Sigma l)\]
Dove \(\Sigma l= V \) è il prodotto tra superficie e lunghezza del solenoide, quindi il suo volume. Per svincolarsi dal caso particolare del solenoide dividiamo per il volume, ottenendo un'espressione di densità di energia magnetica che non presente termini dipendenti da caratteristiche del solenoide (si dimostra che ha validità generale)
\begin{align}\label{eq:densità_energia_magnetica}
u_L = \frac{U}{V} =\frac{1}{2}\frac{B^2}{\mu_0} 
\end{align}
\newpage
\section{Natura ondulatoria del campo elettromagnetico}
Riassumendo quanto fatto fin ora, abbiamo ottenuto delle nuove relazioni di validità generale per campi elettrici e magnetici variabili nel tempo, di cui l'elettrostatica è un caso particolare. Riscriviamo le quattro equazioni presentate ad inizio capitolo correggendole
\begin{align*}
	&\n\wedge\mathbf{E}=-\frac{\d \mathbf{B}}{\d t}&&\n\wedge\mathbf{B}=\mu_0\mathbf{J}+\mu_0\varepsilon_0\frac{\d \mathbf{E}}{\d t}\\
	&\n\mathbf{E}=\frac{\rho}{\varepsilon_0}&&\n\mathbf{B}=0
\end{align*}
Queste quattro sono anche conosciute come \textbf{equazioni di Maxwell}, a partire da queste è possibile studiare qualsiasi fenomeno elettrico e magnetico dal punto di vista classico.\\
Un grande contributo che fornì il genio di Maxwell alla fisica fu quello di dimostrare la natura ondulatoria del campo elettromagnetico, e quindi della luce, risolvendo apparentemente l'annosa disputa sulla natura della luce (più avanti si scoprì la doppia natura della luce ma questa è un'altra storia). Per semplicità trattiamo il caso particolare di un punto in cui la densità di carica è nulla ($\rho = 0$), non vi è corrente (\(\mathbf{J}=0\)) e le uniche fonti di campo elettromagnetico sono poste a distanza infinita (in modo da trattare il caso semplice di onde piane poiché, ad esempio, un'onda sferica generata a distanza infinita ha raggio di curvatura tanto grande da poter essere approssimata ad onda piana). Le quattro equazioni diventano quindi
\begin{align*}
	&\n\wedge\mathbf{E}=-\frac{\d \mathbf{B}}{\d t}&&\n\wedge\mathbf{B}=\mu_0\varepsilon_0\frac{\d \mathbf{E}}{\d t}\\
	&\n\mathbf{E}=0&&\n\mathbf{B}=0
\end{align*}
 L'equazione dell'onda piana, che poniamo si propaghi solo sull'asse x, è
\[\frac{\d^2 \mathbf{\Psi}}{\d x^2} = \frac{1}{v^2}\frac{\d^2 \mathbf{\Psi}}{\d t^2}\]
dove v è la velocità dell'onda.\\
Svolgiamo ora una serie di considerazioni a partire dalle quattro equazioni del campo elettrico e magnetico di cui sopra 
\begin{itemize}
	\item \(\n\mathbf{E}=\frac{\d E_x}{\d x}+\frac{\d E_y}{\d y}+\frac{\d E_z}{\d z} =\frac{\d E_x}{\d x}=0\)
	da cui otteniamo che il campo elettrico sull'asse x è costante nello spazio e, non essendoci sorgenti per ipotesi, esso dev'essere nullo. 
	\item \(\n\mathbf{B}=\frac{\d B_x}{\d x}+\frac{\d B_y}{\d y}+\frac{\d B_z}{\d z} =\frac{\d B_x}{\d x}=0\)
	da cui otteniamo che il campo magnetico sull'asse x è costante nello spazio e, non essendoci sorgenti per ipotesi, esso dev'essere nullo. 
	\item 
	\begin{align*}
		&\n\wedge\mathbf{E}=det
		\begin{pmatrix}
			\hat{\mathbf{i}}&\hat{\mathbf{j}}&\hat{\mathbf{k}}\\
			\frac{\d}{\d x}&\frac{\d}{\d y}&\frac{\d}{\d z}\\
			E_x&E_y&E_z
		\end{pmatrix}=\\\\
		&\left(\frac{\d E_z}{\d y}-\frac{\d E_y}{\d z}\right)\hat{\mathbf{i}}
		-\left(\frac{\d E_z}{\d x}-\frac{\d E_x}{\d z}\right)\hat{\mathbf{j}}
		+\left(\frac{\d E_y}{\d x}-\frac{\d E_x}{\d y}\right)\hat{\mathbf{k}}=\\\\
		&0\mathbf{\hat{i}}-\frac{\d Ez}{\d x}\mathbf{\hat{j}}+ \frac{\d E_y}{\d x}\mathbf{\hat{k}}=-\frac{\d B_x}{\d t}\mathbf{\hat{i}} - \frac{\d B_y}{\d t}\mathbf{\hat{j}} -\frac{\d B_z}{\d t}\mathbf{\hat{k}}\\\\
		&\Rightarrow
		\begin{cases}
			\frac{\d B_x}{\d t} = 0\\\\
			\frac{\d Ez}{\d x} = \frac{\d B_y}{\d t}\\\\
			\frac{\d E_y}{\d x} = -\frac{\d B_z}{\d t}
		\end{cases}
		\end{align*}
	Dalla prima di queste tre equazioni deduciamo che il campo elettrico sull'asse x è costante nel tempo, non essendoci sorgenti di campo magnetico per ipotesi deduciamo che deve essere costante e nullo. 
	\item Analogamente a quanto fatto per il rotore del campo elettrico per il campo magnetico otteniamo 
	\begin{align*}
		&0\mathbf{\hat{i}}-\frac{\d Bz}{\d x}\mathbf{\hat{j}}+ \frac{\d B_y}{\d x}\mathbf{\hat{k}}=\mu_0\varepsilon_0\frac{\d E_x}{\d t}\mathbf{\hat{i}} + \mu_0\varepsilon_0\frac{\d E_y}{\d t}\mathbf{\hat{j}} +\mu_0\varepsilon_0\frac{\d E_z}{\d t}\mathbf{\hat{k}}\\\\
		&\Rightarrow
		\begin{cases}
			\frac{\d E_x}{\d t} = 0\\\\
			\frac{\d Bz}{\d x} = -\mu_0\varepsilon_0\frac{\d E_y}{\d t}\\\\
			\frac{\d B_y}{\d x} = \mu_0\varepsilon_0\frac{\d E_z}{\d t}
		\end{cases}		
	\end{align*}
	Come prima, dalla prima equazione deduciamo che il campo magnetico è nullo sull'asse x e costante nel tempo.
\end{itemize}
Infine, prendiamo dal punto 3 la seconda equazione del sistema e, derivando parzialmente rispetto ad x, otteniamo
\[\frac{\d^2 E_z}{\d x^2} = \frac{\d^2 B_y}{\d x\d t} \]
dalla terza equazione del sistema del punto quattro, derivando rispetto al tempo otteniamo
\[\frac{\d^2 B_y}{\d x\d t} = \mu_0\varepsilon_0 \frac{\d^2 E_z}{\d t^2}\] 
Uguagliando le precedenti otteniamo
\[\frac{\d^2 E_z}{\d x^2}=\mu_0\varepsilon_0 \frac{\d^2 E_z}{\d t^2}\]
Che ha la forma dell'equazione di un'onda sull'asse z. Possiamo ripetere lo stesso ragionamento sull'asse y sfruttando l'altra coppia di equazioni 
\[\frac{\d^2 E_y}{\d x^2}=\mu_0\varepsilon_0 \frac{\d^2 E_y}{\d t^2}\]
Inoltre, facendo prima le derivate rispetto al tempo e poi rispetto ad x otteniamo un risultato analogo per il campo magnetico
\[\frac{\d^2 B_z}{\d x^2}=\mu_0\varepsilon_0 \frac{\d^2 B_z}{\d t^2}\]
\[\frac{\d^2 B_y}{\d x^2}=\mu_0\varepsilon_0 \frac{\d^2 B_y}{\d t^2}\]
Sommando vettorialmente, ricordando che la componente sull'asse x di campo elettrico e magnetico è nulla e costante nel tempo e sulle x, otteniamo l'espressione
\[\frac{\d^2 \mathbf{E}}{\d x^2}=\mu_0\varepsilon_0 \frac{\d^2 \mathbf{E}}{\d t^2}\]
\[\frac{\d^2 \mathbf{B}}{\d x^2}=\mu_0\varepsilon_0 \frac{\d^2 \mathbf{B}}{\d t^2}\]
dove nel caso particolare considerato abbiamo \(\mathbf{E}=(0, E_y,E_z)\); \(\mathbf{B}=(0, B_y, B_z)\). Deduciamo che un campo elettrico o magnetico generati all'infinito, che si propagano solo su due assi, sono descrivibili come un'onda piana sul terzo asse. Campo elettrico e magnetico si propagano ortogonalmente alla direzione in cui si propaga l'onda
\begin{definizione}[onda trasversale]
	Un' onda trasversale è un'onda in movimento che è composta da oscillazioni che avvengono perpendicolari alla direzione del trasferimento di energia. Se un'onda trasversale si muove nella direzione positiva x, le sue oscillazioni sono nelle direzioni sopra e sotto che giacciono nel piano yz.
\end{definizione}
Notiamo infine che $\mu_0\varepsilon_0$ ha, come ci aspettavamo dalla forma generale dell'equazione delle onde piane, le dimensioni di una velocità alle meno 2 dunque la velocità di un'onda elettrica o magnetica è 
\[v = \frac{1}{\sqrt{\mu_0\varepsilon_0}}\]
che è una costante che risulta essere uguale alla velocità della luce. 
\newpage
\section{Campi magnetici stazionari nella materia}
\subsection{Evidenze sperimentali}
Consideriamo un solenoide (solenoide 1) di raggio R con asse giacente sull'asse delle x nel quale passa una corrente i. Poniamo un altro solenoide (solenoide 2) concentrico al primo, di raggio minore, sulla soglia del primo solenoide e connesso ad un dinamometro per misurare la forza con cui è attratto o respinto dal solenoide in cui passa corrente. La forza è data da 
\[\mathbf{F} = -\n U = -\n(\mathbf{m}\mathbf{B})\]
dove \(\mathbf{m}\) è il momento magnetico del solenoide 2. Questo è costante in direzione e modulo nell'apparato da noi considerato ma a seconda del verso in cui scorre la corrente può essere positivo o negativo.
\[\mathbf{F}=\pm m \frac{d B_x}{dx}\mathbf{\hat{i}}\]
Se la corrente è oraria la forza va verso il basso, se antioraria va verso l'alto.\\
Se ora al posto del solenoide 2 poniamo un campione di materiale di egual forma e volume si osserva che alcuni materiali sono attratti, altri respinti con forze che sono solitamente deboli ma che possono arrivare ad essere molto grandi per alcuni materiali. Nella maggior parte dei casi la direzione della forza dipende dal verso della corrente ma talvolta si osserva che la forza ha verso indipendente dal verso della corrente (sorprendentemente per gli sperimentatori ottocenteschi che osservarono questo fenomeno per la prima volta).\\
Per svolgere gli esperimenti, come detto, bisogna usare sempre lo stesso volume di materiale dunque introduciamo una nuova grandezza
\[\frac{\mathbf{F}}{\tau} = \frac{m}{\tau}\frac{d B_x}{dx}\mathbf{\hat{i}}\equiv \mathbf{M}\frac{dB_x}{dx}\]
dove abbiamo definito M, vettore di magnetizzazione. A seconda delle caratteristiche di \(\mathbf{M}\) che si osservano mediante l'apparato sperimentale di cui sopra possiamo distinguere i materiali in tre categorie
\begin{itemize}
	\item Diamagnetici: se M è negativa di modulo piccolo ed equiversa al campo magnetico, ovvero quando il materiale è debolmente respinto
	\item Paramagnetici: se M è positiva di modulo piccolo ed equiversa al campo magnetico, ovvero quando il materiale è debolmente attratto
	\item Ferromagnetici: se M è positiva di modulo grande ed equiversa al campo magnetico, ovvero quando il materiale è fortemente attratto. In questo caso non vi è una relazione tra M e B univoca, che dipende da come si raggiunge il valore di M. 
\end{itemize}
Sempre sperimentalmente si nota che se l'esperimento avviene in un mezzo invece che nel vuoto si ha che il campo magnetico è proporzionale a quello che avremmo osservato nel vuoto con una costante di proporzionalità dipendente unicamente dal mezzo, detta costante di permeabilità magnetica relativa (\(k_m\)).
\[B = B_0 k_m = \mu_0 k_m n i = \mu n i\]
dove $\mu$ è la costante di permeabilità magnetica assoluta. 
Il valore di \(k_m\) dipende dal mezzo e può essere maggiore o minore di 1, a differenza della costante di permeabilità elettrica che era sempre minore di uno e riduceva sempre il campo elettrico. Questo fatto, osservato per la prima volta nel caso specifico di un solenoide, si osserva avere validità generale,per mezzi omogenei (\(k_m\) è la stessa in tutto il mezzo) e con densità uguale in tutti i punti. La formula di Biot-Savart in un mezzo con queste caratteristiche è quindi 
\[\mathbf{dB} = \frac{\mu}{4\pi} i \frac{\mathbf{ds}\wedge \mathbf{r}}{r^3}\] 
Possiamo definire una grandezza analoga alla suscettibilità elettrica: la suscettibilità magnetica 
\[\mathbf{B}-\mathbf{B_0}=(k_m - 1)\mathbf{B_0} \equiv \chi_m \mathbf{B_0}\]
Possiamo quindi riclassificare i materiali in base ai valori di suscettibilità magnetica che presentano
\begin{itemize}
	\item Diamagnetici: \(k_m <1,\ \chi_m < 0\sim 10^{-5}\) \(\Rightarrow\) \(B< B_0\)
	\item Paramagnetici: \(k_m >1,\ \chi_m > 0\sim 10^{-5}\) \(\Rightarrow\) \(B > B_0\)
	\item Ferromagnetici: \(k_m >1,\ \chi_m < 0\sim 10^{3}-10^{5}\) \(\Rightarrow\) \(B > B_0\)
\end{itemize}
Riordinando i termini della precedente relazione otteniamo una forma più espressiva
\[\frac{B-B_0}{B}=\chi_m\]
che mette in luce il fatto che $\chi_m$ è la frazione di quanto varia il campo magnetico in un mezzo rispetto al vuoto. Riordinando nuovamente i termini possiamo anche scrivere
\[\mathbf{B}=\mathbf{B_0}+\mathbf{B_0}\chi_m\]
che mette in luce il fatto che il campo magnetico in un mezzo è costituito da una componente che è il valore che avrebbe avuto nel vuoto più una componente che esprime il contributo del mezzo (che può essere sia riduttivo che accrescitivo). 
\subsection{Interpretazione microscopica della magnetizzazione}
Il contributo di campo magnetico offerto dal mezzo ha natura atomica: esiste una corrente atomica detta "amperiana" causata dai moti degli elettroni, le cui traiettorie possono essere viste come piccole spire. Visto che il loro moto è genera una corrente variabile, si genera per ogni elettrone un piccolo campo magnetico; ad ogni elettrone, come si farebbe per una spira, è possibile associare un momento magnetico \(\mathbf{m_e}\). Essendo gli orientamenti degli atomi casuali il momento magnetico globale di un materiale formato da molti atomi è solitamente nullo. In presenza di un campo magnetico, proprio come nel caso delle spire, i momenti magnetici dei singoli atomi sperimentano un momento delle forze e tendono ad allinearsi col campo, producendo così un momento magnetico globale nel materiale. Vediamo più in particolare l'interpretazione microscopica del fenomeno della magnetizzazione ricordando che quella che possiamo offrire è solo una trattazione qualitativa in quanto per avere precisione quantitativa è necessaria la teoria quantistica.\\
In generale gli elettroni generano una corrente di magnetizzazione \(i_m\), vogliamo trovare un'espressione di questa corrente. Cominciamo con l'osservare che il momento magnetico  medio del materiale (\(<\mathbf{m}>\)) è parallelo al campo magnetico esterno \((\mathbf{B})\) e che il momento magnetico (\(\Delta\mathbf{m}\)) di una porzione di materiale di volume \(\Delta \tau\) è pari al momento magnetico medio per il numero di atomi contenuto in quella porzione di materiale (\(\Delta N\)), inoltre ricordando la definizione del vettore di magnetizzazione possiamo scrivere
\[<\mathbf{m}>//\mathbf{B}\]
\[\Delta \mathbf{m}=\Delta N<\mathbf{m>}\]
\[\mathbf{M}=\frac{\Delta \mathbf{m}}{\Delta \tau}=\frac{\Delta N }{\delta \tau}<\mathbf{m}>=n<\mathbf{m}>\]
Si noti la forte analogia momento di polarizzazione-momento magnetico polarizzazione-magnetizzazione.\\
Consideriamo il caso particolare in cui \(\mathbf{M}\) è costante in tutto il materiale (nei dielettrici avevamo considerato un materiale in cui P era costante in tutti i punti), ovvero ad esempio un cilindro nel quale quindi si genera un momento di magnetizzazione uguale in tutti i punti. Se ferromagnetico o paramagnetico per definizione abbiamo \(\mathbf{M}//\mathbf{B}\) ed equiversi (come con \(\mathbf{P}//\mathbf{E}\) se \(P\) costante in tutto il dielettrico). Consideriamo una sezione di cilindro di altezza dz e dividiamolo in piccoli prismi a superficie di base quadrata. Per ogni volumetto si ha un momento di magnetizzazione infinitesimo
\[\mathbf{dm} = \mathbf{M}d\tau = \mathbf{M} ds dz = M dz \mathbf{ds}\]  
dove l'ultima uguaglianza è possibile perché il vettore di magnetizzazione è parallelo al versore normale la superficie. Per il teorema di equivalenza di Ampere il prisma, che è un piccolo magnete, è equivalente ad una spira in cui scorre una corrente di magnetizzazione \(di_M\) di superficie \(\mathbf{ds}\)
\[\mathbf{dm} = di_M \mathbf{ds} = M dz \mathbf{ds}\]
\[\Rightarrow d i_M = M dz\]
essendo \(\mathbf{M}\) costante per ipotesi la corrente ha sempre la stessa intensità e dipende solo dall'altezza della sezione di cilindro considerata. Ciò vuol dire che in tutti i prismi di cui è composto il cilindro vi è una corrente che circola sempre nello stesso verso, ciò vuol dire che all'interno del cilindro le correnti si annullano vicendevolmente tranne quelle al bordo, ne segue che la corrente id magnetizzazione sta solo sul bordo e, più questo è spesso maggiore sarà la corrente totale. Possiamo considerare prismi di volume tanto piccoli da contenere solo un atomo, dunque la corrente di magnetizzazione a livello atomico è data solo dagli elettroni nell'atomo. per ottenere la corrente di magnetizzazione dell'intero cilindro integriamo
\[i_{ml} = \int_{0}^{h}di_M=Mh\]
Analogamente a quanto già fatto per la corrente elettrica con il vettore di densità di corrente, definiamo un \textbf{vettore di densità lineare di corrente di magnetizzazione}
\[J_{ml} \equiv \frac{i_M}{h}= M\]
o, in forma vettoriale
\[\mathbf{J_{ml}} \equiv \mathbf{M}\wedge \mathbf{\hat{n}}\]
dove \(\mathbf{\hat{n}}\) è il versore normale alla superficie del cilindro. Questa relazione è analoga alla \ref{eq:densità_polarizzazione}. Se \(\mathbf{M}\) non è costante le correnti non si annullano all'interno del cilindro quindi si generano correnti di magnetizzazione interne al cilindro (analogamente a quando \(\mathbf{P}\) non è costante nel dielettrico e si creano densità di polarizzazione interne).\\
Cosa succede sulla superficie di contatto fra i du prismi? Su ogni superficie laterale c'è una corrente che girano nello stesso senso
\[di_1 = M_z dz\]
\[di_2 = M'_z dz\]
Sulla superficie laterale però ci sono entrambe le correnti ed hanno verso opposto
\[di_1 - di_2= (M_z-M'_z)dz = -\left(\frac{\d M_z}{\d x}\right)dzdx \]
dove, essendo i prismi di altezza infinitesima dz abbiamo sostituito
\[dM_z = M_z - M'_z\]
\begin{figure}[h!]
	\centering
	\includegraphics[width=0.7\linewidth]{../images/prismi_1}
	\caption{Rappresentazione di due prismi di altezza dz e superficie di base ds sulla cui superficie è presente una corrente \(di_M\) che gira nello stesso senso. In blu l'asse z, verde l'asse y e rosso l'asse x.}
	\label{fig:prismi1}
\end{figure}
\FloatBarrier
Analogamente possiamo ragionare sull'asse invertendo gli assi, ottenendo
\[di_4 - di_3 = (M'_x - M_x)dx = -\left(\frac{\d M_x}{\d z}\right)dz dx\]
Dunque, in totale, la variazione di corrente sull'asse y è
\[di_y = (di_1 -di_2)+(di_4-di_3)=\left(\frac{\d M_x}{\d z}-\frac{\d M_z}{\d x}\right)dxdz\]
\[J_y = \frac{di_y}{dxdz} = \left(\frac{\d M_x}{\d z}-\frac{\d M_z}{\d x}\right) \]
Quest'ultima è la componente \(\mathbf{\hat{j}}\) del rotore del vettore di magnetizzazione dunque, ripetendo lo stesso ragionamento sui tre assi otteniamo
\[\n\wedge\mathbf{M} = \mathbf{J}_{mS}\]
dove \(\mathbf{J}_{mS}\) è la densità di corrente superficiale, presente all'interno del materiale non omogeneo, anisotropo a causa della non uniformità di \(\mathbf{M}\). Questa relazione è analoga alla \ref{eq:gradiente_vettore_polarizzazione}. Possiamo quindi applicare il teorema di Stokes per ottenere un'equivalente della legge di Ampere ma questa volta per la corrente di magnetizzazione
\[\oint_\Gamma\mathbf{M}\mathbf{dz} = i_{mS}\]
Introducendola corrente di magnetizzazione (sia lineare che superficiale) è possibile render conto delle calamite: basta correggere la legge di Ampere-Maxwell considerando anche la corrente di magnetizzazione nella componente di Ampere della legge
\[\oint_\Gamma \mathbf{B}\mathbf{dz} = \mu_0 i_{tot} + \mu_0\varepsilon_0\frac{\d\Phi(\mathbf{E})}{\d t}\]
dove \(i_{tot} = i+i_m\). In una calamita anche se non scorre corrente è comunque presente una corrente di magnetizzazione di origine atomica. In questo modo ci spieghiamo anche l'osservazione sperimentale sposta all'inizio della trattazione dei magneti per cui spezzando un magnete con due poli si ottiene un'altro magnete con due poli e non è possibile ottenere un monopolo magnetico: In realtà un magnete è formato da tante spire e, dividendolo, non si fa altro che ridurre il numero di spire (e quindi la corrente di magnetizzazione, che dipende solo dall'altezza del magnete, e quindi l'intensità del campo magnetico che genera). \'{E} interessante notare che le più avanzate teorie sulla formazione dell'universo predicono l'esistenza di monopoli che classicamente non sono previsti e che per di più non sono mai stati osservati, questo è un problema tutt'ora aperto nella fisica.\\
Trascuriamo ora i campi variabili nel tempo
\[\oint_\Gamma \mathbf{B}\mathbf{dr} = \mu_0 i_{tot} = \mu_0(i + i_M)=\mu_0 i +\mu_0\oint_\Gamma \mathbf{M}\mathbf{dr}\]
\[\oint_\Gamma\left(\frac{\mathbf{B}}{\mu_0}-\mathbf{M}\right)\mathbf{dr} = i\]
Abbiamo trovato una quantità la cui circuitazione equivale solamente dalla corrente di conduzione, in analogia a quanto fatto per il vettore di spostamento elettrico \(\mathbf{D}\), il cui flusso dipendeva solo dalle cariche libere, definiamo il vettore \(\mathbf{H}\), detta \textbf{densità di flusso magnetico}. 
\[\mathbf{H}\equiv \frac{\mathbf{B}}{\mu_0}- \mathbf{M}\]
\[\oint_\Gamma\mathbf{H}\mathbf{dr} = i\]
\[\n\wedge\mathbf{H}=\mathbf{J}_c\]
dove \(\mathbf{J}_c\) è il vettore densità di corrente di conduzione, che non include la densità di corrente di magnetizzazione \(\mathbf{J}_m\).
In una calamita, chiaramente, non essendoci corrente di conduzione, la circuitazione di \(\mathbf{H}\) è nulla, ci sono solo correnti di magnetizzazione. La circuitazione di \(\mathbf{B}\) non è nulla per una calamita. Osserviamo che \(\mathbf{H}\) all'esterno della calamita vale
\[\mathbf{H}_{ext} = \frac{\mathbf{B}_0}{\mu_0}\]
dove \(\mathbf{B}_0\) è il campo magnetico all'esterno de materiale considerato. Quest'ultima formula si giustifica intuitivamente poichè \(\mathbf{M}\) fuori dalla materia è nullo, non essendoci atomi che si magnetizzano. Mentre all'interno della materia abbiamo
\[\mathbf{ H }= \frac{\mathbf{B}}{\mu_0}- \mathbf{M}\] 
ma se la circuitazione è nulla e all'esterno vi è solo un contributo positivo, ne segue che all'esterno il vettore \(\mathbf{H}\) ha direzione opposta rispetto all'esterno.\\
Si osserva sperimentalmente che per materiali isotropi vale
\[\mathbf{M} = \chi_m\mathbf{H}\]
dunque \(\mathbf{M}\) ha stessa direzione di \(\mathbf{H}\) in materiali isotropi ma può avere verso concorde o discorde a seconda del segno di \(\chi_m\). Quanto appena detto vale solamente per materiali paramagnetici e diamagnetici. Continuando a studiare le proprietà di \(\mathbf{H}\) svolgiamo alcune considerazioni a partire dalla definizione
\[\mathbf{B} = \mu_0 (\mathbf{H}+\mathbf{M}) = \mu_0 (1 +\chi_m)\mathbf{H} = \mu_0 k_m\mathbf{H}=\mu \mathbf{H}\]
\[\Rightarrow \mu\mathbf{M} = \chi_m \mathbf{B}\]
Per la maggior parte dei materiali vi è una relazione lineare tra \(\mathbf{M}\) ed \(\mathbf{H}\), come visto, ma per i ferromagnetici questo non vale. Per mantenere un'analogia con i paramagnetici ed i diamagnetici si mantiene la dipendenza lineare ma si fa diventare \(\chi_m = \chi_m(\mathbf{H})\). Per i paramagnetici \(\mathbf{M}\) ed \(\mathbf{H}\) hanno verso concorde, per i diamagnetici discorde mentre per i ferromagnetici dipende da come si è arrivati a quel valore di H.
\subsection{Circuiti magnetici}
in analogia ai circuiti elettrici, in cui è presente un campo elettrico in un conduttore, esistono anche circuiti magnetici in cui è presente un campo magnetico in un mezzo 
\begin{definizione}[Circuito magnetico]
	Un circuito magnetico è un circuito in cui si induce un campo magnetico, solitamente all'interno di un mezzo ferromagnetico, per mezzo di opportuni avvolgimenti percorsi da corrente elettrica o per mezzo di uno o più magneti permanenti.
\end{definizione}
Se la costante di permeabilità magnetica assoluta \(\mu\) del mezzo è abbastanza elevata il campo magnetico al suo interno non si disperde all'esterno, ne segue che il campo indotto da un'avvolgimento è costante su tutto il circuito ed è equivalente che gli avvolgimenti siano presenti su tutto il circuito o solo in una parte.
\begin{figure}[h!]
	\centering
	\includegraphics[width=0.6\linewidth]{../images/circuito-magnetico}
	\caption{Schematizzazione di un classico circuito magnetico.}
	\label{fig:circuito-magnetico}
\end{figure}
\FloatBarrier
Le leggi fondamentali che regolano un circuito magnetico sono due
\begin{align*}
	&\mathcal{F}\equiv\oint_{\Gamma}\mathbf{H}\mathbf{dl} = N i\\
	&\mathbf{B} = \mu\mathbf{H}
\end{align*}
dove la prima equazione definisce la \textbf{forza magnetomotrice}.\\
Se non viene disperso campo magnetico all'esterno si ha che il flusso di \(\mathbf{B}\) è costante dunque 
\[\Phi(B) = \oint_{\Sigma}\mathbf{B}\mathbf{ds} = costante\]
Mettendo insieme quanto visto possiamo scrivere
\[Ni = \oint_{\Gamma}\mathbf{H}\mathbf{dl} = \oint_{\Gamma}\frac{B}{\mu}\mathbf{dl}=\oint_{\Gamma}\frac{\Phi(B)}{\Sigma\mu}\mathbf{dl} =\frac{\Phi(B)}\oint_{\Gamma}\frac{\mathbf{dl}}{\Sigma\mu}\]
dove nell'ultimo passaggio abbiamo portato il flusso fuori dall'integrale in quanto costante. Definiamo \textbf{riluttanza del circuito}
\[\mathcal{R}\equiv\frac{\mathbf{dl}}{\Sigma\mu} \]
da cui segue
\[\mathcal{F}=\Phi(B)\mathcal{R}\]
che è detta \textbf{legge di Hopkinson}. Bisogna però osservare in queste leggi, essendo basate sulla presenza di un materiale ferromagnetico, presuppongono il fatto che la $\mu$ considerata è quella corrispondente al punto definito dai valori specifici (H,B) sul ciclo di isteresi. Ciò vuol dire che a rigore in un circuito magnetico le incognite sono 3: H, B, $\mu$. Tuttavia l'utilità dello studio dei circuiti magnetici sta proprio nella possibilità di calcolare H a partire da B o vice versa. Ne segue che nei circuiti magnetici si approssima $\mu$ a costante, ciò è possibile solo se il materiale ferromagnetico è dolce ed ha un ciclo di isteresi molto stretto. 
\subsubsection*{Il traferro}
Spesso nelle applicazioni pratiche è utile la presenza di un campo magnetico di valore noto e controllabile all'esterno della materia, per questo si pratica una sezione nel materiale ferromagnetico, detta \textbf{traferro}, di lunghezza s piccola rispetto alla lunghezza l del mezzo ferromagnetico e della sua dimensione trasversa. Date le piccole dimensioni, il campo magnetico non si disperde nel tratto di vuoto quindi il campo interno ed esterno sono gli stessi. Possiamo vedere la riluttanza in modo analogo alla resistenza (matematicamente hanno la stessa forma dunque si applicano le stesse formule), e sommare in serie le riluttanze del mezzo ferromagnetico e del traferro
\[\mathcal{R} = \mathcal{R}_1 + \mathcal{R}_2 = \frac{l}{\mu \Sigma}+\frac{s}{\mu_0\Sigma} = \frac{l+k_m s}{\mu_0 k_m\Sigma}\]
Notiamo che la riluttanza del circuito magnetico aumenta notevolmente se è presente un traferro, essendo il campo magnetico ottenibile con una data corrente nell'avvolgimento inversamente proporzionale alla riluttanza, questo diminuisce notevolmente in presenza di un traferro.\\
Nel traferro \(\mathbf{B}\) è costante ma \(\mathbf{H}\) non lo è infatti
\[\mathbf{H} = \frac{B}{\mu}\quad \mathbf{H}_0 = \frac{B}{\mu_0}\]
\[\Rightarrow \mathbf{H_0} = \mathbf{H}k_m\]
\subsection{Classificazione e proprietà magnetiche dei materiali}
Abbiamo già visto la classificazione dei materiali in base al loro valore di $\chi_m$ o equivalentemente di \(k_m\). Questi materiali presentano diversi comportamenti in presenza di un campo magnetico esterno e le cause sono da ricercare in ambito microscopico, basandoci su quanto appena visto dal punto di vista generale.
\subsubsection{Paramagnetici}
In questo caso gli atomi che vengono coinvolti nella magnetizzazione sono in numero esiguo dunque la massima magnetizzazione totale raggiunta è piccola.
Il caso paramagnetico è il più intuitivo in quanto segue senza particolarità l'interpretazione microscopica della magnetizzazione precedentemente esposta.\\ 
Consideriamo un atomo all'interno di un materiale paramagnetico, schematizzabile in prima approssimazione come una carica negativa (l'elettrone) che si muove in orbita circolare attorno ad una carica positiva (il protone) ferma, la traiettoria che segue l'elettrone è analoga ad una spira con corrente variabile a cui associamo il momento magnetico \(\mathbf{m}\). Se consideriamo un campo magnetico \(\mathbf{B}\) che inizialmente forma un angolo $\theta$ con il vettore \(\mathbf{m}\), abbiamo già visto che la spira (in questo caso l'atomo) sperimenta un momento delle forze che la fa allineare al campo magnetico. Se inizialmente la distribuzione delle direzioni dei momenti di magnetizzazione era isotropa e la magnetizzazione del materiale era globalmente nulla, una volta "acceso" il campo magnetico, a causa dell'allineamento dei momenti magnetici, la magnetizzazione globale non sarà più nulla, questa avrà stessa direzione e verso del campo magnetico e sarà debole in quanto i materiali ferromagnetici presentano un numero esiguo di elettroni abbastanza debolmente debolmente legati da potersi orientare liberamente, per questo la magnetizzazione di questi materiali è esigua.\\
Si osserva sperimentalmente una dipendenza inversa tra suscettibilità magnetica e temperatura nei materiali paramagnetici, ciò accade perché aumentando la temperatura aumenta anche l'agitazione termica degli atomi, in questo modo diventa più difficile allineare elettroni e quindi più difficile accrescere il momento magnetico globale del materiale.
La relazione tra suscettibilità e temperatura per materiali paramagnetici è data dall'espressione
\[\chi_m (T) = C \frac{\rho}{T}\]
dove C è una costante, $\rho$ è la densità e T la temperatura. Nei paramagnetici qualche elettrone si allinea generando una debole attrazione, nei ferromagnetici si allineano la maggior parte degli elettroni producendo una grande attrazione, ciò è dovuto al fatto che i materiali ferromagnetici sono solitamente metalli o materiali di raggio atomico grande in cui sono presenti elettroni poco legati al nucleo che possono muoversi più liberamente. 
\subsubsection{Ferromagnetici}
In questo tipo di materiali, tipicamente i materiali metallici, gli elettroni sono in gran parte debolmente legati grazie alla presenza delle bande di conduzione e di valenza, per questo motivo la magnetizzazione raggiungibile è molto maggiore che nel caso paramagnetico o diamagnetico (quasi tutti gli elettroni sonno orientati nel verso del campo magnetico).\\  
Il caso ferromagnetico è particolare dal punto di vista sperimentale perché accade che la magnetizzazione permanga anche dopo aver spento il campo magnetico esterno e perché il segno e il modulo della magnetizzazione dipendono dalla storia della variazione di \(\mathbf{H}\) ovvero di come si è arrivati a quel valore. Consideriamo un materiale ferromagnetico inizialmente smagnetizzato in un solenoide, all'aumentare della corrente di conduzione nel solenoide aumenta in modulo il vettore \(\mathbf{H}\), graficando \(\mathbf{B}\) ed \(\mathbf{M}\) al variare di H otteniamo 
\begin{figure}[h!]
	\centering
	\includegraphics[width=0.6\linewidth]{../images/isteresi_M}
	\quad
	\includegraphics[width=0.6\linewidth]{../images/isteresi_B}
	\caption{Ciclo di isteresi rispettivamente con M e B sulle ordinate.}
	\label{fig:isteresim}
\end{figure}
\FloatBarrier
Concentriamoci sul grafico H-M: all'aumentare di H (che per ipotesi inizialmente è nullo) inizialmente (curva di prima magnetizzazione) la magnetizzazione aumenta, seguendo una curva non lineare, fino a raggiungere la magnetizzazione di saturazione, valore su cui si stabilizza asintoticamente, ciò avviene perché arrivati a questo punto tutti gli spin sono allineati con il campo magnetico. Se arrivati a questo punto si diminuisce H si nota sperimentalmente che non si segue la stessa curva precedente a ritroso ma se ne segue una distinta, che decresce più lentamente, detta curva di demagnetizzazione; per questo motivo quando H è nullo vi sarà ancora una magnetizzazione residua che si azzera solo se H è minore di zero, ad un valore detto H coercitivo. Continuando a diminuire il valore di h si raggiunge un minimo. Se ora torniamo ad aumentare H riscontreremo che si segue un'ulteriore curva detta di seconda magnetizzazione. Risulta chiaro che la relazione non solo non è lineare ma dipende dalla "storia" della variazione di H: ad uno stesso valore di H sono associati più valori di M a seconda che ci troviamo su una curva o su un'altra. Per il grafico del campo magnetico la situazione è simile con la principale differenza che in corrispondenza del punto in cui M diventa costante B comincia a crescere linearmente poiché
\[\mathbf{B} = \mu_0\mathbf{M}_{sat}+\mu_0\mathbf{H}\]
se M è costante questa diventa l'equazione di una retta. Anche per \(\mathbf{B}\) nella curva di demagnetizzazione c'è un valore residuo tale che
\[\mathbf{B}_{res} = \mu_0\mathbf{M}_{res}\] 
Basandosi sul ciclo di isteresi è possibile magnetizzare un materiale con qualsiasi magnetizzazione compresa tra il massimo e il minimo (caratteristici del materiale): è sufficiente cominciare a ridurre \(\mathbf{H}\) prima di giungere al massimo di \(\mathbf{M}\) e, in seguito, aumentarlo prima di raggiungere il minimo e continuare così restringendo sempre gli intervalli
\begin{figure}[h!]
	\centering
	\includegraphics[width=0.4\linewidth]{../images/loop_isteresi}
	\caption{Variando h in intervalli sempre più piccoli è possibile raggiungere qualsiasi valore di M compreso fra minimo e massimo.}
	\label{fig:loopisteresi}
\end{figure}
\FloatBarrier
Si noti che la smagnetizzazione, di cui abbiamo esperienza ad esempio con le carte di credito, avviene semplicemente quando il magnete all'interno della carta, a causa di un prolungato contatto con un'altro magnete, come quello di un cellulare, raggiunge lo zero nel grafico di cui sopra.\\
Ciò che avviene in ogni ciclo più stretto è che sempre più spin si disordinano fino ad arrivare a magnetizzazione zero asintoticamente quando sono tutti disordinati. Il motivo per cui non tutti gli spin cambiano orientamento in una volta sola è che nei materiali ferromagnetici gli spin sono disposti con spin uguali in regioni di materiale dette \textbf{domini di Weiss} giustificabili con la meccanica quantistica; ad ogni ciclo si cambia orientamento di un erto numero di domini
\begin{figure}[h!]
	\centering
	\includegraphics[width=0.6\linewidth]{../"images/domini di weiss"}
	\caption{Rappresentazione della variazione di orientamento degli spin nei domini di Weiss per un materiale ferromagnetico sottoposto ad una \(\mathbf{H}\) crescente.}
	\label{fig:domini-di-weiss}
\end{figure}
\FloatBarrier
Ogni materiale ha un ciclo di isteresi specifico, li distinguiamo in stretti se il valore della \(\mathbf{H}_c\) di coercizione è piccolo mentre larghi se è grande, se il ferromagnete è stretto è anche facilmente smagnetizzabile e viene detto "materiale dolce" (usati ad esempio negli elettromagneti) se invece è largo sarà anche difficile da smagnetizzare, sono usati per creare magneti permanenti.\\
La trattazione appena svolta per i materiali ferromagnetici è valida solo in un range di temperature oltre il quale ogni materiale ferromagnetico smette di essere tale diventando paramagnetico perché a causa dell'agitazione termica non è più possibile orientare la maggior parte degli spin. Detta \(T_c\) la temperatura critica, o di Curie, oltre la quale il materiale perde le caratteristiche ferromagnetiche, esiste la relazione
\[\frac{\chi_m(T-T_c)}{\rho} = C\] 
dove C è una costante. 
\subsubsection{Diamagnetici}
Consideriamo un atomo, schematizzabile in prima approssimazione come un atomo di idrogeno in cui è presente solo una carica negativa che orbita circolarmente attorno ad un nucleo carico positivamente. Conoscendo il valore del raggio dell'atomo allo stato fondamentale e del tempo che impiega ad effettuare una rivoluzione
\[R_0 \simeq 0.5\cdot 10^{-10}\ m \quad T_0 \simeq 1.5\cdot 10^{-16}s\]
possiamo vederlo come una spira circolare in cui transita una corrente di
\[I_e = \frac{q}{T_0} \simeq 10^{-3}\ A\]
e di momento di magnetizzazione
\[\mathbf{m} = I_e \mathbf{S} = I_e \pi R_0^2\mathbf{\hat{u}_n} = \frac{e}{T_0}\pi R_0\mathbf{\hat{u}_n}\]
dove \(\mathbf{\hat{u}_n}\) è il versore normale al piano su cui ruota l'elettrone, nonché alla superficie S.  Visto che l'elettrone orbita possiede anche un momento angolare \(\mathbf{L}\) che ha stessa direzione di \(\mathbf{m}\)
\[\mathbf{L} = \mathbf{R_0}\wedge m_e\mathbf{v_0} = R_0 m_e v_0 \mathbf{\hat{u}_n} = \frac{2\pi R_0^2 m_e}{T_0}\mathbf{\hat{u}_n} \]
Possiamo ricavare la relazione tra \(\mathbf{L}\) ed \(\mathbf{m}\) sostituendo
\[\mathbf{L} = -\frac{2 m_e}{e}\mathbf{m}\]
deduciamo che \(\mathbf{L}\) ed \(\mathbf{m}\) hanno stessa direzione e verso opposto.\\
Immergendo ora l'atomo in un campo magnetico la cui direzione forma un angolo $\theta$ con a direzione di \(\mathbf{L}\) ed \(\mathbf{m}\), l'atomo-spira risente di un momento delle forze pari a 
\[\tau = \mathbf{m}\wedge \mathbf{B} = \frac{d\mathbf{L}}{dt} = -\frac{e}{2m_e} \mathbf{L}\wedge \mathbf{B}\]
Visto che per definizione il prodotto scalare fra due vettori risulta in un vettore ortogonale ad entrambe quelli di partenza, ne risulta che la derivata di \(\mathbf{L}\) è ortogonale ad \(\mathbf{L}\), ciò vuol dire che non ha componenti sull'asse \(\mathbf{\hat{u}_n}\) e quindi non può \(\mathbf{L}\) ma solo sulla sua direzione
\[\mathbf{L} = L \mathbf{\hat{u}}(t)\]
Usando le leggi di Poisson per la derivazione dei versori
\[\frac{d\mathbf{L}}{dt} = \omega_L \wedge \mathbf{L}\]
dove \(\omega_L\) è la velocità angolare con cui ruota il vettore \(\mathbf{L}\), si verifica dunque un moto di precessione dell'elettrone.\\
Uguagliando con la precedente espressione della derivata del momento angolare otteniamo
\[-\frac{e}{2m_e} \mathbf{L}\wedge \mathbf{B} = \frac{e}{2m_e} \mathbf{B}\wedge\mathbf{L}\]
\[\Rightarrow \omega_L = \mathbf{B}\frac{e}{2 m_e}\]
Questo moto, detto \textbf{precessione di Larmor} genera una corrente atomica di Larmor
\[i_e = \frac{e}{T_L}= e\left(\frac{\omega_L}{2\pi}\right)=\frac{Be^2}{4\pi m_e}\]
a cui è associato un momento magnetico di Larmor
\[\mathbf{m_e} = i_e \mathbf{S}\]
che è negativo poiché \(e = -|e|\). Questo momento magnetico dà origine al comportamento diamagnetico, ci spieghiamo il debole respingimento proprio per il segno negativo del momento di magnetizzazione che genera una forza repulsiva. 
\subsection{Legge di rifrazione del campo magnetico}
Consideriamo una superficie di separazione fra due mezzi di costanti di permeabilità magnetica relativa \(k_1\) e \(k_2\) ed un'onda elettromagnetica che incide sulla superficie 1 con angolo $\theta_1$ e viene rifratta con angolo $\theta_2$. A partire dalle equazioni già ricavate
\[\oint_{\Sigma(\Gamma)} \mathbf{B} \mathbf{ds} = 0\]
\[\oint_\Gamma \mathbf{H} \mathbf{dr} = 0\]
possiamo ricavare, analogamente a quanto fatto per il campo elettrico, la legge di rifrazione del campo magnetico. Consideriamo inizialmente un circuito rettangolare tale che uno dei due lati sia immerso nel mezzo 1 e l'altro nel mezzo 2, inoltre consideriamo la lunghezza dei lati verticali tendente a zero, la circuitazione di \(\mathbf{H}\) è costituita solamente dalla componente tangente di separazione perché quella normale si annulla con il prodotto scalare della circuitazione
\[\oint_\Gamma \mathbf{H} \mathbf{dr} = \oint_\Gamma H_{//} dr = l B_1\sin(\theta_1)\mu_2 -  l B_2\sin(\theta_2)\mu_1 = 0\]
\[\Rightarrow \frac{B_{1//}}{\mu_1} = \frac{B_{2//}}{\mu_2}\]
\[B_{//1}k_2 =  B_{//2}k_1\]
Considerando ora il flusso del campo magnetico, consideriamo un cilindro di altezza \(dh\to 0\) che attraversa la superficie di discontinuità, la componente di flusso per la superficie laterale è irrilevante essendo l'altezza tendente a zero, il flusso è quindi
\[\oint_\Sigma(\Gamma)\mathbf{B} = \mathbf{B}_1\mathbf{S} +  \mathbf{B}_1\mathbf{S} = S(B_1\cos\theta_1 -B_2\cos\theta_2)=S(B_{\perp 1}- B_{\perp 2})=0 \]
\[\Rightarrow B_{\perp 1} = B_{\perp 2}\]
Mettendo assieme le due relazioni ricavate
\begin{align*}
	&\begin{cases}
		B_{//1}k_2 =  B_{//2}k_1\\
		B_{\perp 1} = B_{\perp 2}
	\end{cases}\\
	&\begin{cases}
		B_{1}\sin\theta_1 k_2 =  B_{2}\sin\theta_2 k_1\\
		B_{1}\cos\theta_1 = B_{2}\cos\theta_2
	\end{cases}\\
&\Rightarrow \frac{\tan\theta_1}{\tan\theta_2} =\frac{k_1}{k_2}
\end{align*}
che è la stessa trovata per il campo elettrico (entrambi i campi sono onde quindi si comportano allo stesso modo).
\newpage
\section{Teorema di Poynting: conservazione dell'energia}
Abbiamo ricavato le equazioni (\ref{eq:densità_energia_elettrica}) e (\ref{eq:densità_energia_magnetica}) a partire dai casi particolari di un condensatore a facce piane parallele e di un solenoide per poi assumerle come generali senza dimostrarlo. Vogliamo ora ricavarle rigorosamente ed in generale ottenere un'espressione dell'energia scambiata da un generico sistema di cariche con velocità \(\mathbf{v}\) e carica q immerso in un generico campo elettrico e magnetico nell'unità di tempo (potenza). Scriviamo tutto ciò che sappiamo su questo sistema a partire dalla teoria sviluppata fin'ora
\[\mathbf{F} = q\mathbf{E}+q\mathbf{v}\wedge\mathbf{B}\]
\[\n\mathbf{E} = \frac{q}{\varepsilon_0}\]
\[\n\wedge\mathbf{E} = -\frac{\d \mathbf{B}}{\d t}\]
\[\mathbf{J} = n e \mathbf{v} = \rho \mathbf{v}\]
\[\n\mathbf{B} = 0\]
\[\n\wedge\mathbf{B} = \mu_0\mathbf{J}+\mu_0\varepsilon_0\frac{\d\mathbf{E}}{\d t}\]
Consideriamo ora una porzione di volume infinitesimo \(d\tau\) della distribuzione di carica
\[\mathbf{dF} = dq\mathbf{E} + dq\mathbf{v}\wedge\mathbf{B} = (\rho\mathbf{E} + \rho \mathbf{v}\wedge\mathbf{B})d\tau\]
la forza moltiplicata scalarmente per la velocità risulta nella potenza scambiata dal volumetto $d\tau$
\[\mathbf{dP} = (\rho\mathbf{v}\mathbf{E} + \rho (\mathbf{v}\wedge\mathbf{B})\mathbf{v})d\tau =  \rho\mathbf{v}\mathbf{E}d\tau = \mathbf{J}\mathbf{E}d\tau\]
dove il contributo offerto dal campo magnetico si annulla perché per definizione di prodotto vettoriale il vettore risultante è ortogonale a \(\mathbf{v}\) che, moltiplicato per \(\mathbf{v}\) si annulla. Si noti che questa è la formula dell'effetto Joule (\ref{eq:effetto_Joule}).\\
Ci chiediamo da cosa sia prodotta questa energia scambiata nell'unità di tempo, vogliamo ricavare un'espressione di \(\mathbf{J}\mathbf{E}\), usiamo la legge di Ampere-Maxwell e la moltiplichiamo scalarmente per \(\mathbf{E}\)
\[(\n\wedge\mathbf{B})\mathbf{E} = \mu_0(\mathbf{J}\mathbf{E}) + \mu_0\varepsilon_0 \frac{\d \mathbf{E}}{\d t}\mathbf{E}\]
\[\mathbf{J}\mathbf{E} = \frac{(\n\wedge\mathbf{B})\mathbf{E}}{\mu_0}-\varepsilon_0\frac{\d \mathbf{E}}{\d t}\mathbf{E}\]
Osservando che
\[\n(\mathbf{E}\wedge\mathbf{B}) = (\n\wedge\mathbf{E})\mathbf{B}-(\n\wedge\mathbf{B})\mathbf{E}\]
\[(\n\wedge\mathbf{B})\mathbf{E} = (\n\wedge\mathbf{E})\mathbf{B}- \n(\mathbf{E}\wedge\mathbf{B})\]
per la legge di Faraday possiamo scrivere
\[(\n\wedge\mathbf{B})\mathbf{E} = -\frac{\d B}{\d t}\mathbf{B}- \n(\mathbf{E}\wedge\mathbf{B})\]
sostituiamo
\[\mathbf{J}\mathbf{E} = -\varepsilon_0 \frac{\d \mathbf{E}}{\d t} = -\frac{1}{\mu_0}\left[\frac{\d B}{\d t}\mathbf{B}+\n(\mathbf{E\wedge\mathbf{B}})\right]\]
Osserviamo che 
\[\frac{\d B^2}{\d t} = \frac{\d }{\d t}(\mathbf{B}\mathbf{B}) = \mathbf{B}\frac{\d B}{\d t}+\frac{\d B}{\d t}\mathbf{B}=2\frac{\d B}{\d t}\mathbf{B}\]
\[\frac{\d E^2}{\d t} = 2\frac{\d E}{\d t}\mathbf{E}\]
sostituendo
\[\mathbf{J}\mathbf{E} = -\frac{1}{2}\varepsilon_0\frac{\d E^2}{\d t}-\frac{1}{2\mu_0}\frac{\d B^2}{\d t}-\frac{1}{\mu_0}\n(\mathbf{E}\wedge\mathbf{B})\]
dunque la potenza è
\[dW = d\tau (\mathbf{J}\mathbf{E})\]
per ottenere l'energia scambiata nell'unità di tempo sull'intera distribuzione di carica integriamo sul volume
\[W = \int_V\left[ -\frac{1}{2}\varepsilon_0\frac{\d E^2}{\d t}-\frac{1}{2\mu_0}\frac{\d B^2}{\d t} \right] d\tau-\frac{1}{\mu_0}\int_V \n(\mathbf{E}\wedge\mathbf{B})d\tau \]
Possiamo applicare il teorema della divergenza per il secondo addendo 
\[\int_V \n(\mathbf{E}\wedge\mathbf{B}) d\tau = \oint_{\Sigma(\Gamma)}(\mathbf{E}\wedge\mathbf{B})\mathbf{ds}\]
Infine otteniamo
\[W = -\frac{\d }{\d t}\left[\int_V \frac{1}{2}\varepsilon_0 E^2+\frac{1}{2\mu_0}B^2 \right] d\tau-\frac{1}{\mu_0}\oint_{\Sigma(\Gamma)}(\mathbf{E}\wedge\mathbf{B})\mathbf{ds}\]

questo è il \textbf{teorema di Poynting} (dal fisico britannico John Henry Poynting (1852 – 1914)), che altro non è che il principio di conservazione dell'energia in elettromagnetismo. Questo infatti ci dice che l'aumento (diminuzione) di energia nell'unità di tempo di un sistema di cariche è accompagnata da una diminuzione (emissione) di energia in tre modalità, che corrispondono ai tre addendi della legge di Poynting:
\begin{itemize}
	\item Il primo, che corrisponde alla densità di energia elettrica precedentemente ottenuta nel caso particolare del condensatore a facce piane parallele, e che ora abbiamo dimostrato avere validità generale, è legato alla presenza di cariche all'interno della distribuzione di carica che producono campo elettrico che porta con sè energia. 
	\item Il secondo, che corrisponde alla densità di energia magnetica precedentemente ottenuta nel caso particolare del solenoide cilindrico infinito, e che ora abbiamo dimostrato avere validità generale, è legato al moto della distribuzione di carica che produce campo magnetico che porta con sè energia. 
	\item Il terzo, che ha la forma di un flusso per la superficie chiusa della distribuzione di carica, tiene conto di una possibile energia, dovuta ad un campo elettrico o magnetico, proveniente dall'esterno della distribuzione di carica e passante per essa. \'{E} possibile dimostrare che quest'ultimo termine equivale all'energia trasportata da un'onda elettromagnetica. data l'importanza fisica di questa grandezza, ridefiniamo l'argomento dell'integrale del terzo termine, detto \textbf{vettore di Poynting}
	\[\mathbf{S}_p \equiv \frac{1}{\mu_0}(\mathbf{E}\wedge\mathbf{B})\]
\end{itemize}
L'onda elettromagnetica trasporta energia, come un corpuscolo puntiforme di massa m e quantità di moto \(\mathbf{p}\), di dimostra che anche le onde elettromagnetiche trasportano quantità di moto (derivante dal dualismo onda-corpuscolo)
\[\mathbf{dp} = \mu_0\varepsilon_0 \mathbf{S}_p d\tau\]
Un semplice esperimento che dimostra che anche la luce (e quindi le onde elettromagnetiche) hanno quantità di moto è la girandola solare, costituita da alcune pale colorate da un lato di nero (assorbe la luce) e dall'altro riflettenti. Visto che la riflessione della luce fa variare la quantità di moto della pala dal lato riflettente ma non da quello nero si crea una differenza di quantità di moto che mette in moto le pale facendole girare. 
\newpage
\section{Potenziale vettore}
Osserviamo che nonostante campo elettrico e magnetico presentino tante simmetrie ed analogie, resta una forte discrepanza: il campo elettrico possiede un potenziale (\(V(\mathbf{x})\)) mentre il campo magnetico non ha potenziale e non può averlo scalare perché abbiamo osservato che è non conservativo (per definizione un campo conservativo possiede un potenziale scalare). Tuttavia ciò non esclude la possibilità dell'esistenza di un potenziale che sia un campo vettoriale, ci proponiamo di trovare la forma analitica del potenziale vettore del campo magnetico.\\
Dalle equazioni di Maxwell abbiamo due proprietà  del campo magnetico, una ci offe informazioni sulla divergenza ed una sul rotore, sappiamo che il rotore non è nullo mentre la divergenza è sempre nulla. Sfruttiamo questa seconda proprietà, sempre valida, per ricavare un'espressione del potenziale vettore. Sia \(\mathbf{A}\) il potenziale vettore del campo magnetico, osserviamo che se lo definiamo mediante la relazione
\[\mathbf{B} = \n \wedge \mathbf{A}\]
la proprietà suddetta è sempre rispettata perché la divergenza del rotore è identicamente nulla. 
\[\n\mathbf{B} = \n(\n \wedge \mathbf{A})=0\]
Si osservi inoltre che in questo modo definiamo il potenziale vettore non solo a meno di una costante (analogamente al potenziale, a causa delle derivate della divergenza), ma anche a meno della divergenza di un campo scalare infatti
\[\mathbf{A'} = \mathbf{A} + \n f\]
\[\n\wedge\mathbf{A'} = \n\wedge\mathbf{A}+\n\wedge\n f = \n\wedge\mathbf{A} = \mathbf{B}\]
dunque sia \(\mathbf{A}\) sia \(\mathbf{A'}\) rispettano la relazione con cui sono state definite. Inoltre, applicando la divergenza si nota 
\[\n\mathbf{A'} = \n\mathbf{A}+\n^2 f = 0\]
\[\Rightarrow \n\mathbf{A} = - \n^2f\]
Cominciamo, per semplicità, con il definire il potenziale vettore, aggiungendo alla condizione \(\mathbf{B} = \n \wedge \mathbf{A}\) anche l'ipotesi \(\n A = 0\) da cui (\(\n^2f = 0\)) per semplificare i calcoli (torneremo su questo punto in seguito, per ora lo si accetti come una considerazione di tipo matematico). Inoltre, per semplicità, escludiamo la presenza di campi elettrici variabili.\\
Sfruttiamo l'altra informazione che abbiamo sul campo magnetico dalle equazioni di Maxwell
\[\n\wedge\mathbf{B} = \n\wedge(\n\wedge\mathbf{A}) = \n(\n A)-\n^2A = -\n^2 A = \mu_0\mathbf{J}\]
dove è stata usata la regola del BAC-CAB e il fatto che per ipotesi \(\n A = 0\). Ciò che abbiamo ottenuto è un'equazione differenziale alle derivate parziali del secondo ordine, trovarne la risoluzione è possibile ma difficile, sfruttiamo dunque un'analogia con il campo elettrico per semplificare i calcoli. Abbiamo già notato come la risoluzione del problema di Laplace, ovvero la risoluzione dell'equazione differenziale alle derivate parziali del secondo ordine
\[\n^V = -\frac{\rho}{\varepsilon_0}\]
è risolta in modo generale dall'equazione
\[V = \int_V \frac{\rho}{4\pi\varepsilon_0 r}d\tau\] 
Per analogia possiamo risolvere l'equazione differenziale del potenziale vettore come
\[\mathbf{A} = \int_V \frac{\mu_0 i}{4\pi r}d\tau\]
Abbiamo così trovato l'espressione del potenziale vettore sotto l'ipotesi di assenza di campi elettrici variabili nel tempo e \(\n A = 0\), cosa significa fisicamente quest'ultima ipotesi? 
\[\n A = \n \int \frac{\mu_0 \mathbf{J}}{4\pi r}d\tau = 0\]
\[\Rightarrow \n \mathbf{J} = 0  \]
dunque dire \(\n\mathbf{A} = 0\) equivale ad assumere il regime stazionario.\\
Possiamo infine ottenere la forma integrale del potenziale vettore 
\[\int_{\Sigma} \mathbf{B}\mathbf{ds} = \int_{Sigma} \n\wedge \mathbf{A}\mathbf{ds} = \oint_\Gamma(\Sigma) \mathbf{A}\mathbf{dl}\]
Possiamo anche riscrivere l'equazione di Maxwell
\[\n\wedge E = -\frac{\d \mathbf{B}}{\d t} = \frac{\d (-\n\wedge\mathbf{A})}{\d t} = \n\wedge\left(-\frac{\d \mathbf{A}}{\d t}\right)\]
Dunque il campo elettrico consta di una componente conservativa, che è la differenza di potenziale elettrico, e di una componente non conservativa, presente solo quando si hanno campi magnetici variabili nel tempo, che è uguale alla variazione nel tempo del potenziale vettore. Possiamo quindi scrivere un'espressione generale del campo elettrico come
\[\mathbf{E} = -\n V - \frac{\d \mathbf{A}}{\d t}\]
\section{La fisica è finita?}
Le equazioni di Maxwell permettono di studiare una quantità enorme di fenomeni prima sconosciuti le cui implicazioni tecnologiche ebbero un'impatto enorme sulla società della seconda metà dell' '800. Ciò avvenne grazie a (o a causa di) un periodo di pace che ormai durava in Europa da decenni, prosperità economica verificatasi grazie all'imposizione del ceto borghese nella società e in generale una incondizionata fiducia nel progresso e nella scienza testimoniata da movimenti filosofici come il positivismo o letterari e artistici come il realismo. Nel contesto di questa atmosfera si sviluppò un dibattito nella comunità scientifica attorno alla credenza che le principali scoperte scientifiche fossero ormai già state fatte e che ciò che mancava erano solamente conferme sperimentali e piccoli dettagli. L'ottimismo della comunità scientifica era già da tempo avversato da voci minori, ma non meno illustri, soprattutto di cultura umanistica, che criticavano le "magnifiche sorti e progressive" in cui la società dell'epoca credeva. Tuttavia una forte botta d'arresto a questo clima venne proprio dall'interno delle scienze, due problemi fondamentali erano stati lasciati aperti dalla teoria di Maxwell: il primo è che un elettrone in orbita attorno ad un nucleo (come veniva concepito dalla fisica classica), generando campo elettrico e magnetico, perde energia dunque non ci si spiega come faccia la materia stessa ad esistere poiché l'elettrone dovrebbe collassare sul nucleo. Inoltre dalle equazioni di Maxwell, come visto, risulta che le onde elettromagnetiche e quindi la luce hanno una velocità costante, la domanda è: "rispetto a quale sistema di riferimento?" inizialmente si pensò potesse essere il fantomatico etere ma le evidenze sperimentali (esperimenti di Michelson e Morley) confutarono questa ipotesi.\\
Dal primo di questi problemi nacque la fisica quantistica, dal secondo la relatività (Appendice \ref{ap:relatività}): la fisica moderna stava per nascere. 
\newpage
\appendix
\section{Operatori differenziali}\label{ap:operatori_differenziali}
\subsection{Teorema del gradiente}
\begin{definizione}[campo scalare]
	Si dice campo scalare da \(\mathbb{R}^3\) a \(\mathbb{R}\) una funzione del tipo 
	\[\varphi(x, y, z):\ \mathbb{R}^3\rightarrow \mathbb{R}\]
	Fisicamente, questa è una funzione che associa ad ogni punto dello spazio uno scalare (come pressione o temperatura).
\end{definizione}
\begin{definizione}[campo vettoriale]
	Si dice campo vettoriale da \(\mathbb{R}^3\) a \(\mathbb{R}^3\) una funzione del tipo 
	\[\varphi(x, y, z):\ \mathbb{R}^3\rightarrow \mathbb{R}^3\]
	Fisicamente, questa è una funzione che associa ad ogni punto dello spazio un vettore (come campo gravitazionale o elettrico).
\end{definizione}
Per derivare una funzione di questo tipo si introduce l'operatore differenziale \textbf{gradiente} e l'operatore \textbf{nabla}
\[\mathbf{\n} \equiv \left(\frac{\d}{\d x}, \frac{\d }{\d y}, \frac{\d }{\d z}\right)\]
\[\mathbf{grad}\varphi \equiv \mathbf{\n}\varphi = \left(\frac{\d\varphi }{\d x}, \frac{\d\varphi }{\d y}, \frac{\d\varphi }{\d z}\right)\]
Per invertire l'operazione di derivazione di un campo scalare consideriamo il teorema fondamentale del calcolo (valido per funzioni da \(\mathbb{R}\) ad \(\mathbb{R}\))
\[f(B)-f(A) = \int_{A}^{B}f'(x)dx\]
definiamo il differenziale di una funzione f come
\[df = f'(x)dx\] 
in modo da avere
\[f(B)-f(A) = \int_{A}^{B}df\]
Per estendere il concetto a più dimensioni, consideriamo due punti A e B ed un generico percorso L che li congiunge, il vettore tangente al percorso è $d\mathbf{l}$.  Definiamo il differenziale di un campo scalare come 
\[d\varphi = \mathbf{\n}\varphi d\mathbf{l}\]
Possiamo quindi integrare il differenziale e si dimostra che vale la seguente proprietà (estensione del teorema fondamentale del calcolo)
\[_L\int_{A}^{B}d\varphi = _L\int_{A}^{B}\mathbf{\n}\varphi d\mathbf{l} = \varphi(A)-\varphi(B)\]
questa relazione viene detta \textbf{teorema del gradiente}. Si noti che grazie a questo teorema abbiamo messo in relazione due oggetti di dimensione diversa: uno scalare (\(\varphi\)) di dimensione zero con una "linea", che è l'integrale di percorso da A a B, di dimensione uno.
\subsection{Teorema della divergenza}\label{ap:thm_divergenza}
Consideriamo ora una superficie S nello spazio da cui ne ritagliamo una porzione infinitesima ds; vogliamo che questo diventi un vettore, in analogia con $d\mathbf{l}$ visto in precedenza. Associamo quindi un vettore normale alla superficie (di cui ancora non abbiamo definito il verso, sarà fatto a breve). Abbiamo quindi
\[d\mathbf{s} = ds \hat{n}\]
Se fosse presente un campo vettoriale $\mathbf{F}$ possiamo definire un nuovo concetto:
\begin{definizione}[Flusso]
	Il flusso infinitesimo è la quantità di "linee di campo" che attraversano una superficie infinitesima. 
	\[d\Phi = \mathbf{F}d\mathbf{s} = Fds\cos\alpha\]
	Il flusso sulla superficie è 
	\[_S\iint d\Phi = \iint \mathbf{F}d\mathbf{s}\]
\end{definizione}
Quello appena visto è un integrale di superficie.\\
Prima di procedere è utile fare una digressione sulle superfici: queste possono essere aperte (come un foglio nello spazio) o chiuse (come un pallone). Nelle superfici chiuse in vettore normale ad una superficie è preso convenzionalmente "uscente" dalla superficie. Consideriamo ora una superficie chiusa come in figura
\begin{figure}[h!]
	\centering
	\includegraphics[width=0.6\linewidth]{../images/flusso2bis}
	\caption{Schema di una superficie chiusa (da immaginare tridimensionalmente) attraversata da un campo \(\mathbf{E}\).}
	\label{fig:flusso2bis}
\end{figure}
\FloatBarrier
Il flusso è 
\[\Phi(\mathbf{F}) = _S\oint \mathbf{F}d\mathbf{s}\]
dove il simbolo del doppio integrale cerchiato indica l'integrale di superficie chiusa.\\
Se tagliamo la superficie chiusa in due parti \(S_1\) ed \(S_2\), mediante una superficie trasversale G, l'integrale sull'intera superficie, per l'additività dell'integrale, è semplicemente la somma degli integrali di superficie su \(S_1\) ed \(S_2\).
\[\Phi(\mathbf{F}) = _{S_1}\iint \mathbf{F}d\mathbf{s}+_{S_2}\iint \mathbf{F}d\mathbf{s}\]
Possiamo aggiungere e sottrarre l'integrale di superficie aperte su G 
\[\Phi(\mathbf{F}) = \iint_{S_1} \mathbf{F}d\mathbf{s}+\iint_{S_2} \mathbf{F}d\mathbf{s}+\iint_{G} \mathbf{F}d\mathbf{s}-\iint_{G} \mathbf{F}d\mathbf{s}\]
Si noti che avendo diviso la superficie chiusa S in due mediante G, abbiamo ottenuto due nuove superfici chiuse se consideriamo \(S_1^* = S_1+G\) e \(S_2^* = S_2+G\). Il contributo che fornisce G nell'integrale delle due superfici chiuse è uguale in modulo ma di segno opposto, quindi possiamo unire a due a due gli addendi dell'espressione precedente ottenendo due integrali di superficie chiusa (il doppio o triplo integrale chiuso sarà indicato sempre con il simbolo "\(\oint\)")
\[\Phi(\mathbf{F}) = _{S_1^*}\oint \mathbf{F}d\mathbf{s}+_{S_2^*}\oint \mathbf{F}d\mathbf{s}\]
Possiamo reiterare questo processo di divisione n volte ottenendo
\[\Phi(\mathbf{F}) = \sum_{i=1}^{n}\ _{S_i^*}\oint \mathbf{F}d\mathbf{s}\]
Moltiplichiamo e dividiamo per il volume \(V_i\) sotteso dall'i-esima superficie chiusa
\[\Phi(\mathbf{F}) = \sum_{i=1}^{n}\left(\frac{_{S_i^*}\oint \mathbf{F}d\mathbf{s}_i}{V_i}\right)V_i\]
Definiamo la \textbf{divergenza} del campo vettoriale $\mathbf{F}$ 
\[div\mathbf{F} = \frac{_{S_i}\oint \mathbf{F}d\mathbf{s}_i}{V_i}\]
In un campo vettoriale, chiamiamo "pozzo" un punto in cui convergono le linee di campo e "sorgente" uno in cui divergono, . Intuitivamente la divergenza ci offre informazioni su quanto un punto sia "pozzo" o "sorgente". 
\begin{figure}[h!]
	\centering
	\includegraphics[width=0.6\linewidth]{../images/div}
	\caption{Se le linee di campo convergono la divergenza è negativa, se divergono positiva.}
	\label{fig:div}
\end{figure}
\FloatBarrier
Possiamo quindi sostituire e ottenere
\[\Phi(\mathbf{F}) = \sum_{i=1}^{n}div\mathbf{F}V_i\]
Se il numero di divisioni n tende a infinito possiamo passare al continuo 
\[\Phi(\mathbf{F}) = \oint div\mathbf{F}dV\]
quest'ultima relazione è conosciuta come teorema della divergenza o di Gauss. Si noti che la superficie presa in considerazione deve essere chiusa. si noti come in questo teorema mettiamo in relazione il flusso su una superficie con un integrale su un volume, aumentando di 1 la dimensione.\\
\subsubsection{Calcolo della divergenza}
Si consideri un cubo di volume infinitesimo di lati (dx, dy, dz) attraversato da un campo vettoriale $\mathbf{F}(x,y,z) = (F_x, F_y, F_z)$.  
\begin{figure}[h!]
	\centering
	\includegraphics[width=0.6\linewidth]{../images/div_quadrato}
	\caption{Schema per il calcolo della divergenza di un campo vettoriale su un cubo.}
	\label{fig:divquadrato}
\end{figure}
\FloatBarrier
La variazione di flusso sull'intero cubo è uguale alla somma dei flusso infinitesimi sulle singole facce. Considerando la faccia "su"
\[d\Phi_{su}(\mathbf{F}) = \mathbf{F}d\mathbf{s}_{su} = \mathbf{F}(x, y, z + dz)dxdy\hat{\mathbf{k}}=F_z(x, y, z + dz)dxdy\]
dove \(F_z\) è la proiezione di sull'asse z ottenuta dal prodotto scalare col versore \(\hat{\mathbf{k}}\). 
Analogamente per la faccia "giù" abbiamo una espressione analoga (in segno meno è dovuto alla convenzione di scegliere il versore normale come uscente dalla superficie chiusa)
\[d\Phi(\mathbf{F}) = -F_z(x, y, z)dxdy\]
Sommando i due flussi infinitesimi otteniamo la variazione infinitesima di flusso sull'asse z
\[d\Phi_z= d\Phi_{su}+d\Phi_{giu}=\left[F_z(x,y,z+dz)-F_z(x,y,z)\right]dxdy=\]
\[\frac{\left[F_z(x,y,z+dz)-F_z(x,y,z)\right]}{dz}dxdydz=\frac{\d F_z}{\d z}dxdydz\]
Ripetendo procedimenti analoghi per gli altri assi e sommando otteniamo
\[d\Phi = \left(\frac{\d F_z}{\d z}+ \frac{\d F_z}{\d z}+ \frac{\d F_z}{\d z}\right)dxdydz \]
\[\frac{d\Phi}{dV} = \left(\frac{\d F_z}{\d z}+ \frac{\d F_z}{\d z}+ \frac{\d F_z}{\d z}\right)\]
Ricordando le definizioni di divergenza di $\mathbf{F}$ e di \(\mathbf{\n}\) si ottiene
\[div\mathbf{F} = \frac{d\Phi}{dV} = \mathbf{\n}\mathbf{F}\]
Si noti che a differenza del gradiente in questo caso il vettore nabla è moltiplicato ad un altro vettore risultando in uno scalare (la divergenza è un operatore scalare definito come il prodotto scalare tra nabla ed un vettore)
\[\Rightarrow \Phi_S(\mathbf{F}) = \iiint \mathbf{\n}\mathbf{F}dV = \oint_{V(\Sigma)} \mathbf{\n}\mathbf{F}dV\]
che è un'altro modo per esprimere il teorema di Gauss.
\subsection{Teorema di Stokes (teorema del rotore)}
Si consideri la linea chiusa L, fissiamo un verso di percorrenza ed un vettore tangente ad essa $d\mathbf{l}= dl\hat{u}$. 
\begin{definizione}[Circuitazione]
	Definiamo circuitazione infinitesima la quantità
	\[d\Gamma(\mathbf{F}) = \mathbf{F}d\mathbf{l}\]
	La circuitazione è l'integrale di percorso della circuitazione infinitesima
	\[\Gamma(\mathbf{F}) = _L\oint \mathbf{F}d\mathbf{l}\]
\end{definizione}
Analogamente a quanto fatto per la superficie chiusa divisa da una superficie aperta trasversale, possiamo dividere la linea chiusa L mediante una corda G in \(L_1\) ed \(L_2\), e calcolare la circuitazione per le due nuove linee chiuse ottenute \(L_1^* = L_1+G\) e \(L_2^* = L_2+G\).
\[\Gamma_L(\mathbf{F}) = _{L_1^*}\oint \mathbf{F}d\mathbf{l}+_{L_2^*}\oint \mathbf{F}d\mathbf{l} \]
Estendendo il ragionamento ad n divisioni otteniamo
\[\Gamma(\mathbf{F}) = \sum_{i=1}^{n}\ _{L_i}\oint\mathbf{F}d\mathbf{l}\]
Moltiplichiamo e dividiamo per la superficie \(s_i\) sottesa da ciascuna linea chiusa, per trasformarla in vettore, come accennato inizialmente, prendiamo il versore ad essa normale e scegliamo il verso per convenzione usando la regola della mano destra, prendendo come riferimento il verso di percorrenza della linea al contorno.
\[\Gamma(\mathbf{F}) = \sum_{i=1}^{n}\ \left( \frac{_{L_i}\oint\mathbf{F}d\mathbf{l}}{s_i}\right)s_i\]
moltiplichiamo per \(\hat{n}\cdot\hat{n}=1\)
\[\Gamma(\mathbf{F}) = \sum_{i=1}^{n}\ \left( \frac{_{L_i}\oint\mathbf{F}d\mathbf{l}}{s_i}\right)\hat{n}\mathbf{s}_i\]
Definiamo, analogamente a quanto fatto per la divergenza, un nuovo operatore differenziale: il \textbf{rotore}
\[\mathbf{rot}\mathbf{F} = \frac{_{L_i}\oint\mathbf{F}d\mathbf{l}}{s_i}\hat{n}\]
\[\Gamma(\mathbf{F}) = \sum_{i=1}^{n}\mathbf{rot}\mathbf{F}\mathbf{s}_i \]
Se il numero di divisioni tende a infinito passiamo al continuo ottenendo
\[_L\oint\mathbf{F}d\mathbf{l} =\Gamma(\mathbf{F}) = _{S_L}\iint(\mathbf{rot}\mathbf{F})d\mathbf{s}_i =_{S_L}\oint(\mathbf{rot}\mathbf{F})d\mathbf{s}_i\]
Questa relazione è conosciuta come il teorema del rotore o, più frequentemente, come il \textbf{teorema di Stokes}. Si noti come questo metta in relazione un integrale di linea (una dimensione) con un integrale di superficie (due dimensioni).\\\\
In sintesi, per evidenziare un'analogia, i tre teoremi esposti mettono in relazione le dimensioni 0-1 (gradiente), 2-3 (divergenza) ed 1-2 (rotore).
\subsubsection{Calcolo del rotore}
Come per la divergenza, la definizione con cui è stato introdotto il rotore non è operativa: cerchiamo una espressione operativa calcolando la circuitazione di un quadrato infinitesimo ABCD di lati dx e dy scegliendo il senso di percorrenza antiorario. Il quadrato è disposto nello spazio in modo da avere versore coincidente con $\hat{\mathbf{k}}$. 
\begin{align*}
	&\Gamma_L(\mathbf{F}) = _L\oint \mathbf{F} d\mathbf{l} = \int_{A}^{B} \mathbf{F} d\mathbf{l}+\int_{B}^{C} \mathbf{F} d\mathbf{l}+\int_{C}^{D} \mathbf{F} d\mathbf{l}+\int_{D}^{A} \mathbf{F} d\mathbf{l}\\
	&= \int_{x}^{x+dx} F_x(x,y) dx+\int_{y}^{y+dy} F_y(x+dx,y) dy+\int_{x+dx}^{x} F_x(x, y+dy) dx+\int_{y+dy}^{y} F_y(x, y) dy
\end{align*}  
applichiamo quindi il teorema della media integrale che in formule enuncia
\[\int_{A}^{B}f(x)dx=(B-A)f(\overline{x})\]
\[\Rightarrow\Gamma_L(\mathbf{F}) = dxF_x(\overline{x}, y)+dyF_y(x+dx, \overline{y})-dxF_x(\overline{x}, y+dy)-dyF_y(x, \overline{y})\]
\[=-dx[F_x(\overline{x}, y+dy)-F_x(\overline{x}, y)]+dy[F_y(x+dx, \overline{y})-F_y(x, \overline{y})]\]
\[=-dxdy\left[\frac{\d F_x}{\d y}\right]+dxdy\left[\frac{\d F_y}{\d x}\right] = dxdy\left[\frac{\d F_y}{\d x}-\frac{\d F_x}{\d y}\right]=\oint\mathbf{F}d\mathbf{l}\]
Ricordando come avevamo definito il rotore, otteniamo 
\[(\mathbf{rot}\mathbf{F})\hat{\mathbf{k}} = \left[\frac{\d F_y}{\d x}-\frac{\d F_x}{\d y}\right]\]
Per le componenti sugli altri assi il ragionamento è del tutto analogo. Notiamo che le tre componenti del rotore appena ottenute si possono ricavare semplicemente calcolando il determinante di una matrice che sulle righe ha rispettivamente i versori, l'operatore di derivata parziale sui tre assi e le componenti del campo vettoriale. In conclusione
\begin{align*}
&\mathbf{rot}\mathbf{F}=\mathbf{\n}\wedge\mathbf{F}=det
\begin{pmatrix}
	\hat{\mathbf{i}}&\hat{\mathbf{j}}&\hat{\mathbf{k}}\\
	\frac{\d}{\d x}&\frac{\d}{\d y}&\frac{\d}{\d z}\\
	F_x&F_y&F_z
\end{pmatrix}=\\
&(\frac{\d F_z}{\d y}-\frac{\d F_y}{\d z})\hat{\mathbf{i}}
-(\frac{\d F_z}{\d x}-\frac{\d F_x}{\d z})\hat{\mathbf{j}}
+(\frac{\d F_y}{\d x}-\frac{\d F_x}{\d y})\hat{\mathbf{k}}
\end{align*}
Si dimostra infine che 
\[div(\mathbf{rot}(\mathbf{F})) =\mathbf{\n}(\mathbf{\n}\wedge\mathbf{F})=0\]
Le quattro condizioni necessarie e sufficienti (equivalenti) perché un campo si dica conservativo sono
\begin{itemize}
	\item \(\oint \mathbf{F}d\mathbf{l} = 0 \)
	\item \(\mathbf{\n}\wedge\mathbf{F} = 0\)
	\item \(\exists\varphi \text{campo scalare} : \mathbf{F}=-\mathbf{\n}\varphi\)
	\item \(\varphi(B)-\varphi(A) = -\int_{A}^{B}\mathbf{\n}\varphi d\mathbf{l}\ \forall L\)
\end{itemize}
\begin{align*}
\frac{c_1(c_2+c_4)}{c_1+c_2+c_3+c_4}-\frac{c_1c_2}{c_1+c_2}
\end{align*}
\newpage
\section{Elettromagnetismo e relatività}\label{ap:relatività}
Consideriamo due cariche uguali \(q_1\) \(q_2\) di segno positivo a distanza R in un S.R.I. O in quiete rispetto all'osservatore, che ipotizziamo sia solidale allo stesso S.R.I. delle cariche. Sappiamo che la forza agente su \(q_1\) è
\[\mathbf{F}=q_1 \mathbf{E_2}= \frac{1}{4\pi\varepsilon_0}\frac{q_1q_2}{R^2}\mathbf{\hat{j}}\]
Consideriamo ora lo stesso sistema rispetto ad un osservatore solidale ad un S.R.I. O' che si muove con velocità costante \(\mathbf{v}_2\) rispetto ad O. Per un osservatore solidale ad O' il sistema in O ha velocità \(\mathbf{v_1}=-\mathbf{v_2}= -\mathbf{v}\) secondo la relatività galileiana. Osserviamo immediatamente un fatto curioso: mentre nel primo caso era presente solo un campo elettrico, ora abbiamo anche un campo magnetico dovuto al movimento delle cariche, calcoliamo quello generato da \(q_2\). 
\[\mathbf{B_2}= \frac{\mu_0q_2}{4\pi}\frac{\mathbf{v}\wedge\mathbf{r}}{r^3}=\frac{\mu_0q_2}{4\pi}\frac{v}{r^2}\mathbf{\hat{u}}\]
La forza sulla carica \(q_1\) è data dalla somma della forza elettrica e magnetica
\[\mathbf{F'}= \frac{1}{4\pi\varepsilon_0}\frac{q_1q_2}{R^2}\mathbf{\hat{j}}+q_1\mathbf{v}\wedge\mathbf{B_2}=\frac{1}{4\pi\varepsilon_0}\frac{q_1q_2}{R^2}\mathbf{\hat{j}}+\frac{\mu_0q_1q_2}{4\pi}\frac{v}{r^2}\mathbf{\hat{u}}\]
per semplicità consideriamo la velocità \(\mathbf{v}\) con componente solo sull'asse y di modo che \(\mathbf{\hat{u}}= - \mathbf{\hat{j}}\)
\[\mathbf{F'}= \frac{1}{4\pi\varepsilon_0}\frac{q_1q_2}{R^2}\mathbf{\hat{j}}-\frac{\mu_0q_1q_2}{4\pi}\frac{v}{r^2}\mathbf{\hat{j}}\frac{\varepsilon_0}{\varepsilon_0}=\frac{q_1q_2}{4\pi\varepsilon_0 r^2}\left(1-\frac{v^2}{c^2}\right)\mathbf{\hat{j}}\]
dove è stato usato \(\mu_0\varepsilon_0 = \frac{1}{c^2}\). 
\[\Rightarrow \mathbf{F'}=\mathbf{F}\left(1-\frac{v^2}{c^2}\right)\]
otteniamo il sorprendente risultato che la forza dipende dalla velocità relativa del S.R.I. da cui è osservato il sistema! Questo risultato profondamente contro intuitivo è difficile da accettare a partire dalla cornice concettuale della fisica classica. Ciò ci porta a mettere in dubbio o la correttezza delle teorie dell'elettromagnetismo sviluppate fin ora o la relatività galileiana (entrambe accettate ed in uso da centinaia di anni al tempo in cui questi calcoli vennero sviluppati). Si arriverà, con la teoria della relatività, alla conclusione che entrambi sono sbagliati e che anche il risultato da noi ottenuto lo è (poichè abbiamo usato come presupposto la relatività galileiana che è sbagliata). In realtà la formula sarebbe
\[\mathbf{F'}=\mathbf{F}\sqrt{1-\frac{v^2}{c^2}}\]
Ma come risposero i fisici dell'epoca a questi risultati (dal momento che ancora la teoria della relatività non era stata sviluppata)? Semplicemente aggiunsero come postulato la seguente proposizione
\[\mathbf{E}=\mathbf{E}'+\mathbf{v}\wedge\mathbf{B}'\]
cioè il campo elettrico del sistema in quiete è uguale a quello del sistema in moto più un contributo dato dal campo magnetico.\\
Da questa cui deriva
\[\mathbf{B}=\mathbf{B}'-\frac{1}{c^2}\mathbf{v}\wedge\mathbf{E}\]
e di conseguenza
\[\mathbf{F}=\mathbf{F}'\]
In sintesi, i fisici di fine '800 preferirono salvare la relatività galileiana, tanto intuitiva e semplice da sembrare assurdo metterla in discussione, a patto di aggiungere un postulato che mette in luce il fatto che \(\mathbf{B}\) ed \(\mathbf{E}\) sono strettamente correlati. In realtà questo postulato è falso, per risolvere questo problema si scoprirà che spazio e tempo sono un'unica cosa, detta spaziotempo, che possiede 4 (e non 3) dimensioni. La forza quindi si conserva nei diversi sistemi di riferimento, ma solo nello spazio a 4 dimensioni; limitandosi alle tre dimensioni è ovvio il fatto che non si conservi perché stiamo considerando solo 3 delle 4 componenti del vettore forza! Inoltre risulta che la forza di cui abbiamo fin ora parlato non è elettrica o magnetica ma elettromagnetica, in quanto i due fenomeni sono due facce della stessa medaglia. Le leggi che regolano campo elettrico, campo magnetico e Forza elettromagnetica, alla luce delle correzioni relativistiche, sono
\begin{align*}
	&\begin{cases}
		E'_x=E_x \\
		E'_y=\frac{1}{\sqrt{1-\frac{v^2}{c^2}}}[E_y-vB_z]\\
		E'_z=\frac{1}{\sqrt{1-\frac{v^2}{c^2}}}[E_z+vB_y]
	\end{cases}\\
	&\begin{cases}
		B'_x=B_x \\
		B'_y=\frac{1}{\sqrt{1-\frac{v^2}{c^2}}}[B_y+\frac{v}{c^2}E_z] \\
		B'_z=\frac{1}{\sqrt{1-\frac{v^2}{c^2}}}[B_z-\frac{v}{c^2}E_y]
	\end{cases}\\
	&\begin{cases}
		F'_x=F_x \\
		F'_y=F_y\sqrt{1-\frac{v^2}{c^2}} \\
		F'_z=F_z\sqrt{1-\frac{v^2}{c^2}}
	\end{cases}	
\end{align*}
\end{document}